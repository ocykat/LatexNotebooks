\chapter{Euclidean Space}

\hi{Inner Product}
  \hii{Inner Product}
    \hiii{Notation}
      $\iprod{x, y}$: inner product of $x$ and $y$
    \hiii{Definition}
      \par Let $V$ be a real vector space. An inner product in $V$ is a real
        function:
        \begin{eqbox}
          \begin{array}{cccc}
            f: & V \times V & \longrightarrow & R \\
               & (x, y)     & \longrightarrow & \iprod{x, y}
          \end{array}
        \end{eqbox}
        with 4 conditions:
        \begin{itemize}
          \item $\forall y \in V, \iprod{x, y} \geq 0 \mbox{ and }
            \iprod{x, x} = 0 \ra x = 0$
          \item $\forall x, y \in V, \iprod{x, y} = \iprod{y, x}$
          \item $\forall x, y \in V, \forall \alpha \in \mathbb{R},
            \iprod{\alpha x, y} = \alpha \iprod{x, y}$
          \item $\forall x, y, z \in V,
            \iprod{x + y, z} = \iprod{x, z} + \iprod{y, z}$
        \end{itemize}

  \hii{Euclidean Space}
    \par A real finite dimensional vector space with inner products is called
      an \tb{Euclidean space}.
    \begin{smfont}
      \par True definition: A real finite dimensional vector space with inner
        products is called an \tb{inner product} space. An \tb{Euclidean space}
        is an inner product space in which each inner product is a dot product.
        In other words, an Euclidean space is one case of inner product space.
    \end{smfont}

  \hii{Quantities}
    \begin{itemize}
      \item The length of a vector $x$:
        \begin{eqbox}
          ||x|| = \sqrt{\iprod{x, x}}
        \end{eqbox}
      \item The distance between two vectors $x$ and $y$:
        \begin{eqbox}
          dist(x, y) = \|x - y\| = sqrt{\iprod{x - y, x - y}}
        \end{eqbox}
      \item The angle between two vectors $x$ and $y$:
        \begin{eqbox}
          \cos(\alpha) = \frac{\iprod{x, y}}{\|x\|\|y\|}
        \end{eqbox}
    \end{itemize}

  \hii{Numeric Calculation}
    \par Given two inner vectors:
      \begin{align*}
        x = (x_{1}, x_{2}, \ldots, x_{n}) \\
        y = (y_{1}, y_{2}, \ldots, y_{n})
      \end{align*}
    and their inner product:
      \begin{align*}
        \iprod{x, y} = m_{11} x_{1}y_{1}
                     + m_{12} x_{1}y_{2}
                     + \ldots
                     + m_{1n} x_{1}y_{n}
                     + m_{21} x_{2}y_{1}
                     + \ldots
                     + m_{nn} x_{n}y_{n}
      \end{align*}
    \par We can organize all coefficient $m_{ij}$ into a matrix $M$:
    \begin{align*}
      M = \begin{pmatrix}
        m_{11} & m_{12} & \ldots & m_{1n} \\
        m_{21} & m_{22} & \ldots & m_{2n} \\
        \vdots & \vdots & \vdots & \vdots \\
        m_{n1} & m_{n2} & \ldots & m_{nn} \\
      \end{pmatrix}
    \end{align*}
    \par The inner product of $x$ and $y$ can be written as:
    \begin{eqbox}
      \iprod{x, y} = xMy^{T}
    \end{eqbox}
    where $x$ and $y$ are both row matrices.


\hi{Orthogonal complement of a subspace}
  \hii{Definitions}
    \par Let $V$ be an Euclidean space. Given two vectors $x, y \in V$.
      \begin{enumerate}
        \item $\iprod{x, y} = 0 \ra x \bot y$ ($x$ is orthogonal to $y$)
        \item $\forall f \in M: x \bot f \ra x \bot M$
          ($x$ is orthogonal to $M$)
      \end{enumerate}
    \par Let $F$ be a subspace of $V$. The orthogonal complement of $F$:
      \begin{align*}
        F^{\bot} = \{\forall v \in V | v \bot F\}
      \end{align*}
  \hii{Theorems}
    \hiii{Theorem 1}
      \par A vector $v$ is orthogonal to a subspace $F$ if and only if the
        vector $v$ is orthogonal to a spanning set of $F$.
        \begin{align*}
          v \bot F \lra v \bot S_{F}
        \end{align*}
        where $S_{F}$ is a spanning set of $F$.
    \hiii{Theorem 2}
      \par Let $F$ be a subspace of $V$. These statements hold true:
        \begin{enumerate}
          \item $F^{\bot}$ is a subspace of $V$.
          \item $F^{\bot} \cap F = {O}$ (the origin)
          \item $F + F^{\bot} = V$
          \item $dim(F) + dim(F^{\bot}) = dim(V)$
          \item $\forall v \in V:
                  \; !\exists f \in F, \; !\exists g \bot F: v = f + g$.
            \begin{itemize}
              \item The vector $f$ is called an \tb{orthogonal projection}
                of $v$ onto $F$ and is denoted by $proj_{F}v = f$.
              \item The length of the vector $g$ is called the distance from $v$
                to $F$ and is denoted by $dist(v, F) = \|g\| = \|v - f\|$
            \end{itemize}
        \end{enumerate}
  \hii{Problems}
    \hiii{Find a basis of $F^{\bot}$ and $dim(F^{\bot}$}
      \begin{itemize}
        \item \tb{Step 1}: Find a spanning set $S_{F}$ of $F$.
          \begin{flalign*}
            S_{F} = \{f_{1}, f_{2}, \ldots, f_{m}\}
          \end{flalign*}
          \par \smf{Usually, a spanning set is enough. But to simplify the
            next step, it is best to find a basis. To find a basis, try to
            simplify the equations until obtaining the least number of
            vectors.}
        \item \tb{Step 2}: Find a basis of $F^{\bot}$:
%          \begin{flalign*}
            %& \forall x \in F^{\bot}: x \bot $F$ && \\
            %& \ra \forall x \in F^{\bot}: x \bot S_{F} && \\
            %& \ra
            %\begin{cases}
              %x \bot f_{1} \\
              %x \bot f_{2} \\
              %\ldots       \\
              %x \bot f_{m}
            %\end{cases}
            %\ra
            %\begin{cases}
              %\iprod{x, f_{1}} = 0 \\
              %\iprod{x, f_{2}} = 0 \\
              %\ldots               \\
              %\iprod{x, f_{m}} = 0 \\
            %\end{cases} &&
          %\end{flalign*}
          \par Solve the system, find a general solution $\ra$ basis and
            dimension of $F^{\bot}$.
      \end{itemize}
