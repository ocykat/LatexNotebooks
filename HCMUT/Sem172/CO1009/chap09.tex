\chapter{MSI Circuits}

\hi{Decoder}
  \hii{Introduction}
    \par A decoder is a logic circuit that accepts a set of inputs that
      represents a binary number and activates only the output that
      corresponds to that input number. All other outputs remain inactive.

  \hii{Inputs}
    \par For a decoder which has $N$ inputs, it may not utilize all $2^{N}$
      possible input codes but only certain ones. (BCD-to-decimal decoder is
      one example). For decoders of this type, if any of the unused code is
      applied to the input, none of the outputs will be activated.

  \hii{Referring decoder}
    \par Some examples:
    \begin{itemize}
      \item 3-line-to-8-line decoder: 3 input lines, 8 output lines
      \item binary-to-octal decoder: convert a 3bit binary input and activates
        one of the eight outputs corresponding to that code.
      \item 1-of-8 decoder: only 1 of the 8 outputs is activated at one time
    \end{itemize}

  \hii{ENABLE Inputs}
    \par Some decoders have one or more ENABLE inputs to control the decoder.

  \hii{BCD-to-Decimal Decoders - Driver}
    \par Difference between a decoder and a driver: the driver has
      \footnotemark{open-collector} outputs that can operate at higher current
      and voltage limits than a normal decoder output.
      \footnotetext{Open-collector output: the output is connected to the base (B)
        of a NPN transistor. If B is triggered, then there is electric current going
        from collector (C) to emitter (E).}

\hi{Encoder}
  \hii{Introduction}
    \par An \tb{encoder} has a number of input lines, only \tb{one} of which is
    activated at a given time, and produces an $N$-bit output code, depending on
    which input is activated.
    \par Example: octal-to-binary encoder.
  \hii{Priority Encoder}
    \par Drawback of simple encoder: when more than one input is activated at one time,
      the output is unexpected.
    \par A \tb{priority encoder} ensure that when two or more inputs are activated, the
    output code will correspond to the highest-numbered input.
    \par Example: 74147 Decimal-to-BCD Priority Encoder
  \hii{Switch Encoder}
  \par Example: 74147: switch priority decoder
    \par \tb{Switch encoder} is used whenever BCD data must be entered manually into a
      digital system.
    \par When a key is pressed, the BCD code for the corresponding digit is sent to a
      four-bit FF register;
    \par When another key is pressed, the BCD code for the corresponding digit is sent to
      \ti{another} four-bit FF register;
    \par A calculator that can handle eight digits will have eight four-bit registers to
    store the BCD. Each four-bit register drives a decoder/driver and a numerical display. 

\hi{Multiplexers (Data selectors)}
  \hii{Introduction}
    \par A \tb{digital multiplexer} or \tb{data selector} is a logic circuit that accepts
      several digital data inputs and selects one of them at any given time to pass on to
      the output.
    \par The routing of the desired data input is controlled by SELECT (ADDRESS) inputs.
  \hii{Two-Input Multiplexer}
  \hii{Quad Two-Input MUX (74ALS157)}
  \hii{Four-Input Multiplexer}
    \par Another approach is to use tristate buffers.
    \par The decoder is used to make sure that only one input is selected.
  \hii{Eight-Input Multiplexer}
    \par Example: 74ALS151

\hi{Demultiplexers (Data Distributors)}
  \hii{Introduction}
    \par A \tb{demultiplexer} takes a single input and distributes it over several outputs.
    \par 1 DATA input signal is transmitted to \tb{only 1 output} at the time. The output is
      selected by SELECT inputs.
  \hii{1-Line-to-8-Line Demultiplexer}
    \par A single data input line $I$ is connected to all eight AND gates.
    \par At any moment, only one AND gate is enabled.
    \par Data input $I$ will appear at output of the only enabled AND gate.
