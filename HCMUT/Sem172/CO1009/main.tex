\documentclass[12pt, a4paper]{report}


% MATH
% + packages
\usepackage{amsmath}
\usepackage{amssymb}
\usepackage{relsize}
% + macros
\newcommand{\SUM}[1]{\sum\limits #1}
\newcommand{\PROD}[1]{\prod\limits #1}
\newcommand{\INT}[1]{\int\limits #1}


% EQUATION BOXES
% + package
\usepackage{empheq}
% + environment
\newenvironment{eqbox}{%
    \setkeys{EmphEqEnv}{align}
        \setkeys{EmphEqOpt}{box=\fbox}
            \EmphEqMainEnv%
}{%
    \endEmphEqMainEnv%
}


% MARGINS
\usepackage[margin=0.75in]{geometry}


% HEADINGS
% + macros
\newcommand{\hi}{\section}
\newcommand{\hii}{\subsection}
\newcommand{\hiiBEGIN}[1]{\subsection{#1} \begin{enumerate}}
\newcommand{\hiiEND}{\end{enumerate}}
\newcommand{\hiii}{\item\textbf}
\newcommand{\hiiiBEGIN}[1]{\item\textbf{#1} \begin{enumerate}}
\newcommand{\hiiiEND}{\end{enumerate}}
\newcommand{\hiv}{\item\textbf}
% + set counter depth to 4 (e.g. 1.1.1.1)
\setcounter{secnumdepth}{4}


% TABLE OF CONTENT
% + link for TOC
\usepackage{hyperref}
\hypersetup{
    colorlinks,
    citecolor=black,
    filecolor=black,
    linkcolor=black,
    urlcolor=black
}

% TABULAR
\newcommand{\specialcell}[2][c]{%
  \begin{tabular}[#1]{@{}c@{}}#2\end{tabular}}


% INDENTATION
% + indent the first paragraph after a heading
\usepackage{indentfirst}


% TEXT
% Bold, italic, underlined text
\newcommand{\tb}[1]{\textbf{#1}}
\newcommand{\ti}[1]{\textit{#1}}
\newcommand{\tbi}[1]{\textbf{\textit{#1}}}
\newcommand{\tu}[1]{\underline{#1}}
\newcommand{\tbu}[1]{\textbf{\underline{#1}}}


% FOOTNOTE
% + one footnote stays on one page
\interfootnotelinepenalty=10000


% APPENDICES
\usepackage[toc,page]{appendix}


% IMAGES
\usepackage{graphicx}
\usepackage{subcaption}
\usepackage{float}
\graphicspath{ {img/} }


% TIKZ
\usepackage{tikz}


% FLOWCHART
% + Tikz library
\usetikzlibrary{shapes.geometric}
\usetikzlibrary{arrows}
% + style
\tikzstyle{rect}    = [rectangle, draw=black, text centered,
                       minimum width=3cm, minimum height=1cm]
\tikzstyle{arrow}   = [thick, ->, >=stealth]
\tikzstyle{twarrow} = [thick, <->, >=stealth]
\tikzstyle{darrow}  = [dashed, ->, >=stealth]


% LOGIC CIRCUIT
\usetikzlibrary{shapes.gates.logic.US, calc}
\tikzstyle{branch}  = [fill, shape=circle, minimum size=3pt, inner sep=0pt]
\tikzstyle{AND}     = [and gate US, draw]
\tikzstyle{OR}      = [or gate US, draw]
\tikzstyle{XOR}     = [xor gate US, draw]
\tikzstyle{NOT}     = [not gate US, draw]
\tikzstyle{NAND}    = [nand gate US, draw]
\tikzstyle{NOR}     = [nor gate US, draw]
\newcommand{\inp}[1]{logic gate inputs=#1}


% KARNAUGH MAP
\usepackage{karnaugh-map}

% CODE
\usepackage{listings}


\begin{document}

\tableofcontents

\chapter{The Role of Algorithms in Computing}
\chapter{Number Systems and Codes}

\hi{Conversions}
    \hi{$m$-ary to Decimal}
        \par To convert a $m$-ary number, which has $n$ digits in the integer part and $p$ digits
        in the fractional part, to a decimal number:
        \begin{equation*}
            D = a_{n} \cdot m^{n} + a_{n - 1} \cdot m^{n - 1} + ... +
            a_{0} \cdot m^{0} + ... +
            a_{(-p + 1)} \cdot m^{(-p + 1)} + a_{(-p)} \cdot m^{(-p)}
        \end{equation*}
    \hi{Decimal to $m$-ary}
        \par To convert a decimal number to $m$-ary, there are two steps to carry out:
        \begin{itemize}
            \item Convert the integer part.
            \item Convert the fractional part.
        \end{itemize}

\chapter{Interpolation}

\par \tb{Interpolation}: The process of predicting new data points based on
  a known set of data points.
\par This chapter considers methods of finding the function that fits the
  given data.

\hi{Interpolation and the Lagrange Polynomial}
  \hii{Algebraic Polynomial and Interpolation}
    \par One popular class of function for interpolation is \tb{algebraic
      polynomials}. A algebraic polynomial has the form:
    \begin{align*}
      P_{n}(x) = a_{n} x^{n} + a_{n - 1} x^{n - 1} + \ldots + a_{1}x + a_{0}
    \end{align*}
      where $n$ is a nonnegative integer and $a_{0}, a_{1}, \ldots, a_{n}$ are
      real constant.
    \par One important property of algebraic polynomial is that a polynomial $P$
      can get as ``close" to the given function $f$ as possible. This property
      is expressed in the Weierstrass Approximation Theorem.
    \par Another reason for using polynomials is their simplicity: the
      derivatives and integrals of polynomials are also polynomials and are easy
      to find.

  \hii{Weierstrass Approximation Theorem}
    \par \tb{Theorem}: Suppose that $f$ is defined and continuous on $[a, b]$.
      For each and every $\epsilon > 0$, there exists a polynomial $P(x)$ so
      that:

    \begin{align*}
      \abs{f(x) - P(x)} < \epsilon, \quad \forall x \in [a, b]
    \end{align*}


  \hii{Lagrange Interpolating Polynomials}
    \hiii{Example: Linear Lagrange Interpolation}
      \par \tb{Example}: Given two distinct points $(x_{0}, y_{0})$ and
        $(x_{1}, y_{1})$. Determine a function $f$ for which $f(x_{0}) = y_{0}$
        and $f(x_{1}) = y_{1}$.

      \par Define the following functions:

        \begin{align*}
          p_1^{(0)}(x) = \frac{x - x_1}{x_0 - x_1}
          \qquad \mbox{and} \qquad
          p_1^{(1)}(x) = \frac{x - x_0}{x_1 - x_0}
        \end{align*}

      \par The linear \tb{Lagrange interpolation polynomial} through
        $(x_0, y_0)$ and $(x_1, y_1)$ is:

        \begin{align*}
          \mathcal{L}_1(x) = y_0 p_1^{(0)}(x) + y_1 p_1^{(1)}(x)
        \end{align*}

      \par Note that:
        \begin{align*}
          \begin{cases}
            p_1^{(0)}(x_0) = 1 \qquad \mbox{and} \qquad p_1^{(1)}(x_0) = 0 \\
            p_1^{(0)}(x_1) = 0 \qquad \mbox{and} \qquad p_1^{(1)}(x_1) = 1
          \end{cases} \\
          \ra
          \begin{cases}
            \mathcal{L}_1(x_0) = y_0 p_1^{(0)}(x_0) + y_1 p_1^{(1)}(x_0) = y_0 \\
            \mathcal{L}_1(x_1) = y_0 p_1^{(0)}(x_1) + y_1 p_1^{(1)}(x_1) = y_1
          \end{cases}
        \end{align*}

        \par $\mathcal{L}_1(x)$ is called a linear Lagrange polynomial of the
          function $f(x)$.

    \hiii{nth Lagrange Interpolating Polynomial}
      \par \tb{Theorem}: If $x_0, x_1, \ldots, x_n$ are $n + 1$ distinct numbers
        and $y_0, y_1, \ldots, y_n$ are given, then a unique polynomial
        $\mathcal{L}_n$ exists that satisfies:
        \begin{itemize}
          \item $deg(\mathcal{L}_n) \leq n$
          \item $\mathcal{L}_n(x_k) = y_k \quad \forall k = 0, 1, \ldots, n$
        \end{itemize}
      \par The first step of finding $\mathcal{L}_n$ is to find the basic
        polynomials $p_n^{(k)}$ with $k = 0, 1, \ldots, n$ such that:
        \begin{itemize}
          \item $deg(p_n^{(k)}) \leq n$
          \item
            \begin{align*}
              p_n^{(k)}(x_i) =
              \begin{cases}
                1 \mbox{ if } i = k \\
                0 \mbox{ if } i \neq k
              \end{cases}
              \ra p_n^{(k)}(x)
                = \frac{(x - x_0) \ldots (x - x_{k - 1})(x - x_{k + 1}) \ldots (x - x_n)}
                       {(x_k - x_0) \ldots (x_k - x_{k - 1})(x_k - x_{k + 1}) \ldots (x_k - x_n)}
            \end{align*}
        \end{itemize}
      \par Then, we have:
        \begin{align*}
          \mathcal{L}_n(x) &= y_0 p_n^{(0)}(x)
                          + y_1 p_n^{(1)}(x) + \ldots
                          + y_n p_n^{(n)}(x) \\
                          &= \sum\limits_{k = 0}^n y_k p_n^{(k)} (x)
        \end{align*}

\hi{Newton's Polynomial}
  \hii{Initial Form}
    \par The Newton's Interpolating Polynomial $P_n(x)$ which agrees with the
      function $f$ at distinct numbers $x_0, x_1, \ldots, x_n$ has the form:
      \begin{align*}
        P_n(x) = a_0 + a_1(x - x_0) + a_2(x - x0)(x - x1) + \ldots
              + a_n(x - x_0)(x - x_1) \ldots (x - x_n)
      \end{align*}
  \hii{Divided Difference}
    \par \tb{Definition}: Divided Difference has a recursive definition:
      \begin{itemize}
        \item The zeroth divided difference of the function $f$ with respect
          to $x_i$, denoted $f[x_i]$, is the value of $f$ at $x_i$:
          \begin{align*}
            f[x_i] = f(x_i)
          \end{align*}
        \item The first divided difference of $f$ with respect to $x_i$ and
          $x_{i + 1}$, denoted $f[x_i, x_{i + 1}]$, is defined as:
          \begin{align*}
            f[x_i, x_{i + 1}] = \frac{f[x_{i + 1}] - f[x_i]}{x_{i + 1} - x_i}
          \end{align*}
        \item The second divided difference is defined as:
          \begin{align*}
            f[x_i, x_{i + 1}, x_{i + 2}]
              = \frac{f[x_{i + 1}, x_{i + 2}] - f[x_i, x_{i + 1}]}{x_{i + 2} - x_i}
          \end{align*}
        \item In general, the kth divided difference is:
          \begin{align*}
            f[x_i, x_{i + 1}, \ldots, x_{i + k}]
              = \frac{f[x_{i + 1}, x_{i + 2}, \ldots, x_{i + k}]
                    - f[x_i, x_{i + 1}, \ldots, x_{i + k - 1}]}
                     {x_{i + 2} - x_i}
          \end{align*}
      \end{itemize}

  \hii{Forms of Newton's Polynomial}
    \begin{itemize}
      \item Forward form:
        \begin{align*}
          \mathcal{N}_n^{(1)} = f[x_0] + f[x_0, x_1](x - x_0)
                            + f[x_0, x_1, x_2](x - x_0)(x - x_1)
                            + \ldots
                            + f[x_0, x_1, \ldots, x_n](x - x_0)(x - x_1)\ldots(x - x_{n - 1})
        \end{align*}
      \item Backward form:
        \begin{align*}
          \mathcal{N}_n^{(2)} = f[x_n] + f[x_{n - 1}, x_n](x - x_n)
                            + f[x_{n - 2}, x_{n - 1}, x_n](x - x_{n - 1})(x - x_n)
                            + \ldots
                            + f[x_0, x_1, \ldots, x_n](x - x_1)(x - x_2)\ldots(x - x_n)
        \end{align*}
    \end{itemize}

  \hii{Special Case: Uniform-Distanced Points}
    \hiii{Uniform-Distanced Points}
      \par We consider the case where
        $x_{k + 1} - x_{k} = h = const \quad \forall k = 0, 1, \dots, n - 1$.

    \hiii{Finite Differences}
      \par Finite Differences also have a recursive definition:
      \begin{itemize}
        \item $\Delta y_k = y_{k + 1} - y_k \quad \forall k = 0, 1, \ldots, n - 1$
          is called \tb{finite differences} of first order at $x_k$.
        \item $\Delta^2 y_k = \Delta y_{k + 1} - \Delta y_k \quad \forall k = 0, 1, \ldots, n - 2$
          is called \tb{finite differences} of second order at $x_k$.
        \item $\Delta^p y_k
          = \Delta^{p - 1} y_{k + 1} - \Delta^{p - 1} y_k \quad \forall k = 0, 1, \ldots, n - 2$
          is called \tb{finite differences} of pth order at $x_k$.
      \end{itemize}

      \par The relationship between Finite Differences and Divided Differences:
        \begin{align*}
          f[x_k, x_{k + 1}, \ldots, x_{k + p}] = \frac{\Delta^p y_k}{p!h^p}
        \end{align*}

    \hiii{Newton's Polynomial Forms with Finite Differences}
      \begin{itemize}
        \item Forward form:
          \begin{align*}
            \begin{aligned}
              \mathcal{N}_n^{(1)} = y_0 + \frac{\Delta y_0}{1!}q
                                  + \frac{\Delta^2 y_0}{2!}q(q - 1)
                                  + \ldots
                                  + \frac{\Delta^n y_0}{2!}q(q - 1)\ldots(q - n + 1) \\
              \mbox{ where } q = \frac{x - x_0}{h}
            \end{aligned}
          \end{align*}
        \item Backward form:
          \begin{align*}
            \begin{aligned}
              \mathcal{N}_n^{(2)} = y_n + \frac{\Delta y_n - 1}{1!}p
                                  + \frac{\Delta^2 y_0}{2!}p(p + 1)
                                  + \ldots
                                  + \frac{\Delta^n y_0}{2!}p(p - 1)\ldots(p + n - 1) \\
              \mbox{ where } p = \frac{x - x_n}{h}
            \end{aligned}
          \end{align*}
      \end{itemize}


\clearpage
\begin{multicols}{2}
\hi{The Processor}

  \hii{Control Signals}
    \begin{tabular}{|p{2cm}|p{3.5cm}|p{4cm}|}
    \hline
    \textbf{Signal}   & \textbf{0}                                                                    & \textbf{1}                                                       \\ \hline
    \textbf{RegDst}   & write register = rt{[}20:16{]} (I-format)                                     & write register = rd{[}15:11{]} (R-format \& beq)                 \\ \hline
    \textbf{RegWrite} & -                                                                             & control signal for writing data to register                      \\ \hline
    \textbf{ALUSrc}   & \begin{tabular}[c]{@{}c@{}}second ALU operand = rt\\  (R-format)\end{tabular} & second ALU operand = {[}15:0{]} (sign-extend)                    \\ \hline
    \textbf{PCSrc}    & PC = PC + 4                                                                   & PC = branch target                                               \\ \hline
    \textbf{MemRead}  & -                                                                             & Data are read from memory (lw) and put on DataMemory.ReadOutput  \\ \hline
    \textbf{MemWrite} & -                                                                             & Data are put on DataMemory.WriteInput and written in memory (sw) \\ \hline
    \textbf{MemToReg} & write register is fed from ALU                                                & write register is fed from DataMemory                            \\ \hline
    \end{tabular}

    \begin{tabular}{|c|c|c|c|c|c|}
    \hline
    \textbf{I/O}                      & \textbf{Instruction} & \textbf{R-format} & \textbf{lw} & \textbf{sw} & \textbf{beq} \\ \hline
    \multirow{6}{*}{\textbf{Inputs}}  & \textbf{Op5}         & 0                 & 1           & 1           & 0            \\ \cline{2-6} 
                                      & \textbf{Op4}         & 0                 & 0           & 0           & 0            \\ \cline{2-6} 
                                      & \textbf{Op3}         & 0                 & 0           & 1           & 0            \\ \cline{2-6} 
                                      & \textbf{Op2}         & 0                 & 0           & 0           & 1            \\ \cline{2-6} 
                                      & \textbf{Op1}         & 0                 & 1           & 1           & 0            \\ \cline{2-6} 
                                      & \textbf{Op0}         & 0                 & 1           & 1           & 0            \\ \hline
    \multirow{9}{*}{\textbf{Outputs}} & \textbf{RegDst}      & 1                 & 0           & X           & X            \\ \cline{2-6} 
                                      & \textbf{ALUSrc}      & 0                 & 1           & 1           & 0            \\ \cline{2-6} 
                                      & \textbf{MemToReg}    & 0                 & 1           & X           & X            \\ \cline{2-6} 
                                      & \textbf{RegWrite}    & 1                 & 1           & 0           & 0            \\ \cline{2-6} 
                                      & \textbf{MemRead}     & 0                 & 1           & 0           & 0            \\ \cline{2-6} 
                                      & \textbf{MemWrite}    & 0                 & 0           & 1           & 0            \\ \cline{2-6} 
                                      & \textbf{Branch}      & 0                 & 0           & 0           & 1            \\ \cline{2-6} 
                                      & \textbf{ALUOp1}      & 1                 & 0           & 0           & 0            \\ \cline{2-6} 
                                      & \textbf{ALUOp0}      & 0                 & 0           & 0           & 1            \\ \hline
    \end{tabular}

  \hii{States of Instruction Execution}
    \begin{itemize}
      \item \tb{IF}: Instruction fetch
      \item \tb{ID}: Instruction decode \& register file read
      \item \tb{EX}: Execution or address calculation
      \item \tb{MEM}: Data memory access
      \item \tb{WB}: Write back
    \end{itemize}

  \hii{Data Hazard}
    \hiii{Forwarding}
      \par \tb{Hazard conditions}:
\begin{multicols}{2}
      \begin{enumerate}[1.]
        \item EX hazard:
          \begin{enumerate}[a.]
            \item \lstinline{EX/Mem.Rd = ID/EX.Rs}
            \item \lstinline{EX/Mem.Rd = ID/EX.Rt}
          \end{enumerate}
        \item MEM hazard:
          \begin{enumerate}[a.]
            \item \lstinline{Mem/WB.Rd = ID/EX.Rs}
            \item \lstinline{Mem/WB.Rd = ID/EX.Rt}
          \end{enumerate}
      \end{enumerate}
\end{multicols}
      \par \tb{EX hazard}:
\begin{lstlisting}
if (EX/MEM.RegWrite && EX/MEM.Rd != 0 && EX/MEM.Rd == ID/EX.RegRs)
  ForwardA = 10
if (EX/MEM.RegWrite && EX/MEM.Rd != 0 && EX/MEM.Rd == ID/EX.RegRt)
  ForwardB = 10
\end{lstlisting}
      \par \tb{MEM hazard}:
\begin{lstlisting}
if (MEM/WB.RegWrite && MEM/WB.Rd != 0 && MEM/WB.Rd == ID/EX.RegRs)
  ForwardA = 01
if (MEM/WB.RegWrite && MEM/WB.Rd != 0 && MEM/WB.Rd == ID/EX.RegRt)
  ForwardB = 01
\end{lstlisting}

\begin{tabular}{|c|c|p{8cm}|}
\hline
\textbf{MUX Control}   & \textbf{Source} & \textbf{Explanation}                                                \\ \hline
\textbf{ForwardA = 00} & \textbf{ID/EX}  & 1st ALU operand comes from register file                            \\ \hline
\textbf{ForwardA = 10} & \textbf{EX/MEM} & 1s ALU operand is forwarded from the prior ALU result               \\ \hline
\textbf{ForwardA = 01} & \textbf{MEM/WB} & 1st ALU operand is forwarded from data memory or earlier ALU result \\ \hline
\textbf{ForwardB = 00} & \textbf{ID/EX}  & 2nd ALU operand comes from register file                            \\ \hline
\textbf{ForwardB = 10} & \textbf{EX/MEM} & 2nd ALU operand is forwarded from the prior ALU result              \\ \hline
\textbf{ForwardB = 01} & \textbf{MEM/WB} & 2nd ALU operand is forwarded from data memory or earlier ALU result \\ \hline
\end{tabular}

    \hiii{Stalling}
      \par Happen in only one case: \lstinline{lw} following by a read of the
        same register.
\begin{lstlisting}
if (ID/EX.MemRead && ((ID/EX.Rt == IF/ID.Rs) or (ID/EX.Rt == IF/ID.Rt)))
  stall pipeline
\end{lstlisting}

  \hii{Control Hazard}
    \par Solutions:
      \begin{itemize}
        \item Predict that branch not taken: flush instructions in the IF, ID,
          EX if branch is taken
        \item Move branch execution from MEM to ID to reduce delay
        \item Dynamic branch prediction: 1-bit/2-bit
      \end{itemize}
\end{multicols}



\begin{multicols*}{3}
  
\hi{Hypothesis Testing for One Sample}

\hii{Terminology}
  \begin{itemize}
    \item \tb{null hypothesis}:  the claim that is initially assumed to be true, denoted by $H_0$
    \item \tb{alternative hypothesis}: the assertion that is contradictory to $H_0$, denoted by $H_1$
    \item \tb{test of hypotheses}: a method for using sample data to decide \tb{whether the null hypothesis should be rejected}.
    \item \tb{significant level}: the probability of a type I error, denoted by $\alpha$
  \end{itemize}

\hii{Tips}
  \begin{itemize}
    \item The \tb{equal sign} only appears in the \tb{null hypothesis} $H_0$.
    \item The thing we want to prove appears in the \tb{alternative hypothesis} $H_1$. 
  \end{itemize}

\hii{Case 1: Normal Population with Known Standard Deviation}
  \hiii{Hypothesis Testing}
  \begin{itemize}
    \item \tb{Step 1}: Compute the statistic:
      \[
        z = \frac{\bar{x} - \mu_0}{\sigma / \sqrt{n}}
      \]
    \item \tb{Step 2}: Apply the decision rule:
      \begin{center}
        \begin{tabular}{|c|c|}
          \hline
          \textbf{$H_1$}   & \textbf{Rejection Region} \\ \hline
          $\mu \neq \mu_0$ & $|z| > z_{\alpha/2}$      \\ \hline
          $\mu < \mu_0$    & $z < -z_{\alpha}$         \\ \hline
          $\mu > \mu_0$    & $z > z_{\alpha}$          \\ \hline
        \end{tabular}
      \end{center}
  \end{itemize}

    \hiii{Hypothesis Testing with Propotion}
    \begin{itemize}
      \item \tb{Step 1}: Compute the statistic:
        \[
          z = \frac{\hat{p} - p_0}{\sqrt{\hat{p}\hat{q}/n}}
        \]
      \item \tb{Step 2}: Apply the decision rule:
        \begin{center}
          \begin{tabular}{|c|c|}
            \hline
            \textbf{$H_1$}   & \textbf{Rejection Region} \\ \hline
            $p \neq p_0$ & $|z| > z_{\alpha/2}$      \\ \hline
            $p < p_0$    & $z < -z_{\alpha}$         \\ \hline
            $p > p_0$    & $z > z_{\alpha}$          \\ \hline
          \end{tabular}
        \end{center}
    \end{itemize}

\hii{Case 2: Normal Population with Unknown Standard Deviation}

\begin{itemize}
  \item \tb{Step 1}: Compute the statistic:
    \[
      t = \frac{\bar{x} - \mu_0}{s / \sqrt{n}}
    \]
  \item \tb{Step 2}: Apply the decision rule:
      \begin{center}
        \begin{tabular}{|c|c|}
          \hline
          \textbf{$H_1$}   & \textbf{Rejection Region}   \\ \hline
          $\mu \neq \mu_0$ & $|t| > t_{\alpha/2, n - 1}$ \\ \hline
          $\mu < \mu_0$    & $t < -t_{\alpha, n - 1}$    \\ \hline
          $\mu > \mu_0$    & $t > t_{\alpha, n - 1}$     \\ \hline
        \end{tabular}
      \end{center}
\end{itemize}

\hii{Case 3: Any Distribution - Large Sample Size}

\begin{itemize}
  \item \tb{Step 1}: Compute the statistic:
    \[
      z = \frac{\bar{x} - \mu_0}{s / \sqrt{n}}
    \]
  \item \tb{Step 2}: Apply the decision rule:
      \begin{center}
        \begin{tabular}{|c|c|}
          \hline
          \textbf{$H_1$}   & \textbf{Rejection Region} \\ \hline
          $\mu \neq \mu_0$ & $|z| > z_{\alpha/2}$      \\ \hline
          $\mu < \mu_0$    & $z < -z_{\alpha}$         \\ \hline
          $\mu > \mu_0$    & $z > z_{\alpha}$          \\ \hline
        \end{tabular}
      \end{center}
\end{itemize}

\par \tb{Testing with Population Proportion - Large Sample Size}

\begin{itemize}
  \item \tb{Step 1}: Compute the statistic:
    \[
      z = \frac{\hat{p} - p_0}{\sqrt{p_0 q_0 / n}}
    \]
  \item \tb{Step 2}: Apply the decision rule:
      \begin{center}
        \begin{tabular}{|c|c|}
          \hline
          \textbf{$H_1$}   & \textbf{Rejection Region} \\ \hline
          $p \neq p_0$ & $|z| > z_{\alpha/2}$      \\ \hline
          $p < p_0$    & $z < -z_{\alpha}$         \\ \hline
          $p > p_0$    & $z > z_{\alpha}$          \\ \hline
        \end{tabular}
      \end{center}
\end{itemize}


\end{multicols*}

\begin{multicols}{3}

\hi{Digital Arithmetic}
  \hii{Terminology}
    \begin{itemize}
      \item \tb{1's complement}: obtained by convert $0 \to 1$ and $1 \to 0$.
      \item \tb{2's complement}: obtained by adding 1 to 1's complement.
      \item \tb{Signed numbers}: include positive and negative numbers.
      \item \tb{Unsigned numbers}: include only non-negative numbers.
      \item \tb{Sign bit}: the bit in front of the MSB to store the sign
        of a signed number.
    \end{itemize}

  \hii{Number Representation}
    \par Representing signed number requires two parts: magnitude and sign bit.
    \begin{itemize}
      \item \tb{Positive numbers}: magnitude is in \tb{true binary} form, sign bit is 0.
      \item \tb{Negative numbers}: magnitude is in \tb{2's complement} form, sign bit is 1.
    \end{itemize}

  \hii{Operations}
    \hiii{Negation}
      \par To negate a number, take it 2's complement.
    \hiii{Subtraction}
      \par To subtract $a$ to $b$, add $a$ with the 2's complement of $b$.
    \hiii{BCD Addition}
      \par Add digit by digit. For each pair of digits:
      \begin{itemize}
        \item If sum is $\leq 9$: accept sum.
        \item If sum is $> 9$: add 6 to sum and carry 1 to the next pair.
      \end{itemize}
    \hiii{HEX Addition}
      \par Add digit by digit.
      \begin{itemize}
        \item If sum is $\leq F$: accept sum.
        \item If sum is $> F$: subtract $F$ from sum and carry 1 to the next pair.
      \end{itemize}
    \hiii{HEX 2's Complement}
      \par Subtract each digit \ti{from} $F$. Then add 1.

\end{multicols}
\chapter{Counters and Registers}

\hi{Asynchronous (Ripple) Counters}
  \hii{Introduction}
    \par An four-bit binary counter comprises of 4 J-K FF connecting in sequence.
    \par \ti{\tu{Note:} Most counters in the course is triggered by NGT. In
      practical use, PGT-triggered counters can also be used.}
    \begin{figure}[H]
      \centering
      \includegraphics[width=16cm]{c07/four-bit-counter.jpg}
    \end{figure}

  \hii{MOD number}
    \begin{equation*}
      \mbox{MOD number} = N
    \end{equation*}
    where $n$ is the number of J-K FFs.

  \hii{Frequency Division}
    \par In any counter, the signal at the output of the last FF (i.e., the
    MSB) will have a frequency equal to the input clock frequency
    divided by the MOD number of the counter.
    \begin{align*}
      f_{\mbox{last FF}} = \frac{f_{CLK}}{MOD} = \frac{f_{CLK}}{2^{N}}
    \end{align*}
    where $n$ is the number of J-K FFs.


\hi{Propagation Delay In Ripple Counters}
  \hii{The cause of propagation delay}
    \par Each J-K FF in the counter has a certain amount of propagation delay
    $t_{pd}$.
    \par Because of the inherent propagation delay time ($t_{pd}$) of each
      FF, this means that the second FF will not respond until a time $t_{pd}$
      after the first FF receives an active clock transition; the third FF
      will not respond until a time equals to $2 \times t_{pd}$ after that
      clock transition; and so on.
    \par In general, the $n-th$ FF will not respond until a time equals to
    $(n - 1) \times t_{pd}$ after that clock transition.
  \hii{Requirement}
    \par For proper counter operation we need:
    \begin{itemize}
      \item In terms of the period of the input-clock:
      \begin{equation*}
        T_{\mbox{CLK}} \geq N \times t_{pd}
      \end{equation*}
      \item In terms of the frequency of the input-clock:
      \begin{equation*}
        f_{\mbox{CLK max}} \geq N \times t_{pd}
      \end{equation*}
    \end{itemize}


\hi{Counters with MOD Numbers $\leq 2^{N}$}
  \par Each input which is at $HIGH$ at the state $MOD$ is connected to a
  common NAND gate. At that state, everything is cleared.

\hi{Decade/BCD}
  \par \ti{\tu{Note:} A BCD counter is a decade counter, while a decade
    counter is NOT necessary a BCD counter}.


\hi{Asynchronous Counter IC: 74LS293}


\hi{Designing Counter Problems}
  \hii{Asynchronous Counters}
    \begin{itemize}
      \item \tb{CLK inputs}
        \begin{itemize}
          \item \tb{UP} counter: $Q$ of the previous FF is connected to
            $CLK$ of the next FF.
          \item \tb{DOWN} counter: $\bar{Q}$ of the previous FF is connected to
              $CLK$ of the next FF.
        \end{itemize}
      \item \tb{JK FF}
        \par Both $J$ input and $K$ input are connected to $1$. (toggle state)
      \item \tb{D FF}
        \par The $D$ input is connected to the $\bar{Q}$ output of the same FF.
    \end{itemize}
    
  \hii{Synchronous Counters}
    \begin{itemize}
      \item \tb{CLK inputs}
        \par All $CLK$ inputs of all FFs are connected to \tb{one} common
          clock signal.
      \item \tb{JK FF}
        \par Both $J$ input and $K$ input of the $i_{th}$ FF are connected to
          $Q_{0} \cdot Q_{1} \cdot \ldots \cdot Q_{i - 1}$.
          \begin{equation}
            J_{i} = K_{i} = \PROD{_{j = 0}^{i - 1} Q_{j}}
          \end{equation}
      \item \tb{D FF}
        \par The $D$ input of the $i_{th}$ FF is connected to
          $Q_{0} \cdot Q_{1} \cdot \ldots \cdot Q_{i - 1} \cdot \bar{Q}$.
          \begin{equation}
            D_{i} = \bigg( \PROD{_{j = 0}^{i - 1} Q_{j}} \bigg) \cdot \bar{Q_{i}}
          \end{equation}
    \end{itemize}

\chapter{Rotation of Rigid Bodies}


\hi{Angular Velocity and Acceleration}
  \par Define $z$ as the axis of rotation.
  \hii{Angular Velocity}
    \par Average angular velocity:
      \begin{eqbox}
        \omega_z = \frac{\Delta\theta}{\Delta t}
      \end{eqbox}
    \par Instantaneous angular velocity:
      \begin{eqbox}
        \omega_z = \dif{\Delta\theta}{\Delta t}
      \end{eqbox}
  \hii{Angular Acceleration}
    \par Average angular acceleration:
      \begin{eqbox}
        \alpha_z = \frac{\Delta\omega_z}{\Delta t}
      \end{eqbox}
    \par Instantaneous angular acceleration:
      \begin{eqbox}
        \alpha_z = \dif{\Delta\omega_z}{\Delta t} = \ddif{\Delta\theta}{\Delta t}
      \end{eqbox}


\hi{Rotation with Constant Angular Acceleration}


\hi{Energy in Rotational Motion - Moment of Inertia}
  \par Kinetic energy of a rotating particle:
    \begin{align*}
      K_i = \frac{1}{2} m_i v_i^2 = \frac{1}{2} m_i r_i^2 \omega_i^2
    \end{align*}
  \par For a rigid body consisting of particles:
    \begin{align*}
      K = \SUM_{i} \frac{1}{2}  m_i r_i^2 \omega_i^2
    \end{align*}
  \par For a rigid body, $\forall i, \omega_i = \omega$.
    \begin{align*}
      K = \frac{1}{2} \bigg(\SUM_{i} m_i r_i^2\bigg) \omega^2
    \end{align*}
  \par \tb{Moment of inertia} is defined as:
  \begin{eqbox}
    I = \SUM_{i} m_i r_i^2
  \end{eqbox}
  \par Then, rotational kinetic energy can be calculated by:
  \begin{eqbox}
    K = \frac{1}{2} I\omega^2
  \end{eqbox}


\hi{Parallel-Axis Theorem}
  \begin{eqbox}
    I_p = I_{CM} + md^2
  \end{eqbox}
  where
  \begin{itemize}
    \item $I_{CM}$: moment of inertia of the body about the axis through the
      center of mass
    \item $I_p$: moment of inertia of the body about an axis $p$ parallel to
      the axis through the center of mass
    \item $m$: mass of the body
    \item $d$: distance between two axes
  \end{itemize}


\hi{Moment of inertia calculations}

\begin{appendices}
  \chapter{Complex Numbers}

\hi{Complex number}
  %\hii{

  \chapter{Adder Circuits}

\hi{Half-Adder Circuit}
  \hii{Truth table}
    \begin{tabular}{|c|c|c|c|}
      \hline 
      IN1 & IN2 & SUM & CARRY \\ 
      \hline 
      0 & 0 & 0 & 0 \\ 
      \hline 
      0 & 1 & 1 & 0 \\ 
      \hline 
      1 & 0 & 1 & 0 \\ 
      \hline 
      1 & 1 & 0 & 1 \\ 
      \hline 
    \end{tabular} 


\hi{Full-Adder Circuit}
  \hii{Truth table}
\begin{tabular}{|c|c|c|c|c|}
    \hline 
    CARRY-IN & IN1 & IN2 & SUM & CARRY-OUT \\ 
    \hline 
    0 & 0 & 0 & 0 & 0 \\ 
    \hline 
    0 & 0 & 1 & 1 & 0 \\ 
    \hline 
    0 & 1 & 0 & 1 & 0 \\ 
    \hline 
    0 & 1 & 1 & 0 & 1 \\ 
    \hline 
    1 & 0 & 0 & 1 & 0 \\ 
    \hline 
    1 & 0 & 1 & 0 & 1 \\ 
    \hline 
    1 & 1 & 0 & 0 & 1 \\ 
    \hline 
    1 & 1 & 1 & 1 & 1 \\ 
    \hline 
\end{tabular} 


\end{appendices}

\end{document}
