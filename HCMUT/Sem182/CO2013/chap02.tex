\chapter{Database System Concepts and Architecture}

\hi{Data Models, Schemas, and Instances}
  \begin{itemize}
    \item \tb{Data abstraction} generally refers to the suppression of details
      of data organization and storage, and the highlighting of the essential
      features for an improved understanding of data. One of the main
      characteristics of the database approach is to support data abstraction
      so that different users can perceive data at their preferred level of
      detail.
    \item \tb{Data model}: A collection of concepts that can be used to
      describe the structure of a database - provides the necessary means to
      achieve this abstraction.
    \item \tb{Structure of a database}: including the data types,
      relationships, and constraints that apply to the data.
  \end{itemize}

  \hii{Categories of Data Models}
    \begin{itemize}
      \item \tb{High-level} or \tb{conceptual data models}: provide concepts
        that are close to the way many users perceive data.
      \item \tb{Representational} or \tb{implementation data models}: provide
        concepts that may be easily understood by end users but that are not
        too far removed from the way data is organized in computer storage.
        Representational data models hide many details of data storage on disk
        but can be implemented on a computer system directly.
      \item \tb{Low-level} or \tb{physical data models}: provide concepts that
        describe the details of how data is stored on the computer storage
        media.
      \item \tb{Self-describing data models}: combines the description of the
        data with the data value themselves.
    \end{itemize}

    \hiii{Conceptual Data Models}
      \begin{itemize}
        \item An \tb{entity} represents a real-world object or concept.
        \item An \tb{attribute} represents some property of interest that
          further describe an entity.
        \item A \tb{relationship} among two or more entities represents an
          association among the entities.
      \end{itemize}

    \hiii{Representational Data Model}
      \par Representational data models include the widely used \tb{relational
        data model}, as well as the legacy \tb{network and hierarchical
        models}.
      \par Representational data models represent data by using record
        structures and hence are sometimes called \tb{record-based data
        models}.
      \par \tb{Object data model} is an example of a new family of higer-level
        implementation data models that are closer to conceptual data models.

    \hiii{Physical Data Models}
      \par \tb{Physical data models} describe how data is stored as files in
        the computer by representing information such as record formats, record
        orderings, and access path.
      \begin{itemize}
        \item An \tb{access path} is a search structure that makes the search
          for particular database records efficient, such as indexing or
          hashing.
        \item An \tb{index} is an example of an acess path that allows direct
          access to data using an index term or a keyword.
      \end{itemize}

    \hiii{Self-describing Data Models}
      \par Different from traditional data models, \tb{self-describing data
        models} combines the description of the data with the data balues
        themselves. These models include \tb{XML} as well as many of the
      \tb{key-value stores} and \tb{NOSQL systems} that were recently created
        for managing big data.

  \hii{Schemas, Instances, and Database States}
    \hiii{Schemas}
      \begin{itemize}
        \item The \tb{database schema} is the description of a database, which
          is specified during database design and is not expected to change
          frequently. Most data models have certain conventions for displaying
          schemas as diagrams.
        \item A \tb{schema diagram} is a displayed schema. A schema diagram
          displays only \ti{some aspects} of constraints. Other aspects are not
          specified in the schema diagram.
        \item A \tb{schema construct} is an object in the schema.
      \end{itemize}

    \hiii{Instances}
      \begin{itemize}
        \item A \tb{database state} or \tb{snapshot} is the database at a
          particular moment in time. It is also called the current set of
          \tb{occurrence} or \tb{instances} in the database.
      \end{itemize}
    
    \hii{Database States}
      \begin{itemize}
        \item When we \tb{define} a new database, we specify its database schema
          only to the DBMS. At this point, the corresponding database state is
          the \ti{empty state} with no data.
        \item We get the \tb{initial state} of the data when the database is
          first \tb{populated} or \tb{loaded} with the initial data. From then
          on, every time an update operation is applied to the database, we get
          another \tb{database state}.
        \item A \tb{valid state} is a state that satisfies the structure and
          constraints specified in the schema. \ti{Hence, specifying a correct
          schema to the DBMS is extremely important}.
        \item \tb{Meta-data} are the descriptions of the schema constructs and
          constraints, stored by the DBMS in the catalog so that DBMS software
          can refer to the schema whenever it needs to.
        \item The \tb{schema} is sometimes called the \tb{intension}, and the
          database state is called an \tb{extension} of the schema.
        \item A \tb{schema evolution} is a change in the schema.
      \end{itemize}

\hi{Three-Schema Architecture and Data Independence}
  \par As discussed in chapter 1, the important characteristics of the database
    approach are:
    \begin{itemize}
      \item Self-describing nature of a database system
      \item Insulation between programs and data, and data abstraction
      \item Support of multiple views of the data
      \item Sharing of data and multiuser transaction processing
    \end{itemize}
  \par The first three characteristics can be archieved an visualized by an
  architecture for database systems called the \tb{three-schema} architecture.

  \hii{The Three-Schema Architecture}
    \par The goal of the \tb{three-schema} architecture is to separate the user
      applications from the physical database.
    \par In this architecture, the schema can be defined at the following three
      levels:
      \begin{itemize}
        \item The \tb{internal level} has an \tb{internal schema}, which
          describes the physical storage structure of the database.
        \item The \tb{conceptual level} has a \tb{conceptual schema}, which
          describes the structure of the whole database for a community or
          user.
        \item The \tb{external} or \tb{view level} includes a number of
          \tb{external schemas} or \tb{user views}.
      \end{itemize}
    \par The three-level schema are only \ti{descriptions of data}; the actual
      data is stored at the physical level only. In the three-level schema,
      each user group refers to its own external schema. Hence, the DBMS must
      transform a request specified on an external schema into a request
      against the conceptual schema, and then into a request on the internal
      schema for processing over the stored database. If the request is a
      database retrieval, the data extracted from the stored database must be
      reformatted to match the user's external view.
    \par The processes of transforming requests and results between levels are
    called \tb{mappings}.
  
  \hii{Data Independence}
    \par The three-level architecture can be used to further explain the
    concept of \tb{data independence}, which can be defined as the capacity to
    change the schema at one level of a database system without having to
    change the schema at the next higher level.
