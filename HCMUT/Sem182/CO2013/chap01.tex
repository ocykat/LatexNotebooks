\chapter{Databases and Database Users}

\hi{Introduction}

  \hii{Terminology}
    \begin{itemize}
      \item \tb{data}: known facts that can be recorded and have implicit
        meaning.
      \item \tb{database}: collection of related data 
      \item \tb{database management system (DBMS)}: computerized system that
        enables users to create and maintain database. The DBMS is a
        \ti{general-purpose software system} that facililtates the processes of
        defining, constructing, manipulating, and sharing databases among various
        users and applications.
        \begin{itemize}
          \item \tb{Defining} a database involves specifying the datatypes,
            structures, and constraints of the data to be stored in the database.
            The database definition or descriptive information is also stored by
            the DBMS in the form of databas catalog or dictionary, called
            \tb{meta-data}.
          \item \tb{Constructing} the database is the process of storing the data
            on some storage medium that is controlled by the DBMS.
          \item \tb{Manipulating} a database includes functions such as querying,
            updating, and generating reports from the data.
          \item \tb{Sharing} a database allows multiple users and programs to
            access the database \ti{simultaneously}.
        \end{itemize}
      \item A \tb{query} causes some data to be retreived
      \item A \tb{transaction} causes some data to be read and some data to be
        written into the database.
      \item A \tb{data record} is a row in a table.
      \item A \tb{data element} is a cell in a row. Each data element has a
        \tb{data type}.
    \end{itemize}

\hi{Characteristics of the Database Approach}
  \begin{itemize}
    \item Self-describing nature of a database system
    \item Insulation between programs and data, and data abstraction
    \item Support of multiple views of the data
    \item Sharing of data and multiuser transaction processing
  \end{itemize}
  
  \hii{Self-Describing Nature of a Database System}
    \par A fundamental characteristic of the database approach is that the
      database system contains not only the database itself but also a
      complete definition or description of the database structure and
      constraints. The definition or description, called \tb{meta data}, is
      stored in the DBMS catalog.

  \hii{Insulation between Programs and Data, and Data Abstraction}
    \begin{itemize}
      \item \tb{Program-data independence}: In the traditional data files
        approach, the structures of the files is embedded in the application
        programs, resulting in any changes to the structure of a file may
        require changing all programs that access that file. By contrast,
        DBMS access programs do not require such changes, since the structure
        of data files is stored in the DBMS catalog separately from the
        program. This property is called \tb{program-data independence}.
      \item \tb{Program-operation independence}: The characteristic that
        allows the \ti{interface} and the \ti{implementation} of an operation
        to separate so that a change in the implementation will not affect the
        interace.
        \begin{itemize}
          \item \tb{Operation}: a function or a method. An operation is
            specified in two parts: interface and implementation.
          \item \tb{Interface}: (also known as \ti{signature}) of an
            operation includes the operation name and the data types of its
            arguments (or parameters).
        \end{itemize}
      \item \tb{Data abstraction}: The characteristic that allows
        program-data independence and program-operation independence is
        called \tb{data abstraction}.
      \item \tb{Conceptual representation}: the representation of data of the
        DBMS that does not include many of the details of how the data is
        sotred or how the operation are implemented.
      \item \tb{Data model}: a type of data abstraction that is used to
        provide a conceptual representation. The data model uses logical
        concepts, such as objects, their properties, and their
        interrelationships, that may be easier for most users to understand
        than computer storage concepts. Hence, the data model hides storages
        and implementation details that are not of interest to most database
        users.
    \end{itemize}

  \hii{Support of Multiple Views of the Data}
    \par A database typically has many types of users, each of whom may
    require a different perspective or \tb{view} of the database.

  \hii{Sharing of Data and Multiuser Transaction Processing}
    \par The DBMS must include \tb{concurrency control} software to ensure
    that several users trying to update the same data do so in a controlled
    manner so that the result of the updates is correct.
    \begin{itemize}
      \item \tb{Online Transaction Processing (OLTP)}: An application of this
        type facillitates and manages transaction-oriented applications,
        typically for data entry and retrieval transaction processing.
      \item \tb{Transaction}: An executing program or process that includes
        one or more database accesses, such as reading or updating of
        database records. Each transaction is supposed to execute a logically
        correct database access if executed in its entirety without
        interference from other transactions.
      \item \tb{Isolation}: A DBMS transaction property ensuring that each
        transaction appears to execute in isolation from other transactions,
        even though hundreds of transactions may be executing concurrently.
      \item \tb{Atomicity}: A DBMS property ensuring that either all the
        database operations in a transaction are executed or none are.
    \end{itemize}

\hi{DBMS Users}
  \begin{itemize}
    \item \tb{Database Administrators}
    \item \tb{Database Designers} (*)
    \item \tb{End Users}
    \item \tb{System Analysts and Application Programmers (Software Engineers)}
  \end{itemize}
