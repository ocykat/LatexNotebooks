\chapter{Probability}

\hi{Sample Spaces and Events}
  \hii{Random Experiments}
    \par \tb{Definition}: An experiment that can result in different outcomes,
      even though it is repeated in the same manner every time, is called
      \tb{random experiment}.

  \hii{Sample Spaces}
    \hiii{Definition}
      \par \tb{Definition}: The set of all possible outcomes of a random
        experiment is called the \tb{sample space} of the experiment.
        \par The sample space is denoted as $S$.

    \hiii{Discrete and Continuous Sampe Spaces}
      \par A sample space is \tb{discrete} if it consists of a finite or
        countable infinite set of outcomes.
      \par A sample space is \tb{continuous} if it contains an interval (either
        finite or infinite) of real numbers.

  \hii{Events}
    \hiii{Definition}
      \par \tb{Defintion}: An \tb{event} is a subset of the sample space of a
        random experiment.

    \hiii{Set Operations}
      \par Because events are sets, we can apply sets operations on them:
      \begin{itemize}
        \item The \tb{union} $E$ of two events $E_1$ and $E_2$, denoted $E =
          E_1 \cup E_2$, is the event that conists of all outcomes that are
          contained in \ti{either} of the two events $E_1$ or $E_2$.
        \item The \tb{intersection} $E$ of two events $E_1$ and $E_2$, denoted
          $E = E_1 \cap E_2$, is the event that conists of all outcomes that
          are contained in \ti{both} of the two events $E_1$ or $E_2$.
        \item The \tb{complement} $E'$ of an event $E$ \ti{in a sample space}
          is the set of outcomes in the sample space that are not in the event.
      \end{itemize}

    \hiii{Mutually Exclusive Events}
      \par Two events, $E_1$ and $E_2$, are said to be \tb{mutually exclusive}
        if:
        \begin{align*}
          E_1 \cap E_2 = \emptyset
        \end{align*}

  \hii{Counting Techniques}
    \hiii{Multiplication Rule}
      \par \tb{Rule}: Assume an operation can be described as a sequence of $k$
      steps, where the number of ways of completing the $i^{th}$ step $(1 \leq i
      \leq k$ is $n_i$. The total of ways of completing the operation is:
      \begin{align*}
        n = \PROD{_{i = 1}^{k}} n_i = n_1 \times n_2 \times \ldots \times n_k
      \end{align*}

    \hiii{
