\hi{Automata}
\hii{Defintion of Automata}
  \par An \tb{automata} consists of:
    \begin{itemize}
      \item States
      \item Transitions between states
    \end{itemize}

\hii{Alphabet}
  \hiii{Definitions}
    \par An \tb{alphabet} is a finite set of symbols (or characters).
    \par A string/word $u$ over $\Sigma$ is a finite sequence (possibly empty) of symbols (or characters) in $\Sigma$.
    \par An empty string is denoted by $\epsilon$.
    \par The length of the string $u$, denoted by $|u|$, is the number of characters.
    \par All strings over $\Sigma$ is denoted by $\Sigma^*$.

\hii{Operations on Languages}
  \par Let $L$, $L_1$, $L_2$ be languages over $\Sigma$.
  \begin{itemize}
    \item Union
    \[
      L_1 \cup L_2 = \{ u \in \Sigma^* | u \in L_1 \text{ or } u \in L_2 \}
    \]
    \item Intersection
    \[
      L_1 \cap L_2 = \{ u \in \Sigma^* | u \in L_1 \text{ and } u \in L_2 \}
    \]
    \item Difference
    \[
      L_1 \setminus L_2 = \{ u \in \Sigma^* | u \in L_1 \text{ and } u \not \in L_2 \}
    \]
    \item Complement
    \[
      \bar{L} = \Sigma^* \setminus L
    \]
    \item Multiplication (Concatenation)
    \[
      L_1 L_2 = \{uv | u \in L_1, v \in L_2\},
    \]
    \item Product
    \[
      L_0 = \{\epsilon\}, \quad L^n = L^{n - 1} \times L \text{ for } n \geq 1
    \]
    \item Transitive Closure
    \[
      L^* = \bigcup\limits_{i = 0}^{\infty} L_i
    \]
    \[
      L^+ = \bigcup\limits_{i = 1}^{\infty} L_i
    \]
  \end{itemize}

\hii{Regular Expressions}
  \hiii{Regular operations on languages}
    \begin{itemize}
      \item Union: $+$
      \item Concatenation: $.$
      \item Transitive Closure: $*$
    \end{itemize}

  \hiii{Regular Language}
    \par Language $L \subseteq \Sigma^*$ is regular if and only if there exists a regular expression over $\Sigma$ representing language $L$.

\hii{Finite Automata}
  \par A \tb{finite automata} consists of:
  \begin{itemize}
    \item \tb{one} initial state
    \item \tb{one or several} final/accepting states
  \end{itemize}
  and \tb{the number of states must be finite}.

\hii{Nondeterministic Finite Automata (NFA)}
  \par A \tb{NFA} is defined by a 5-tuplet $(Q, \Sigma, q_0, \delta, F)$ where:
  \begin{itemize}
    \item $Q$ a finite set of states.
    \item $\Sigma$ is the alphabet of the automata.
    \item $q0 \in Q$ is the initial state.
    \item $\delta$ : $Q × \Sigma \to Q$ is a transition function.
    \item $F \subseteq Q$: is the set of final/accepting states.
  \end{itemize}
  \par In a NFA, from a state, each input may correspond to \tb{multiple transaction from that state}. 

\hii{Deterministic Finite Automata (DFA)}
    \par A \tb{DFA} is also defined by a 5-tuplet $(Q, \Sigma, q_0, \delta, F)$.
    \par However, in a DFA, from a state, each input corresponds to \tb{one and only one transaction from that state}.

\hii{Configurations and Executions}
  \par A configuration of automata $A$ is a couple $(q, u)$ where $q \in Q$ and $u \in \Sigma^*$.
  \par An execution of automata $A$ is a sequence of configurations $(q0, u0) \to (qn, un)$ such that $(qi, ui) \to (qi+1, ui+1)$, for $i = 0, 1, \to, n − 1$.

\hii{Recognized Language}
  \par A language $L \subseteq \Sigma∗$ over an alphabet $\Sigma$, is \tb{recognized} if there exists a finite automata accepting all strings of $L$.
  \par If $L_1$ and $L_2$ are \tb{recognized languages}, then:
  \begin{itemize}
    \item $L_1 \cup L_2$ and $L_1 \cap L_2$ are also recognized.
    \item $L_1 \dot L_2$ and $L_1^*$ are also recognized.    
  \end{itemize}

\hii{NFA to DFA}
  \par Create a transition table with $k + 1$ columns, where $k = |\Sigma|$ (the number of characters of the language).
    \begin{itemize}
      \item The first column (theoretically) list alls possible subset of the state set $Q$ of the NFA. We can omit the unnessary subsets in practice, by only including newly-created subsets in this column.
      \item The remaining $k$ columns, each correspond to a character of the language. For a NFA, each input character would result in a subset of state $Q$. Any newly-created subset is then add to the first column.
    \end{itemize}
  \par In the end, each valid subset in the first column is a state of the DFA.

\hii{DFA Minimization}
  \par One method of minimization is partitioning states into groups of similar characteristics.
  \par \tb{Steps}:
  \begin{enumerate}
    \item Create a table with $k + 1$ rows, where $k = |\Sigma|$ is the number of characters of the language. The first row indicates the current partition. Each of the following row corresponds to a character.
    \item For each cell in the following $k$ rows, write the group number of the next state.
    \item After finishing the table, try to alter the grouping and create a new table based on the new grouping.
    \item The algorithm stops if two consecutive tables are indentical.
  \end{enumerate}