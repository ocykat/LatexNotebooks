\chapter{Propositional Logic}

\section{Proposition/Declarative Sentences}
  \subsection{Definition}
  \par A \tb{proposition} or a \tb{declarative sentence} is a sentence that is either \tb{True} or \tb{False} but not both.

  \subsection{Atomic Declarative Sentence and Notation}
  \par Usually, \tb{atomic} or \tb{indecomposable} sentences are considered.
  \par Each \tb{atomic declarative sentence} is assign a symbol $p$, $q$, $r$, etc.

\section{Logical Connectives}
  \par \tb{Logical Connective }
  \par Let $p$, and $q$ be propositions (or declarative sentences).

  \subsection{Negation}
    \hiii{Definition}
      \begin{itemize}
        \item The \tb{negation} of $p$, denoted by $\lnot p$, is the statement ``It is not the case that $p$".
        \item $\lnot p$ is read ``\tb{NOT} p".
        \item The truth value of $\lnot p$ is the opposite of the truth value of $p$.
      \end{itemize}
    \hiii{Truth Table}
      \begin{center}
        \begin{tabular}{|c|c|}
          \hline
          $p$ & $\lnot p$ \\
          \hline
          $\lT$ & $\lF$ \\
          \hline
          $\lF$ & $\lT$ \\
          \hline
        \end{tabular}
      \end{center}

  \subsection{Conjunction}
    \hiii{Definition}
      \begin{itemize}
        \item The \tb{conjunction} of $p$ and $q$, denoted by $p \land q$, is the proposition ``$p$ and $q$".
        \item The conjunction $p \land q$ is \tb{True} when both $p$ and $q$ are \tb{True}
        and is \tb{False} otherwise.
      \end{itemize}

    \hiii{Truth Table}
      \begin{center}
        \begin{tabular}{|c|c|c|}
          \hline
          $p$ & $q$ & $p \land q$ \\
          \hline
          $\lT$ & $\lT$ & $\lT$ \\
          \hline
          $\lT$ & $\lF$ & $\lF$ \\
          \hline
          $\lF$ & $\lT$ & $\lF$ \\
          \hline
          $\lF$ & $\lF$ & $\lF$ \\
          \hline
        \end{tabular}
      \end{center}

  \subsection{Disjunction}
    \hiii{Definition}
      \begin{itemize}
        \item Let $p$ and $q$ be propositions. The \tb{disjunction}
        of $p$ and $q$, denoted by $p \lor q$, is the proposition ``$p$ or $q$".
        \item The disjunction $p \lor q$ is \tb{False} when both $p$ and $q$ are \tb{False} and is \tb{True} otherwise.
      \end{itemize}

    \hiii{Truth Table}
      \begin{center}
        \begin{tabular}{|c|c|c|}
          \hline
          $p$ & $q$ & $p \lor q$ \\
          \hline
          $\lT$ & $\lT$ & $\lT$ \\
          \hline
          $\lT$ & $\lF$ & $\lT$ \\
          \hline
          $\lF$ & $\lT$ & $\lT$ \\
          \hline
          $\lF$ & $\lF$ & $\lF$ \\
          \hline
        \end{tabular}
      \end{center}

  \subsection{Exclusive Disjunction}
    \hiii{Definition}
      \begin{itemize}
        \item The \tb{exclusive disjunction} of $p$ and $q$, denoted by $p \lxor q$, is the proposition that is \tb{True} when
        exactly one of $p$ and $q$ is \tb{True} and is \tb{False} otherwise.
      \end{itemize}

    \hiii{Truth Table}
    \begin{center}
      \begin{tabular}{|c|c|c|}
        \hline
        $p$ & $q$ & $p \lxor q$ \\
        \hline
        $\lT$ & $\lT$ & $\lF$ \\
        \hline
        $\lT$ & $\lF$ & $\lT$ \\
        \hline
        $\lF$ & $\lT$ & $\lT$ \\
        \hline
        $\lF$ & $\lF$ & $\lF$ \\
        \hline
      \end{tabular}
    \end{center}

  \subsection{Implication}
    \hiii{Definition}
      \begin{itemize}
        \item Let $p$ and $q$ be propositions. The conditional statement $p \limpl q$ is the proposition ``if p, then q".
        \item The conditional statement $p \limpl q$ is \tb{False} when $p$
        is \tb{True} and $q$ is \tb{False}, and \tb{True} otherwise.
        \item $p$ is called the \tb{assumption} and $q$ is called the \tb{conclusion} of the \tb{implication} $p \limpl q$. \fnmark
        \fntext{This definition is according to Huth and Ryan's book. In Rosen's book, $p$ is called the \tb{hypothesis} (or antedecent or premise) and $q$ is called the \tb{conclusion} (or consequence).}
      \end{itemize}

    \hiii{Truth Table}
      \begin{center}
        \begin{tabular}{|c|c|c|}
          \hline
          $p$ & $q$ & $p \limpl q$ \\
          \hline
          $\lT$ & $\lT$ & $\lT$ \\
          \hline
          $\lT$ & $\lF$ & $\lF$ \\
          \hline
          $\lF$ & $\lT$ & $\lT$ \\
          \hline
          $\lF$ & $\lF$ & $\lT$ \\
          \hline
        \end{tabular}
      \end{center}

  \subsection{Biconditional}
    \hiii{Definition}
      \begin{itemize}
        \item Let $p$ and $q$ be propositions. The \tb{biconditional statement}
        $p \liff q$ is the proposition ``$p$ if and only if $q$".
        \item The \tb{biconditional} is \tb{True} when $p$ and $q$ have the same
        truth value, and is \tb{False} otherwise.
      \end{itemize}

    \hiii{Truth Table}
    \begin{center}
      \begin{tabular}{|c|c|c|}
        \hline
        $p$ & $q$ & $p \liff q$ \\
        \hline
        $\lT$ & $\lT$ & $\lT$ \\
        \hline
        $\lT$ & $\lF$ & $\lF$ \\
        \hline
        $\lF$ & $\lT$ & $\lF$ \\
        \hline
        $\lF$ & $\lF$ & $\lT$ \\
        \hline
      \end{tabular}
    \end{center}

  \subsection{Logical Equivalences}
    \begin{center}
      \begin{tabular}{|c|c|}
        \hline
        \textbf{Equivalence} & \textbf{Name} \\
        \hline

        $p \land \lT \equiv p$ & \multirow{2}{*}{Identity laws} \\
        $p \lor \lF \equiv p$ & \\
        \hline

        $p \land \lF \equiv \lF$ & \multirow{2}{*}{Domination laws} \\
        $p \lor \lT \equiv \lT$ & \\
        \hline

        $p \land p \equiv p$ & \multirow{2}{*}{Idempotent laws} \\
        $p \lor p \equiv p$ & \\
        \hline

        $\lnot (\lnot p) \equiv p$ & Double negation laws \\
        \hline

        $p \land q \equiv q \land p$ & \multirow{2}{*}{Commutative laws} \\
        $p \lor q \equiv q \lor p$ & \\
        \hline

        $(p \land q) \land r \equiv p \land (q \land r)$
          & \multirow{2}{*}{Associative laws} \\
        $(p \lor q) \lor r \equiv p \lor (q \lor r)$ & \\
        \hline

        $\lnot (p \land q) \equiv \lnot p \lor \lnot q$
          & \multirow{2}{*}{De Morgan's laws} \\
        $\lnot (p \land q) \equiv \lnot p \lor \lnot q$ & \\
        \hline

        $p \land (p \lor q) \equiv p$
          & \multirow{2}{*}{Absorption laws} \\
        $p \lor (p \land q) \equiv p$ & \\
        \hline

        $p \lor \lnot p \equiv \lT$ & \multirow{2}{*}{Negation laws} \\
        $p \land \lnot p \equiv \lF$ & \\
        \hline

        $p \land (q \lor r) \equiv (p \land q) \lor (p \land r)$
          & \multirow{2}{*}{Distributed laws} \\
        $p \lor (q \land r) \equiv (p \lor q) \land (p \lor r)$ & \\
        \hline

        $p \limpl q \equiv \lnot p \lor q$
        & \multirow{5}{*}{Logical Equivalences Involving Conditional Statements} \\
        $(p \limpl q) \land (p \limpl r) \equiv p \limpl (q \land r)$
        & \\
        $(p \limpl q) \lor (p \limpl r) \equiv p \limpl (q \lor r)$
        & \\
        $(p \limpl q) \land (q \limpl r) \equiv (p \lor q) \limpl r$
        & \\
        $(p \limpl q) \lor (q \limpl r) \equiv (p \land q) \limpl r$
        & \\
        \hline
      \end{tabular}
    \end{center}

\section{Binding Priority and Association}
  \par The binding priority from high to low is as follows:
    \begin{itemize}
      \item Negation $\lnot$
      \item Conjunction $\land$ and Disjunction $\lor$
      \item Implication $\limpl$
    \end{itemize}
  \par Implication is right-associative, meeaning that $p \limpl q \limpl r$ is equivalent to $p \limpl (q \limpl r)$.

\section{Premises, Conclusion and Sequent}
    \par Suppose we have:
    \begin{itemize}
      \item a set of formulas $\phi_1, \ldots, \phi_n$ called \tb{premises}.
      \item a formula $\psi$ called \tb{conclusion}.
    \end{itemize}
    \par The intention is by applying \tb{proof rules} to the premises, we finally obtain the conclusion. This intention can be denoted by:
    \begin{align*}
      \phi_1, \ldots, \phi_n \lprove \psi
    \end{align*}
    \par The expression is called a \tb{sequent}.
    \par A \tb{sequent} is \tb{valid} if a proof for it can be found.

\section{Natural Deduction}
  \par A natural deduction can be denoted by:
  \begin{center}
    \AxiomC{$\phi_1$}
    \AxiomC{$\phi_2$}
    \AxiomC{$\ldots$}
    \AxiomC{$\phi_n$}
    \RightLabel{[Rule Name]}
    \QuaternaryInfC{$\psi$}
    \DisplayProof
  \end{center}

  \subsection{Rules for Conjunction (AND)}
    \begin{enumerate}[a.]
      \item \tb{Introduction Rule} is denoted by $\land i$ and read \tb{``and-introduction"}.
        \begin{center}
          \AxiomC{$\phi$}
          \AxiomC{$\psi$}
          \RightLabel{$\land i$}
          \BinaryInfC{$\phi \land \psi$}
          \DisplayProof
        \end{center}

      \item \tb{Elimination Rule}: is denoted by $\land e$ and read \tb{``and-elimination"}.
        \begin{center}
          \AxiomC{$\phi \land \psi$}
          \RightLabel{$\land e_1$}
          \UnaryInfC{$\phi$}
          \DisplayProof
          \hskip 2cm
          \AxiomC{$\phi \land \psi$}
          \RightLabel{$\land e_2$}
          \UnaryInfC{$\psi$}
          \DisplayProof
        \end{center}
    \end{enumerate}

    \par \tb{Example}: Prove that $p \land q, r \lprove q \land r$.
    \snoteb{Example 1.4, page 6}
      \begin{logicproof}{1} % 1 = maximum number of subproofs
        p \land q & premise \\
        r         & premise \\
        q         & $\land e_2$ 1 \\
        q \land r & $\land i$ 3, 2
      \end{logicproof}

  \subsection{Rules for Double Negation}
    \begin{enumerate}[a.]
      \item \tb{Introduction Rule} is denoted by $\lnotnot i$.
        \begin{center}
          \AxiomC{$\phi$}
          \RightLabel{$\lnotnot i$}
          \UnaryInfC{$\lnotnot \phi$}
          \DisplayProof
        \end{center}

      \item \tb{Elimination Rule} is denoted by $\lnotnot e$.
        \begin{center}
          \AxiomC{$\lnotnot \phi$}
          \RightLabel{$\lnotnot e$}
          \UnaryInfC{$\phi$}
          \DisplayProof
        \end{center}
    \end{enumerate}

    \par \tb{Example}: Prove the sequent $p, \lnotnot (q \land r) \lprove \lnotnot p \land r$.
    \snoteb{Example 1.5, page 8}
      % =================================== [[[
      \begin{logicproof}{1} % 1 = maximum number of subproofs
        p                    & premise \\
        \lnotnot (q \land r) & premise \\
        \lnotnot p           & $\lnotnot i$ 1 \\
        q \land r            & $\lnotnot e$ 2 \\
        r                    & $\land e_2$ 4 \\
        \lnotnot p \land r   & $\land i$ 3, 5
      \end{logicproof}
      % =================================== ]]]

  \subsection{Rules for Implication}
    \begin{enumerate}[a.]
      \item \tb{Elimination Rule} is denoted by $\limpl e$ and read \tb{``implies-elimination"}. A, also known as \tb{modus ponens}.
        % ================================= [[[
        \begin{center}
          \AxiomC{$\phi$}
          \AxiomC{$\phi \limpl \psi$}
          \RightLabel{$\limpl e$}
          \BinaryInfC{$\psi$}
          \DisplayProof
        \end{center}
        % ================================= ]]]

      \par A well-known related rule is modus tollens, denoted by MT. It is also an elimination rule.
        % ================================= [[[
        \begin{center}
          \AxiomC{$\phi \limpl \psi$}
          \AxiomC{$\lnot \psi$}
          \RightLabel{MT.}
          \BinaryInfC{$\lnot \phi$}
          \DisplayProof
        \end{center}
        % ================================= ]]]

        \par \tb{Example}: Prove that $p \limpl (q \limpl r), p, \lnot r \lprove \lnot q$
      % =================================== [[[
        \begin{logicproof}{1} % 1 = maximum number of subproofs
          p \limpl (q \limpl r) & premise \\
          p                     & premise \\
          \lnot r                & premise \\
          q \limpl r            & $\limpl e$ 2, 1 \\
          \lnot q               & MT 4, 3
        \end{logicproof}
        % =================================== ]]]

      \item \tb{Introduction Rule} is denoted by $\limpl i$ and read \tb{``implies-introduction"}.
        % ================================= [[[
        \newsavebox\ImplIntroAssump
        \sbox\ImplIntroAssump{
          \fbox{
            \AxiomC{$\phi$}
            \noLine
            \UnaryInfC{$\vdots$}
            \noLine
            \UnaryInfC{$\psi$}
            \DisplayProof
          }
        }

        \begin{center}
          \AxiomC{\usebox\ImplIntroAssump}
          \RightLabel{$\limpl i$}
          \UnaryInfC{$\phi \limpl \psi$}
          \DisplayProof
        \end{center}
        % ================================= ]]]

        \par Here, to prove $\phi \limpl \psi$, we need to make a temporary \tb{assumption} $\phi$. \tb{A box} is opened when \tb{an assumption} is made. When the proof no longer depends on the assumption, the box is closed.
    \end{enumerate}

    \snoteb{Example page 11}: Prove that: $p \limpl q \lprove \lnot q \limpl \lnot p$
      % =================================== [[[
        \begin{logicproof}{1} % 1 = maximum number of subproofs
          p \limpl q             & premise \\

          \begin{subproof}
            \lnot q & assumption \\
            \lnot p & MT 1, 2
          \end{subproof}

          \lnot q \limpl \lnot p & $\limpl i$ 2-3
        \end{logicproof}
      % =================================== ]]]

    \snoteb{Example 1.9, page 13}: Prove that: $\lnot q \limpl \lnot p \lprove p \limpl \lnotnot q$
      % =================================== [[[
        \begin{logicproof}{1} % 1 = maximum number of subproofs
          \lnot q \limpl \lnot p & premise \\

          \begin{subproof}
            p          & assumption \\
            \lnotnot p & $\lnotnot i$ 2 \\
            \lnotnot q & MT 1, 3
          \end{subproof}

          p \limpl \lnotnot q & $\limpl i$ 2-4
        \end{logicproof}
      % =================================== ]]]

    \par $\lprove p \limpl p$ express that: \tb{the argumentation for} $p \limpl p$ does not depend on any premises at all.

    \snoteb{Definition 1.10, page 13}: Logical formulas $\phi$ with valid sequent $\lprove \phi$ are \tb{theorems}.

    \par \tb{Note}: To prove a sequent with the consequence having an implication $p \to q$, we need to make the assumption $p$.

    \par If $P \lprove Q$ and $Q \lprove P$ are both correct, then we can write $P \lproveeq Q$.


  \subsection{Rules for Disjunction (OR)}

    \begin{enumerate}[a.]
      \item \tb{Introduction Rule} is denoted by $\lor i$.
        \begin{center}
          \AxiomC{$\phi$}
          \RightLabel{$\lor i_1$}
          \UnaryInfC{$\phi \lor \psi$}
          \DisplayProof
          \hskip 2cm
          \AxiomC{$\psi$}
          \RightLabel{$\lor i_2$}
          \UnaryInfC{$\phi \lor \psi$}
          \DisplayProof
        \end{center}

      \item \tb{Elimination Rule} is denoted by $\lor e$.
        \begin{center}
          \newsavebox\DisjElimAssumpI
          \sbox\DisjElimAssumpI{
            \fbox{
              \AxiomC{$\phi$}
              \noLine
              \UnaryInfC{$\vdots$}
              \noLine
              \UnaryInfC{$\chi$}
              \DisplayProof
            }
          }

          \newsavebox\DisjElimAssumpII
          \sbox\DisjElimAssumpII{
            \fbox{
              \AxiomC{$\psi$}
              \noLine
              \UnaryInfC{$\vdots$}
              \noLine
              \UnaryInfC{$\chi$}
              \DisplayProof
            }
          }

          \AxiomC{$\phi \lor \psi$}
          \AxiomC{\usebox\DisjElimAssumpI}
          \AxiomC{\usebox\DisjElimAssumpII}
          \RightLabel{$\lor e$}
          \TrinaryInfC{$\chi$}
          \DisplayProof
        \end{center}
    \end{enumerate}

    \snoteb{Example 1.16, page 18}: Prove the sequent $q \limpl r \lprove p \lor q \limpl p \lor r$ is valid.
      % =================================== [[[
        \begin{logicproof}{3} % 3 = maximum number of subproofs
          q \limpl r & premise \\
          \begin{subproof}
            p \lor q & assumption \\

            \begin{subproof}
              p        & assumption \\
              p \lor r & $\lor i_1$ 3
            \end{subproof}

            \begin{subproof}
              q        & assumption \\
              r        & $\limpl e$ 1, 5 \\
              p \lor r & $\lor i_2$ 6
            \end{subproof}

            p \lor r & $\lor e$ 2, 3-4, 5-7
          \end{subproof}

          p \lor q \limpl p \lor r
        \end{logicproof}
      % =================================== ]]]

  \subsection{Copy Rule}
    \par The \tb{copy rule} allows repeating something that is known already, \ti{unless it belongs to a temporary assumption whose box has already been closed}.
    \snoteb{Example page 20}: Prove the sequent $\lprove p \limpl (q \limpl p)$
    \begin{logicproof}{2}
      \begin{subproof}
        p & assumption \\
        \begin{subproof}
          q & assumption \\
          p & copy 1
        \end{subproof}
        q \limpl p & $\limpl i$ 2-3
      \end{subproof}
      q \limpl (q \limpl p) & $\limpl i$ 1-4
    \end{logicproof}

  \subsection{Rules for Contradiction}
    \par A \tb{contradiction} is an expression in either of the two forms
    \[
      \phi \land \lnot \phi \qquad \lnot \phi \land \phi
    \]
    where $\phi$ is any formula. A contradiction is denoted by $\lcontrad$.
    \par \tb{Elimination Rule}: This rule says that: \ti{a contradiction can prove anything}.
      \begin{center}
        \AxiomC{$\lcontrad$}
        \RightLabel{$\lcontrad e$}
        \UnaryInfC{$\phi$}
        \DisplayProof
      \end{center}

  \subsection{Rules for Negation}
    \begin{enumerate}[a.]
      \item \tb{Elimination Rule} is denoted by $\lnot e$ and read \tb{``not-elimination"}.
        \begin{center}
          \AxiomC{$\phi$}
          \AxiomC{$\lnot \phi$}
          \RightLabel{$\lnot e$}
          \BinaryInfC{$\lcontrad$}
          \DisplayProof
        \end{center}

      \snoteb{Example 1.20, page 21}: Prove the sequent: $\lnot p \lor q \lprove p \limpl q$
        \begin{logicproof}{4}
          \lnot p \lor q & premise \\
          \begin{subproof}
            \lnot p & assumption \\

            \begin{subproof}
              p         & assumption \\
              \lcontrad & $\lnot e$ 3, 2 \\
              q         & $\lcontrad e$ 4
            \end{subproof}

            p \limpl q  & $\limpl i$ 3-5
          \end{subproof}

          \begin{subproof}
            q & assumption \\
            \begin{subproof}
              p & assumption \\
              q & copy 7
            \end{subproof}

            p \limpl q & $\limpl i$ 8-9
          \end{subproof}

          p \limpl q & $\lor e$ 1, 2-6, 7-10
        \end{logicproof}

      \item \tb{Introduction Rule} is denoted by $\lnot i$.
        \begin{center}
          \newsavebox\NegaIntrAssump
          \sbox\NegaIntrAssump{
            \fbox{
              \AxiomC{$\phi$}
              \noLine
              \UnaryInfC{$\vdots$}
              \noLine
              \UnaryInfC{$\lcontrad$}
              \DisplayProof
            }
          }

          \AxiomC{\usebox\NegaIntrAssump}
          \RightLabel{$\lnot i$}
          \UnaryInfC{$\lnot \phi$}
          \DisplayProof
        \end{center}
    \end{enumerate}

    \snoteb{Example 1.21, page 22}: Prove the sequent: $p \limpl q, p \limpl \lnot q \lprove \lnot p$
    \begin{logicproof}{1}
      p \limpl q       & premise \\
      p \limpl \lnot q & premise \\
      \begin{subproof}
        p         & assumption \\
        q         & $\limpl e$ 1, 3 \\
        \lnot q   & $\limpl e$ 2, 3 \\
        \lcontrad & $\lnot e$ 4, 5
      \end{subproof}
      \lnot p & $\lnot i$ 3-6
    \end{logicproof}

  \hii{Derived Rules}
    \hiii{Modus Tollens}
      \begin{center}
        \AxiomC{$\phi \limpl \psi$}
        \AxiomC{$\lnot \psi$}
        \RightLabel{MT}
        \BinaryInfC{$\lnot \phi$}
        \DisplayProof
      \end{center}

    \hiii{Proof By Contradiction (PCB)}
      \begin{center}
        \newsavebox\PCBAsump
        \sbox\PCBAsump{
          \fbox{
            \AxiomC{$\lnot \phi$}
            \noLine
            \UnaryInfC{$\vdots$}
            \noLine
            \UnaryInfC{$\lcontrad$}
            \DisplayProof
          }
        }

        \AxiomC{\usebox\PCBAsump}
        \RightLabel{PBC}
        \UnaryInfC{$\phi$}
        \DisplayProof
      \end{center}

    \hiii{Law of the Excluded Middle (LEM)}
      \par This rule says that: \ti{the statement $\phi \lor \lnot \phi$ is always \tb{True}}.
      \begin{center}
        % \AxiomC{$\phi$}
        % \AxiomC{$\lnot \phi$}
        \AxiomC{}
        \RightLabel{LEM}
        \UnaryInfC{$\phi \lor \lnot \phi$}
        \DisplayProof
      \end{center}

  \hiii{Provable Equivalence}
    \par Let $\phi$ and $\psi$ be formulas of propositional logic. We say that $\phi$ and $\psi$ are \tb{provably equivalent} iff (we write ``iff" for ``if, and only if" in the sequel) the sequents $\phi \lprove \psi$ and $\psi \lprove \phi$ are valid; that is, there is a proof of $\psi$ from $\phi$ and another one going the other way around. As seen earlier, we denote that $\phi$ and $\psi$ are provably equivalent by $\phi \lproveeq \psi$.

\hi{Propositional Logic as a Formal Language}
  \snoteb{Definition 1.27, page 32-33}: The \tb{well-formed} formulas of propositional logic are those which we obtain by using the construction rules below (and only those) finitely many times:
  \begin{itemize}
    \item \tb{atom}: every propositional atom $p, q, r, \ldots$ and $p_1, \ldots$  is a well-formed formula
    \item $\lnot$: If $\phi$ is a well-formed formula, then so is $\lnot \phi$.
    \item $\land$: If $\phi$ and $\psi$ are well-formed formulas, then so is $\phi \land \psi$.
    \item $\lor$: If $\phi$ and $\psi$ are well-formed formulas, then so is $\phi \land \psi$.
    \item $\limpl$: If $\phi$ and $\psi$ are well-formed formulas, then so is $\phi \land \psi$.
  \end{itemize}

  \par The above definition can be written in a defining grammar called \tb{Backus Naur form (BNF)}. In that form, the definition reads more compactly as:
    \[
      \phi ::= p | (\lnot \phi) | (\phi \land \psi) | (\phi \lor \psi) | (\phi \limpl \psi)
    \]
    where
    \begin{itemize}
      \item $p$ stands for \tb{any atomic proposition}
      \item every occurence of $\phi$ to the right of $::=$ stands for any already constructed formula.
    \end{itemize}

  \par To verify if a formula is well-formed or not, we can draw a syntax tree.

\hi{Semantics of Propositional Logic}
  \hii{The meaning of logical connectives}
  \snoteb{Definition 1.28, page 37}:
    \begin{itemize}
      \item The set of truth values contains two elements: True $\lT$ and False $\lF$.
      \item A \tb{valuation} of \tb{model} of a formula $\phi$ is an assignment of each propositional atom in $\phi$ to a truth value.
    \end{itemize}
%  \par Notation:
    %\[
      %\phi_1, \ldots, \phi_n \models \psi
    %\]

  \hii{Mathematical Induction}
    \begin{itemize}
      \item \tb{Base case}: The natural number 1 has property $M$, i.e. we have a proof of $M(1)$.
      \item \tb{Inductive step}: If $n$ is a natural number which we assume to have property $M(n)$, then we can show that $n + 1$ has property $M(n + 1)$; i.e. we have a proof of $M(n) \limpl M(n + 1)$.
    \end{itemize}
    \par The assumption of $M(n)$ in the inductive step is called the induction hypothesis.

    \par \tb{Course-of-values induction}: A variant of mathematical induction in which the induction hypothesis for proving $M(n + 1)$ is not just $M(n)$, but the conjunction $M(1) \land M(2) \land \ldots \land M(n)$. In that variant, called course-of-values induction, there doesn't have to be an explicit base case at all – everything can be done in the inductive step.

    \snoteb{Definition 1.32, page 44}: Given a well-formed formula $\phi$, we define its height to be
1 plus the length of the longest path of its parse tree.
    \snoteb{Theorem 1.33, page 45}: For every well-formed propositional logic formula, the number of left brackets is equal to the number of right brackets.

  \hii{Soundness of Propositional Logic}
    \snoteb{Definition 1.34, page 46}: If, for all valuations in which all $\phi_1, \ldots, \phi_n$ evaluate to $\lT$, $\psi$ evaluates to $\lT$ as well, we say that:
    \[
      \phi_1, \phi_2, \ldots, \phi_n \models \psi
    \]
    holds and call $\models$ the \tb{semantic entailment} relation.

    \snoteb{Theorem 1.35, page 46}: \tb{Soundness}: Let $\phi_1, \ldots, \phi_n$ and $\psi$ be propositional logic formulas. If $\phi_1, \ldots, \phi_n \lprove \psi$ is valid, then $\phi_1, \ldots, \phi_n \models \psi$

    \par The soundness of propositional logic is useful in ensuring the non-existence of a proof for a given sequent.

  \hii{Completeness of Propositional Logic}
    \par Whenever $\phi_1, \ldots, \phi_n \models \psi$ holds, then there exists a natural deduction proof for the sequent $\phi_1, \ldots, \phi_n \lprove \psi$. Combined with the soundness result, we obtain:
    \[
      \phi_1, \ldots, \phi_n \lprove \psi \text{ is valid iff } \phi_1, \ldots, \phi_n \models \psi \text{holds}
    \]
    \snoteb{Definition 1.36}: A formula of propositional logic $\phi$ is called a \tb{tautology} iff it evaluates to $\lT$ under all its valuations, i.e. iff $\models \phi$.
    \snoteb{Theorem 1.37}: If $\models \eta$ holds, then $\lprove \eta$ is valid. In other words, if $\eta$ is a tautology, then $\eta$ is a theorem.
    \snoteb{Corollary 1.39}: \tb{Soundness and Completeness}: Let $\phi_1, \ldots, \phi_n, \psi$ be formulas of propositional logic. Then $\phi_1, \ldots, \phi_n \models \psi$ holds iff the sequent $\phi_1, \ldots, \phi_n \lprove \psi$ is valid.


\hi{Normal Forms}
  \hii{Semantic equivalence, satisfiability and validity}
    \par Let $\phi$ and $\psi$ be formulas of propositional logic. We say that $\phi$ and $\psi$ are semantically equivalent iff $\phi$ ? $\psi$ and $\psi$ ? $\phi$ hold. In that case we write $\phi \equiv \psi$. Further, we call $\phi$ valid if $\models \phi$ holds.
    \snoteb{Lemma 1.41, page 55}:  Given formulas $\phi_1, \ldots, \phi_n$ and $\psi$ of propositional logic. $\phi_1, \ldots, \phi_n \models \psi$ holds iff $\models \phi_1 \limpl (\phi_2 \limpl \ldots \limpl (\phi_n \limpl \psi))$ holds.

    \par \tb{Proof}:
      \begin{itemize}
        \item \tb{Step 1}: Suppose that $\phi_1, \ldots, \phi_n \models \psi$ holds.
          \par We need to verify that $\models \phi_1 \limpl (\phi_2 \limpl \ldots \limpl (\phi_n \limpl \psi))$ is a tautology. This formula evaluates to $\lF$ only if $\phi_1, \ldots, \phi_n$ all evaluate to $\lT$ and $\psi$ evaluates to $\lF$. But this contradicts the fact that $\phi_1, \ldots, \phi_n \models \psi$ holds. Thus, $\models \phi_1 \limpl (\phi_2 \limpl \ldots \limpl (\phi_n \limpl \psi))$ holds.
        \item \tb{Step 2}: Suppose that $\models \phi_1 \limpl (\phi_2 \limpl \ldots \limpl (\phi_n \limpl \psi))$.
          \par If $\phi_1 ,\phi_2, \ldots, \phi_n$ are all true under some valuation, then $\psi$ has to be true as well for that same valuation. Therefore, $\phi_1, \ldots, \phi_n \models \psi$ holds.
      \end{itemize}

    \par In some cases, we want to transform formulas into ones which:
      \begin{itemize}
        \item don't contain $\limpl$ at all
        \item the occurrences of $\land$ and $\lor$ are confined to separate layers such that validity checks are easy.
      \end{itemize}

    \par This can be done by:
    \begin{itemize}
      \item using the equivalence $\phi \limpl \psi \equiv \lnot \phi \lor \psi$ to remove all occurrences of $\limpl$ from a formula.
      \item specifying an algorithm that takes a formula without any $\limpl$ into a \tb{normal form} (still without $\limpl$) for which checking validity is easy.
    \end{itemize}

    \snoteb{Definition 1.42, page 56}: A formula $C$ is in \tb{conjunctive normal form} (CNF) if it is a conjunction of clauses, where:
      \begin{itemize}
        \item each literal $L$ is either an atom $p$ or the negation of an atom $\lnot p$.
        \item each clause $D$ is a disjunction of literals.
      \end{itemize}
      \begin{align*}
        L &::= p | \lnot p \\
        D &::= L | L \lor D \\
        C &::= D | D \land C \\
      \end{align*}

      \snote{Example}:
      \begin{itemize}
        \item The formula $(\lnot q \lor p \lor r) \land (\lnot p \lor r) \land q$ is in CNF.
        \item The formula $(\lnot (q \lor p) \lor r) \land (q \lor r)$ is not in CNF, since $q \lor p$ is not a literal.
      \end{itemize}

    \snoteb{Lemma 1.43, page 56}: A \ti{disjunction of literals} $L_1 \lor \ldots \lor L_m$ is valid iff there are $1 \leq i, j \leq m$ such that $L_i$ is $\lnot L_j$.

    \snoteb{Definition 1.44, page 57}: Given a formula $\phi$ in propositional logic. We say that $\phi$ is \tb{satisfiable} if it has a \ti{valuation} in which it evaluates to $\lT$.

    \snoteb{Proposition 1.45, page 57}: Let $\phi$ be a formula of propositional logic. Then $\phi$ is satisfiable iff $\lnot \phi$ is not valid.

    \snoteb{Theorem - Transformation to CNF}: Every formula in the propositional calculus can be transformed into an equivalent formula in CNF.

    \snoteb{Transforming Truth Table to CNF}:
      \begin{itemize}
        \item \tb{Step 1}: Find all rows in the truth table that evaluate to $\lF$.
        \item \tb{Step 2}: With each row that we found in step 1, construct a disjunction.
        \item \tb{Step 3}: The CNF is the conjunction of all disjunctions in step 2.
      \end{itemize}

  \hii{Conjunctive normal forms and validity}

  \hii{Extra: Algorithm for CNF Transformation}
    \begin{itemize}
      \item \tb{Step 1}: Eliminate implication, using:
        \[
          \phi_1 \limpl \phi_2 \equiv \lnot \phi_1 \lor \phi_2
        \]
      \item \tb{Step 2}: Push all negations inward using De Morgan's law:
        \[
          \lnot (\phi_1 \land \phi_2) \equiv (\lnot \phi_1 \lor \lnot \phi_2)
        \]
        \[
          \lnot (\phi_1 \lor \phi_2) \equiv (\lnot \phi_1 \land \lnot \phi_2)
        \]
      \item \tb{Step 3}: Eliminate double negations using the equivalence:
        \[
          \lnotnot \phi_1 \equiv \phi_1
        \]
      \item \tb{Step 4}: Use distributive laws to eliminate conjunctions within disjunctions:
        \[
          \phi_1 \lor (\phi_2 \land \phi_3) \equiv (\phi_1 \lor \phi_2) \land (\phi_1 \lor \phi_3)
        \]
        \[
          (\phi_1 \land \phi_2) \lor \phi_3 \equiv (\phi_1 \lor \phi_3) \land (\phi_2 \lor \phi_3)
        \]
    \end{itemize}

  \hii{Horn clauses and satisfiability}