\chapter{Processes}

\hi{Process Concept}
  \hii{The Process}
    \par A \tb{process} is a running program.
    \par A \tb{process} consists of:
      \begin{itemize}
        \item \tb{text section}: the program code
        \item \tb{program counter}: a value stored in the processor's register
          which keeps track of the currently executed instruction of the
          program
        \item \tb{stack}: the memory section which stores temporary data (local
          variables, function parameters, etc.)
        \item \tb{data section}: the memory section which stores the global
          variables
        \item \tb{heap}: memory section that is dynamically allocated during
          process run time.
      \end{itemize}
    \par \ti{Distinguishing between a program and a process}:
      \begin{itemize}
        \item A \ti{program} is a passive entity, such as a file containing a
          list of instructions stored on disk, often called an executable file.
        \item A \ti{process} is an active entity, with a program counter
          specifying the next instruction to execute and a set of associated
          resources.
      \end{itemize}
  
  \hii{Process State}
    \par As a process executes, it changes \tb{state}. A process may be in one
      of the following states:
      \begin{itemize}
        \item \tb{New}: The process is being created
        \item \tb{Running}: Instructions are being executed
        \item \tb{Waiting}: The process is waiting for some event to occur
          (such as an I/O completion or reception of a signal)
        \item \tb{Ready}: The process is waiting to be assigned to a processor
        \item \tb{Terminated}: The process has finished execution.
      \end{itemize}

    \img[width=12cm]{img/process-state.png}

  \hii{Process Control Block (PCB)}
    \par Each process is represented in the operating system by a \tb{process
      control block (PCB)}, or \ti{task control block}.
    \par A PCB contains information associated with a specific process,
    including:
      \begin{itemize}
        \item \tb{Process state}
        \item \tb{Program counter (PC)}
        \item \tb{CPU registers}
        \item \tb{CPU-scheduling information}
        \item \tb{Memory-management information}
        \item \tb{I/O status information}
      \end{itemize}

  \hii{Threads}


\hi{Process Scheduling}
  \hii{Objectives}
    \begin{itemize}
      \item The objective of \ti{multiprogramming} is to maximize CPU
        utilization by having some process running at all times.
      \item The objective of \ti{time sharing} is to switch the CPU among
        processes so frequently that users can interact with each program while
        it is running.
    \end{itemize}
    \par To meet these objectives, the \tb{process scheduler} selects an
      available process for program execution on the CPU.

  \hii{Scheduling Queues}
    \par Some types of queues:
      \begin{itemize}
        \item \tb{Job queue}: The queue of all processes in the system.
        \item \tb{Ready queue}: The list of  processes that are residing in main
          memory and are ready and waiting to execute.
        \item \tb{Device queue}: The list of processes waiting for a particular
          I/O device. Each device has its own device queue.
      \end{itemize}
    \par A new process is initially put in the ready queue. It waits there
      until it is selected for execution, or is \tb{dispatched}.

  \hii{Schedulers}
    \hiii{Types of Schedulers}
      \par A process migrates among various scheduling queues throughout its
        lifetime. The operating system must select, for scheduling purposes,
        processes from these queues in some fashion. The selection process is
        carried out by the appropriate \tb{scheduler}.
      \par Types of scheduler:
      \begin{itemize}
        \item \tb{Long-term scheduler/Job scheduler}: In a batch system, more
          processes are submitted than can be executed immediately. These
          processes are sent to a mass-storage device (typically a disk), where
          they are kept for later execution.
        \item \tb{Short-term scheduler/CPU scheduler}: A CPU scheduler selects
          from among the processes that are ready to execute and allocates the
          CPU to one of them.
      \end{itemize}
      \par The \tb{short-term scheduler} executes much more frequently than a
        \tb{long-term scheduler}.

    \hiii{Types of Processes}
      \par Most processes can be described as either I/O bound or CPU bound:
      \begin{itemize}
        \item An \tb{I/O-bound process} is one that spends more of its time
          doing I/O than it spends doing computations.
        \item A \tb{CPU-bound process}, in contrast, generates I/O request
          infrequently, using more of its time doing computations.
      \end{itemize}

