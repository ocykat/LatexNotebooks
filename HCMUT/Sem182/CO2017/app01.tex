\chapter*{UNIX Process Memory}

\par Memory of a process can be divided into segments:
  \begin{itemize}
    \item \tb{code segment}: contains the binary code of the program which is running as a process
    \item \tb{data segment}: contains the initialized global variables and data structures.
    \item \tb{BSS (block started by symbol}:  contains the uninitialized global data structures
    \item \tb{stack segment}: contains the local variables, return addresses, etc. for a particular process
  \end{itemize}

\par A process can execute in two modes:
  \begin{itemize}
    \item \tb{Kernel mode}: When a process makes a system call, the kernel takes control and does the requested service on behalf of the process. The process is said to be ``in kernel space” during this time.
    \item \tb{User mode}: Normal mode of a process. The process is said to be ``in userland" during this time.
  \end{itemize}

\par Userland’s view of the segments:
\begin{itemize}
  \item \tb{code segment}: consists of the code of the program, including all functions. If \lstinline{p = foo()} where \lstinline{foo()} is an arbitrary function, then \lstinline{p} must point to some location within the code segment.
  \item \tb{data segment}: consists of the initialized global variables of a program. If \lstinline{N} is a global variable, then the address of \lstinline{N} must be in the data segment.
  \item \tb{BSS}: consists of the uninitialized global variables of a program. If \lstinline{N} is a global variable, then the address of \lstinline{N} must be in the data segment.
  \item \tb{stack}: consists of local variables.
  \item \tb{heap}: consists of memory dynamically allocated during process run time.
\end{itemize}


