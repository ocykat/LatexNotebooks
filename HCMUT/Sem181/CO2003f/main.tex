\documentclass[9pt, landscape, a4paper]{article}
% MARGINS
\usepackage{geometry}
\geometry{top=0.75cm,bottom=0.75cm,left=1cm,right=1cm,includehead,includefoot}

% FONT
\renewcommand{\familydefault}{\sfdefault}

% LINE SPACING
% \setlength{\parskip}{6pt}

% HEADER % FOOTER
\usepackage{fancyhdr}
\pagestyle{fancy}
\fancyhead[L]{HCM City University of Technology}
\fancyhead[R]{Dept. of Computer Science \& Technology}
\fancyfoot[L]{Nhat M. Nguyen - 1752039}
\fancyfoot[R]{\thepage}

\fancypagestyle{plain}
{
\fancyhead[L]{HCM City University of Technology}
\fancyhead[R]{Dept. of Computer Science \& Technology}
\fancyfoot[L]{Nhat M. Nguyen}
\fancyfoot[C]{}
\fancyfoot[R]{\thepage}
}

% == Multi-column ==
\usepackage{multicol}
\setlength{\columnsep}{0.5cm}
\setlength{\columnseprule}{0.2pt}
\newcommand{\colbreak}{\vfill\null\columnbreak}

% == TOC ==
\usepackage{tocloft}
\renewcommand*{\ttdefault}{pcr}
\renewcommand\cftsecfont{\fontsize{8}{9}\bfseries}
\renewcommand\cftsecpagefont{\fontsize{8}{9}\mdseries}
\renewcommand\cftsubsecfont{\fontsize{5}{6}\mdseries}
\renewcommand\cftsubsecpagefont{\fontsize{5}{6}\mdseries}
\renewcommand\cftsecafterpnum{\vspace{-1ex}}
\renewcommand\cftsubsecafterpnum{\vspace{-1ex}}

% HEADINGS
\setcounter{secnumdepth}{4}
\newcommand{\hi}{\section}
\newcommand{\hii}{\subsection}
\newcommand{\hiii}{\subsubsection}
\newcommand{\hiiiBEGIN}[1]{\subsubsection \begin{enumerate}}
\newcommand{\hiiiEND}{\end{enumerate}}
\newcommand{\hiv}{\item\textbf}

% == Indentation ==
\usepackage{indentfirst}

% TEXT
% Bold, italic, underlined text
\newcommand{\tb}[1]{\textbf{#1}}
\newcommand{\ti}[1]{\textit{#1}}
\newcommand{\tbi}[1]{\textbf{\textit{#1}}}
\newcommand{\tu}[1]{\underline{#1}}
\newcommand{\tbu}[1]{\textbf{\underline{#1}}}

% FOOTNOTE
% + one footnote stays on one page
\interfootnotelinepenalty=10000
\newcommand{\FNM}{\footnotemark}
\newcommand{\SFNM}{\footnotemark[\value{footnote}]}
\newcommand{\FNT}{\footnotetext}

% IMAGES
\usepackage{graphicx}
\usepackage{subcaption}
\usepackage{float}
\newcommand{\img}[2][]
  {
    \begin{figure}[H]
      \centering
      \includegraphics[#1]{#2}
    \end{figure}
  }

% MATH
\usepackage{amsmath}
\usepackage{amssymb}
\usepackage{gensymb}
\newcommand{\sqbr}[1]{[#1]}
\newcommand{\ls}{<}
\newcommand{\gr}{>}
% equation boxes
\usepackage{empheq}
\newenvironment{eqbox}
  {\setkeys{EmphEqEnv}{align}\setkeys{EmphEqOpt}{box=\fbox}\EmphEqMainEnv}
  {\endEmphEqMainEnv}
% bold in math env
\usepackage{amsbsy}
% word above equal sign
\usepackage{mathtools}
% \newcommand\eqsign[1]{\stackrel{\mathclap{\normalfont\mbox{#1}}}{=}}
\newcommand\eqsign[1]{\mathrel{\overset{\makebox[0pt]{\mbox{\normalfont\tiny\sffamily #1}}}{=}}}
% macros
\newcommand{\vt}{\overrightarrow}
\newcommand{\avg}{\overline}
\newcommand{\ra}{\Rightarrow}
\newcommand{\lra}{\Leftrightarrow}
\newcommand{\dt}{\Delta}
\newcommand{\dif}[2]{\frac{d #1}{d #2}}
\newcommand{\pd}[2]{\frac{\partial #1}{\partial #2}}
\newcommand{\pdd}[3]{\frac{\partial #1}{\partial #2 \partial #3}}
\newcommand{\INT}{\int \limits}
\newcommand{\IINT}{\iint \limits}
\newcommand{\IIINT}{\iiint \limits}
\newcommand{\OINT}{\oint \limits}
\newcommand{\SUM}{\sum \limits}
\newcommand{\PROD}{\prod \limits}

% PSEUDOCODE
\usepackage{algpseudocode,algorithm,algorithmicx}
\usepackage{caption}

% \renewcommand{\thealgorithm}{\arabic{chapter}.\arabic{algorithm}}
\newcommand*\Let[2]{\State #1 $\gets$ #2}
\algrenewcommand\algorithmicrequire{\textbf{Precondition:}}
\algrenewcommand\algorithmicensure{\textbf{Postcondition:}}
\newcommand{\TO}{\textrm{ to }}
\newcommand{\AND}{\textrm{ and }}
\newcommand{\OR}{\textrm{ or }}
\newcommand{\LET}[2]{\Let{$#1$}{$#2$}}
\newcommand{\FOR}[1]{\For{$#1$}}
\newcommand{\ENDFOR}{\EndFor}
\newcommand{\IF}[1]{\If{$#1$}}
\newcommand{\ELSEIF}[1]{\Elseif{$#1$}}
\newcommand{\ELSE}{\Else}
\newcommand{\ENDIF}{\EndIf}
\newcommand{\WHILE}[1]{\While{$#1$}}
\newcommand{\ENDWHILE}{\EndWhile}
\newcommand{\FUNCTION}[2]{\Function{#1}{$#2$}}
\newcommand{\ENDFUNCTION}{\EndFunction}
\newcommand{\RETURN}[1]{\State \Return{$#1$}}
\newcommand{\CALLFUNC}[2]{\Call{#1}{$#2$}}
\newcommand{\CALLPROC}[2]{\Call{#1}{$#2$}}

% === CODE ===
\usepackage{listings}
\usepackage{inconsolata}
\usepackage{color}

\definecolor{dkgreen}{rgb}{0,0.6,0}
\definecolor{gray}{rgb}{0.5,0.5,0.5}
\definecolor{mauve}{rgb}{0.58,0,0.82}

\lstdefinestyle{cpp} {
  language=C++,
  aboveskip=1mm,
  belowskip=1mm,
  basicstyle=\ttfamily,
  numberstyle=\tiny\color{gray},
  commentstyle = \color{dkgreen},
  keywordstyle = \color{blue},
  stringstyle = \color{mauve},
  breaklines=true,
  postbreak=\mbox{\textcolor{red}{$\hookrightarrow$}\space},
}

\lstdefinestyle{java} {
  language=Java,
  aboveskip=1mm,
  belowskip=1mm,
  basicstyle=\ttfamily,
  numberstyle=\tiny\color{gray},
  commentstyle = \color{dkgreen},
  keywordstyle = \color{blue},
  stringstyle = \color{mauve},
  breaklines=true,
  postbreak=\mbox{\textcolor{red}{$\hookrightarrow$}\space},
}

% INFO
\title{\vspace{-4ex}\Large{Data Structures \& Algorithms Notebook}}
\author{Nhat M. Nguyen - ID: 1752039}
\date{January 2019}

\begin{document}

  \maketitle
  \img[width=3cm]{logo.jpeg}
  \setcounter{tocdepth}{2}
  \begin{multicols}{2}
    \tableofcontents
  \end{multicols}
  \begin{center}
  \ti{Allowed in examination room}
  \end{center}
\clearpage
\begin{multicols}{2}
\hi{List}
  \hii{Array-List}
    \lstinputlisting[style=cpp]{code/structures/array-list.h}    
  \hii{Singly-Linked-List}
    \lstinputlisting[style=cpp]{code/structures/singly-linked-list.h}
  \hii{Doubly-Linked-List}
    \lstinputlisting[style=cpp]{code/structures/doubly-linked-list.h}  



\hi{Stack \& Queue}
  \hii{Stack}
    \lstinputlisting[style=cpp]{code/structures/stack.h}  
  \hii{Queue}
    \lstinputlisting[style=cpp]{code/structures/queue.h}  



\hi{Trees}
  \hii{Binary Tree}
    \lstinputlisting[style=cpp]{code/structures/binary-tree.h}  
  \hii{Binary Search Tree}
    \lstinputlisting[style=cpp]{code/structures/binary-search-tree.h}  
  \hii{AVL Tree}
    \lstinputlisting[style=cpp]{code/structures/avl-tree.h}



\hi{Heap}  
    \lstinputlisting[style=cpp]{code/structures/heap.h}



\hi{Sorting}
  \hii{Bubble Sort}
      \lstinputlisting[style=cpp]{code/algorithms/sort/bubble-sort.h}
  \hii{Counting Sort}
        \lstinputlisting[style=cpp]{code/algorithms/sort/counting-sort.h}

  \hii{Heap Sort}
        \lstinputlisting[style=cpp]{code/algorithms/sort/heap-sort.h}

  \hii{Insertion Sort}
        \lstinputlisting[style=cpp]{code/algorithms/sort/insertion-sort.h}

  \hii{Merge Sort}
        \lstinputlisting[style=cpp]{code/algorithms/sort/merge-sort.h}

  \hii{Quick Sort}
    \lstinputlisting[style=cpp]{code/algorithms/sort/quick-sort.h}

  \hii{Radix Sort}
    \lstinputlisting[style=cpp]{code/algorithms/sort/radix-sort.h}

  \hii{Selection Sort}
    \lstinputlisting[style=cpp]{code/algorithms/sort/selection-sort.h}

  \hii{Shell Sort}
    \lstinputlisting[style=cpp]{code/algorithms/sort/shell-sort.h}



\hi{Binary Search}
    \lstinputlisting[style=java]{code/implementations/BinarySearch.java}  



\hi{Graph}
  \hii{BFS}
    \lstinputlisting[style=cpp]{code/implementations/bfs.cpp}
  \hii{DFS}
    \lstinputlisting[style=cpp]{code/implementations/dfs.cpp}
  \hii{Topological Sort}
    \hiii{DFS Topo-Sort}
        \lstinputlisting[style=cpp]{code/implementations/topo-dfs.cpp}
    \hiii{Kahn's Algorithm}
        \lstinputlisting[style=cpp]{code/implementations/topo-kahn.cpp}
  \hii{Dijkstra}
        \lstinputlisting[style=cpp]{code/implementations/dijkstra.cpp}

\end{multicols}
\end{document}