\begin{multicols}{2}

\hi{ISA}
  \hii{Big-endian and Little-endian}
    \par There are two types of architecture:
    \begin{itemize}
      \item Using the address of the leftmost byte, or MSb, of the word address.
        This type of architecture is called \tb{big-endian}.
      \item Using the address of the rightmost byte, or LSb, byte of the word
        adress. This type of architecture is called \tb{little-endian}.
    \end{itemize}

  \hii{Addressing for 32-bit Immediate and Addresses}
    \hiii{Immediate}
    \begin{lstlisting}
  # 4,000,000 decimal = 0000 0000 0011 1101 0000 1001 0000 0000 binary
  lui $s0, 61        # 61 decimal = 0000 0000 0011 1101 binary = upper 16 bits
  ori $s0, $s0, 2304 # 61 decimal = 0000 1001 0000 0000 binary = lower 16 bits
    \end{lstlisting}
    \hiii{Addresses}
      \par To deal with that problem: PC-relative addressing
        \begin{align*}
          \mbox{ Program Counter } = \mbox{ Register } + \mbox { Branch Address }
        \end{align*}

  \hii{Stack}
    \par A stack grows from higher addresses to lower addresses.
            \begin{lstlisting}
addi $sp, $sp, -12 # adjust stack to make room for 3 items
...
addi $sp,$sp,12 # adjust stack to delete 3 items
            \end{lstlisting}

\end{multicols}
\noindent\makebox[\linewidth]{\rule{\paperwidth}{0.4pt}}