\chapter{Differential Equations}

\hi{The Initial-Value Problem (Cauchy's Problem)}
  \par \tb{Problem}: Determine the function $y(x)$ that is differentiable
    for $x \geq x_0$ and satisfies:
    \begin{align*}
      \begin{cases}
        \dif{y}{x} = y'(x) = f(x, y(x)) \\
        y(x_0) = y_0
      \end{cases}
    \end{align*}
    where $f(x, y)$ is a continuous two-variable function.
    \par This is called the \tb{initial-value} problem.

\hi{Euler's Method}
  \hii{General Idea of Euler's Method}
    \par The Euler's Method is based on the idea that: the continuous
      approximation to the solution $y(x)$ is not obtained, but
      approximation to $y(x)$ will be generated at various values, called
      \tb{mesh points}, in the interval $[a, b]$.
    \par Suppose that the mesh points are evenly distributed throughout the
      interval $[a, b]$, dividing this segment into $n$ equal subsegments. Then:
      \begin{align*}
        x_k - x_{k - 1} = \frac{b - a}{h} \qquad \mbox{ and } \qquad x_k = x_0 + kh
      \end{align*}
    \par Taking the idea from the Taylor's theorem, geometrically we have:
    \begin{flalign*}
      & y'(x_0) = f(x_0, y_0) && \\
      & \rightarrow \mbox{ The tangent line of the curve } (C): y = y(x)
        \mbox{ at } (x_0, y_0): &&\\
      & y(x) = y_0 + y'(x_0)(x - x_0) &&\\
      & \rightarrow y(x) = y_0 + f(x_0, y(x_0))(x - x_0) &&\\
      & \mbox {Substitute  $x_1$ into $x$, we have:} &&\\
      & y(x_1) \approx y_1 = y_0 + f(x_0, y(x_0))(x_1 - x_0) && \\
      & \rightarrow y(x_1) \approx y_1 = y_0 + hf(x_0, y(x_0)) &&
    \end{flalign*}
      where $y_1$ is the approximation of $y(x_1)$.
    \par We obtain the \tb{general Euler's formula}:
      \begin{equation}
        y(x_k) \approx y_k = y_{k - 1} + hf(x_{k - 1}, y_{k - 1})
        \quad \forall k = 1, 2, \ldots, n
      \end{equation}

  \hii{Improved Euler's Method}
    \par Define:
      \begin{itemize}
        \item $\vec{p_0}$ as the directional vector of the tangent line of $y(x)$
          at $x_0$.
        \item $\vec{p_1}$ as the directional vector of the tangent line of $y(x)$
          at $x_1$.
        \item $d$ as the bisection line of $\vec{p_0}$ and $\vec{p_1}$.
      \end{itemize}
    \par We have:
      \begin{itemize}
        \item The slope of the tangent line at $x_0$:  $m_0 = f(x_0, y(x_0))$
        \item The slope of the tangent line at $x_1$:  $m_1 = f(x_1, y(x_1))$
      \end{itemize}
      then the slope of $d$ would be:
      \begin{align*}
        \frac{m_0 + m_1}{2} = \frac{f(x_0, y(x_0)) + f(x_1, y(x_1))}{2}
      \end{align*}
    \par $y(x_1)$ can be estimated as follow:
      \begin{align*}
      & y(x_1) \approx y_1 = y_0 + h\frac{f(x_0, y(x_0)) + f(x_1, y(x_1))}{2} && \\
      \end{align*}
    \par The Euler's formula can be improved as follow:
      \begin{align*}
        K_1 = hf(x_0, y_0) \\
        K_2 = hf(x_0 + h, y_0 + K_1) \\
        y(x_1) \approx y_1 = y_0 + \frac{1}{2} (K_1 + K_2)
      \end{align*}
