\chapter{Interpolation}

\par \tb{Interpolation}: The process of predicting new data points based on
  a known set of data points.
\par This chapter considers methods of finding the function that fits the
  given data.

\hi{Interpolation and the Lagrange Polynomial}
  \hii{Algebraic Polynomial and Interpolation}
    \par One popular class of function for interpolation is \tb{algebraic
      polynomials}. A algebraic polynomial has the form:
    \begin{align*}
      P_{n}(x) = a_{n} x^{n} + a_{n - 1} x^{n - 1} + \ldots + a_{1}x + a_{0}
    \end{align*}
      where $n$ is a nonnegative integer and $a_{0}, a_{1}, \ldots, a_{n}$ are
      real constant.
    \par One important property of algebraic polynomial is that a polynomial $P$
      can get as ``close" to the given function $f$ as possible. This property
      is expressed in the Weierstrass Approximation Theorem.
    \par Another reason for using polynomials is their simplicity: the
      derivatives and integrals of polynomials are also polynomials and are easy
      to find.

  \hii{Weierstrass Approximation Theorem}
    \par \tb{Theorem}: Suppose that $f$ is defined and continuous on $[a, b]$.
      For each and every $\epsilon > 0$, there exists a polynomial $P(x)$ so
      that:

    \begin{align*}
      \abs{f(x) - P(x)} < \epsilon, \quad \forall x \in [a, b]
    \end{align*}


  \hii{Lagrange Interpolating Polynomials}
    \hiii{Example: Linear Lagrange Interpolation}
      \par \tb{Example}: Given two distinct points $(x_{0}, y_{0})$ and
        $(x_{1}, y_{1})$. Determine a function $f$ for which $f(x_{0}) = y_{0}$
        and $f(x_{1}) = y_{1}$.

      \par Define the following functions:

        \begin{align*}
          p_1^{(0)}(x) = \frac{x - x_1}{x_0 - x_1}
          \qquad \mbox{and} \qquad
          p_1^{(1)}(x) = \frac{x - x_0}{x_1 - x_0}
        \end{align*}

      \par The linear \tb{Lagrange interpolation polynomial} through
        $(x_0, y_0)$ and $(x_1, y_1)$ is:

        \begin{align*}
          \mathcal{L}_1(x) = y_0 p_1^{(0)}(x) + y_1 p_1^{(1)}(x)
        \end{align*}

      \par Note that:
        \begin{align*}
          \begin{cases}
            p_1^{(0)}(x_0) = 1 \qquad \mbox{and} \qquad p_1^{(1)}(x_0) = 0 \\
            p_1^{(0)}(x_1) = 0 \qquad \mbox{and} \qquad p_1^{(1)}(x_1) = 1
          \end{cases} \\
          \ra
          \begin{cases}
            \mathcal{L}_1(x_0) = y_0 p_1^{(0)}(x_0) + y_1 p_1^{(1)}(x_0) = y_0 \\
            \mathcal{L}_1(x_1) = y_0 p_1^{(0)}(x_1) + y_1 p_1^{(1)}(x_1) = y_1
          \end{cases}
        \end{align*}

        \par $\mathcal{L}_1(x)$ is called a linear Lagrange polynomial of the
          function $f(x)$.

    \hiii{nth Lagrange Interpolating Polynomial}
      \par \tb{Theorem}: If $x_0, x_1, \ldots, x_n$ are $n + 1$ distinct numbers
        and $y_0, y_1, \ldots, y_n$ are given, then a unique polynomial
        $\mathcal{L}_n$ exists that satisfies:
        \begin{itemize}
          \item $deg(\mathcal{L}_n) \leq n$
          \item $\mathcal{L}_n(x_k) = y_k \quad \forall k = 0, 1, \ldots, n$
        \end{itemize}
      \par The first step of finding $\mathcal{L}_n$ is to find the basic
        polynomials $p_n^{(k)}$ with $k = 0, 1, \ldots, n$ such that:
        \begin{itemize}
          \item $deg(p_n^{(k)}) \leq n$
          \item
            \begin{align*}
              p_n^{(k)}(x_i) =
              \begin{cases}
                1 \mbox{ if } i = k \\
                0 \mbox{ if } i \neq k
              \end{cases}
              \ra p_n^{(k)}(x)
                = \frac{(x - x_0) \ldots (x - x_{k - 1})(x - x_{k + 1}) \ldots (x - x_n)}
                       {(x_k - x_0) \ldots (x_k - x_{k - 1})(x_k - x_{k + 1}) \ldots (x_k - x_n)}
            \end{align*}
        \end{itemize}
      \par Then, we have:
        \begin{align*}
          \mathcal{L}_n(x) &= y_0 p_n^{(0)}(x)
                          + y_1 p_n^{(1)}(x) + \ldots
                          + y_n p_n^{(n)}(x) \\
                          &= \sum\limits_{k = 0}^n y_k p_n^{(k)} (x)
        \end{align*}

\hi{Newton's Polynomial}
  \hii{Initial Form}
    \par The Newton's Interpolating Polynomial $P_n(x)$ which agrees with the
      function $f$ at distinct numbers $x_0, x_1, \ldots, x_n$ has the form:
      \begin{align*}
        P_n(x) = a_0 + a_1(x - x_0) + a_2(x - x0)(x - x1) + \ldots
              + a_n(x - x_0)(x - x_1) \ldots (x - x_n)
      \end{align*}
  \hii{Divided Difference}
    \par \tb{Definition}: Divided Difference has a recursive definition:
      \begin{itemize}
        \item The zeroth divided difference of the function $f$ with respect
          to $x_i$, denoted $f[x_i]$, is the value of $f$ at $x_i$:
          \begin{align*}
            f[x_i] = f(x_i)
          \end{align*}
        \item The first divided difference of $f$ with respect to $x_i$ and
          $x_{i + 1}$, denoted $f[x_i, x_{i + 1}]$, is defined as:
          \begin{align*}
            f[x_i, x_{i + 1}] = \frac{f[x_{i + 1}] - f[x_i]}{x_{i + 1} - x_i}
          \end{align*}
        \item The second divided difference is defined as:
          \begin{align*}
            f[x_i, x_{i + 1}, x_{i + 2}]
              = \frac{f[x_{i + 1}, x_{i + 2}] - f[x_i, x_{i + 1}]}{x_{i + 2} - x_i}
          \end{align*}
        \item In general, the kth divided difference is:
          \begin{align*}
            f[x_i, x_{i + 1}, \ldots, x_{i + k}]
              = \frac{f[x_{i + 1}, x_{i + 2}, \ldots, x_{i + k}]
                    - f[x_i, x_{i + 1}, \ldots, x_{i + k - 1}]}
                     {x_{i + 2} - x_i}
          \end{align*}
      \end{itemize}

  \hii{Forms of Newton's Polynomial}
    \begin{itemize}
      \item Forward form:
        \begin{align*}
          \mathcal{N}_n^{(1)} = f[x_0] + f[x_0, x_1](x - x_0)
                            + f[x_0, x_1, x_2](x - x_0)(x - x_1)
                            + \ldots
                            + f[x_0, x_1, \ldots, x_n](x - x_0)(x - x_1)\ldots(x - x_{n - 1})
        \end{align*}
      \item Backward form:
        \begin{align*}
          \mathcal{N}_n^{(2)} = f[x_n] + f[x_{n - 1}, x_n](x - x_n)
                            + f[x_{n - 2}, x_{n - 1}, x_n](x - x_{n - 1})(x - x_n)
                            + \ldots
                            + f[x_0, x_1, \ldots, x_n](x - x_1)(x - x_2)\ldots(x - x_n)
        \end{align*}
    \end{itemize}

  \hii{Special Case: Uniform-Distanced Points}
    \hiii{Uniform-Distanced Points}
      \par We consider the case where
        $x_{k + 1} - x_{k} = h = const \quad \forall k = 0, 1, \dots, n - 1$.

    \hiii{Finite Differences}
      \par Finite Differences also have a recursive definition:
      \begin{itemize}
        \item $\Delta y_k = y_{k + 1} - y_k \quad \forall k = 0, 1, \ldots, n - 1$
          is called \tb{finite differences} of first order at $x_k$.
        \item $\Delta^2 y_k = \Delta y_{k + 1} - \Delta y_k \quad \forall k = 0, 1, \ldots, n - 2$
          is called \tb{finite differences} of second order at $x_k$.
        \item $\Delta^p y_k
          = \Delta^{p - 1} y_{k + 1} - \Delta^{p - 1} y_k \quad \forall k = 0, 1, \ldots, n - 2$
          is called \tb{finite differences} of pth order at $x_k$.
      \end{itemize}

      \par The relationship between Finite Differences and Divided Differences:
        \begin{align*}
          f[x_k, x_{k + 1}, \ldots, x_{k + p}] = \frac{\Delta^p y_k}{p!h^p}
        \end{align*}

    \hiii{Newton's Polynomial Forms with Finite Differences}
      \begin{itemize}
        \item Forward form:
          \begin{align*}
            \begin{aligned}
              \mathcal{N}_n^{(1)} = y_0 + \frac{\Delta y_0}{1!}q
                                  + \frac{\Delta^2 y_0}{2!}q(q - 1)
                                  + \ldots
                                  + \frac{\Delta^n y_0}{2!}q(q - 1)\ldots(q - n + 1) \\
              \mbox{ where } q = \frac{x - x_0}{h}
            \end{aligned}
          \end{align*}
        \item Backward form:
          \begin{align*}
            \begin{aligned}
              \mathcal{N}_n^{(2)} = y_n + \frac{\Delta y_n - 1}{1!}p
                                  + \frac{\Delta^2 y_0}{2!}p(p + 1)
                                  + \ldots
                                  + \frac{\Delta^n y_0}{2!}p(p - 1)\ldots(p + n - 1) \\
              \mbox{ where } p = \frac{x - x_n}{h}
            \end{aligned}
          \end{align*}
      \end{itemize}

