\chapter{The Nature and Propagation of Light}


\hi{The Nature of Light}
  \hii{Waves, Wave Fronts, and Rays}
    \par A \tb{wave front} is the locus (surface) of all adjacent points at
      which the phase of vibration of a phyical quantity associated with the wave
      is the same.


\hi{Reflection and Refraction}
  \hii{Introduction}
    \par Reflection at a definite angle from a very smooth surface is called
      \tb{specular reflection}; scattered reflection from a rough surface is
      called \tb{diffuse reflection}.
    \par \tb{Index of refraction}:
    \begin{eqbox}
      n = \frac{c}{v}
    \end{eqbox}
  \hii{Laws}
    \begin{itemize}
      \item The incident, reflected, and refracted rays and the normal to the
        surface all lie in the same plane.
      \item Law of reflection:
        \begin{align*}
          \theta_r = \theta_a
        \end{align*}
        $\theta_r$: angle of reflection; $\theta_a$: angle of reflection
      \item Law of refraction:
        \begin{align*}
          \frac{\sin(\theta_a)}{\sin(\theta_b)} = \frac{n_b}{n_a}
        \end{align*}
        \begin{align*}
          n_a \sin(\theta_a) = n_b \sin(\theta_b)
        \end{align*}
    \end{itemize}
  \hii{Index of Refraction and the Wave aspect}
    \par $f$ of light does not change when passing from one material to
      another.
    \par $v < c$ for all materials other than vacuum.
    \begin{align*}
      \lambda = \frac{\lambda_0}{n}
    \end{align*}
      $\lambda_0$: wavelength in vacuum; $\lambda$: wavelength in material
        with index of refraction $n$.


\hi{Total Internal Reflection}
  \par Total Internal Reflection may only occurs when rays of light strike
    the surface of a second material $b$ where $n_a > n_b$.
  \par The angle of incidence for which the refracted ray emerges tangent to
    the surface is called the \tb{critical angle} ($\theta_crit$).
    \begin{align*}
      \sin(\theta_crit) = \frac{n_b}{n_a}
    \end{align*}


\hi{Dispersion}
  \par The dependence of wave speed and index of refraction on wavelength
    is called \tb{dispersion}.
    \begin{align*}
      v, n ~ \lambda
    \end{align*}


\hi{Polarization}
  \BOOKSECTION{33.5}
    \par A wave has only $y$-displacements is said to be \tb{linearly polarized}
      in the $y$-direction.


\hi{Huygens' Principle}
  \BOOKSECTION{33.7}
  \hii{Definition}
    \par \tb{Huygen's Principle}: Every point of a wave front may be considered the source
      of secondary wavelets that spread out in all directions with a speed equals to the speed of
      propagation of the wave.
    \img[width=12cm]{chap33/huygen.jpg}

  \hii{Reflection}
    \img[width=8cm]{chap33/huygen-reflection.jpg}
