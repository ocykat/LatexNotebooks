\chapter{Periodic Motions}


\hi{Describing Oscillation}
  \hii{Terminology}
    \begin{itemize}
      \item \tb{equilibrium}
      \item \tb{restoring force}
      \item \tb{cycle}
      \item \tb{amplitude ($A$)}: the maximum magnitude of displacement from
        equilibrium.
        \begin{eqbox}
          A = \max(\abs{x})
        \end{eqbox}
      \item \tb{period ($T$)}: the time for one cycle.
        \par unit: $[s]$
      \item \tb{frequency ($f$)}: the number of cycles in a unit of time.
        \par unit: $[Hz]$
        \begin{eqbox}
          f = \frac{1}{T}
        \end{eqbox}
      \item \tb{angular frequency}:
        \begin{eqbox}
          \omega = 2\pi f = \frac{2\pi}{T}
        \end{eqbox}
    \end{itemize}


\hi{Spring Pendulum}
  \hii{Spring Pendulum}
    \par In a ideal spring pendulum system, the restoring force $F_{x}$ is directly
      propotional to the displacement (from equilibrium).
    \begin{align*}
      F_{x} = -kx
    \end{align*}
    \par When the restoring force is directly propotional to the displacement,
      the oscillation is called \tb{simple harmonic motion} (SHM).
    \begin{flalign*}
      & a_{x} = \frac{d^{2}x}{dt^{2}} = -\frac{k}{m}x
    \end{flalign*}
    \par We can show that the movement of a body under the effect of a spring
      pendulum is exactly the same with the movement of the $x$-component of
      a rotational motion which $r = A/2$ and
      $\omega_{pendulum} = \omega_{rotation}$.
    \par Therefore:
      \begin{align*}
        x = A \cos(\omega t + \varphi)
      \end{align*}
    \par Angular frequency of spring pendulum:
    \begin{eqbox}
      \omega = \sqrt{\frac{k}{m}}
    \end{eqbox}

  \hii{Displacement, Velocity and Acceleration}
    \hiii{Displacement}
      \begin{eqbox}
        x = A \cos(\omega t + \varphi)
      \end{eqbox}
      where
      \begin{itemize}
        \item $\omega$: angular frequency
        \item $\varphi$: phase angle
      \end{itemize}
    \hiii{Velocity}
      \begin{eqbox}
        v = -\omega A \sin(\omega t + \varphi) = -\omega A \cos(\omega t + \varphi + \frac{\pi}{2})
      \end{eqbox}
    \hiii{Acceleration}
      \begin{eqbox}
        a = -\omega^{2} A \cos(\omega t + \varphi) = \omega^{2} A \cos(\omega t + \varphi + \pi)
      \end{eqbox}

  \hii{Amplitude and Period}
    \par The amplitude of the SHM has no effect on the period/frequency of oscillation.
