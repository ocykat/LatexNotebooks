\chapter{Waves}

\hi{Introduction}
  \hii{Definition}
    \par A \tb{mechanical wave} is a disturbance that travels through some
      material or substance called \tb{medium}.
  \hii{Types of wave}
    \begin{itemize}
      \item A \tb{traverse wave} is a wave in which the displacements of the
        medium are perpendicular (or traverse) to the direction of travel of
        the wave.
      \item A \tb{longitudinal wave} is a wave in which the particles of the
        medium are back and forth along the same direction that the wave
        travels.
    \end{itemize}

\hi{Periodic Waves}
  \hii{Definition}
    \par A \tb{periodic wave} is a wave in which every particle undergo
      periodic motion.
  \hii{Wavelength}
    \par Terminology:
    \begin{itemize}
      \item crest
      \item trough
    \end{itemize}
    \par The \tb{wavelength} is the distance between two nearest crests,
      or two nearest trough, or any two nearest points oscillating with the
      same phase (or oscillating in phase with each other).
    \begin{eqbox}
      v = \lambda f
    \end{eqbox}
    where:
    \begin{itemize}
      \item $v$: wave speed/speed of propagation
      \item $\lambda$: wavelength
      \item $f$: frequency
    \end{itemize}
  \hii{Other quantities}
    \par The period $T$ of a wave is the time interval for the wave to travel
      a distance equals to one wavelength.
    \par Frequency
    \par Amplitude

\hi{Wave function}
  \hii{Displacement at arbitrary position of sinusoidal tranverse wave}
    \par Displacement of the left end of the string $x = 0$, where the wave
      originates, is given by:
      \begin{flalign*}
        & y(x = 0, t) = A\cos(\omega t) = A\cos(2\pi ft) &&
      \end{flalign*}

    \par The motion of arbitrary point $x$ at time $t$ is the same as the motion
      of point $x = 0$ at the earlier time $t - x/v$. Hence:
      \begin{flalign*}
        &
          y(x, t) = A\cos\bigg[\omega\bigg(t - \frac{x}{v} \bigg) \bigg] 
                  = A\cos\bigg[\omega\bigg(\frac{x}{v} - t \bigg) \bigg] 
        && \\
        &
          \ra y(x, t) = A\cos\bigg[2 \pi \bigg(\frac{x}{\lambda} - \frac{t}{T} \bigg) \bigg] 
        &&
      \end{flalign*}
    
    \par Define $k$ as the \tb{wave number}:
      \begin{align*}
        k = \frac{2 \pi}{\lambda}
      \end{align*}
      \begin{flalign*}
        &
          \ra y(x, t) = A\cos(\omega t - kx)
        &&
      \end{flalign*}

  \hii{Wave Equation}
    \begin{flalign*}
      &
        y(x, t) = A\cos(\omega t - kx)
      && \\
      & \ra
        \begin{cases}
            \pd{y}{t} = -\omega A\sin(\omega t - kx) \\
            \pd{y}{x} = kA\sin(\omega t - kx)
          \end{cases}
        && \\
      & \ra
        \begin{cases}
          \pdds{y}{t} = -\omega^2  A\cos(\omega t - kx) \\
          \pdds{y}{x} = -k^2 A\cos(\omega t - kx)
        \end{cases}
      && \\
      & \ra
        \frac{pddsi{y}{t}}{pddsi{y}{x}} = \frac{\omega^2}{k^2} = v^2
      && \\
      & \ra
        \frac{\partial^2 y(x, t)}{\partial x^2}
          = \frac{1}{v^2} \frac{\partial^2 y(x, t)}{\partial t^2}
      &&
    \end{flalign*}

\hi{Speed of Traverse Wave}
  \begin{eqbox}
    v = \sqrt{\frac{F}{\mu}}
  \end{eqbox}
  \begin{itemize}
    \item $v$: speed of traverse wave
    \item $F$: force
    \item $\mu$: mass per unit length
  \end{itemize}
