\chapter{Sources of magnetic field}

\hi{Magnetic Field of a Moving Charge}
  \par Terminology:
    \begin{itemize}
      \item source point: the location of the moving charge at an instance (S)
      \item field point: the location where we want to find the magnetic field (P)
    \end{itemize}

  \par Denote the source point as $S$ and the field point as $P$.
  \par Magnetic field vector $\vec{B}$ of a moving charge:
    \begin{itemize}
      \item Vector form:
        \par Direction: perpendicular to the plane containing $SP$ and $\vec{v}$
      \begin{eqbox}
        \vec{B} = \frac{\mu_0}{4 \pi} \frac{\abs{q} \vec{v}\hat{r} \sin{\varphi}}{r^2}
      \end{eqbox}
      \item Magnitude:
      \begin{eqbox}
        B = \frac{\mu_0}{4 \pi} \frac{\abs{q} v \sin{\varphi}}{r^2}
      \end{eqbox}
    \end{itemize}


\hi{Magnetic Field of a Current Element}
  \par The total magnetic field caused by several moving charges is the vector
    sum of the fields caused by the individual charges.
  \par Magnetic field vector $\vec{B}$ of a current element:
    \begin{itemize}
      \item Vector form:
        \par Direction: perpendicular to the plane containing $SP$ and $\vec{v}$
      \begin{eqbox}
        \vec{B} = \frac{\mu_0}{4 \pi} \INT \frac{I \vec{dl} \hat{r}}{r^2}
      \end{eqbox}
      \item Magnitude:
      \begin{eqbox}
        B = \frac{\mu_0}{4 \pi} \frac{\abs{q} v \sin{\varphi}}{r^2}
      \end{eqbox}
    \end{itemize}

\hi{Ampere's Law}


