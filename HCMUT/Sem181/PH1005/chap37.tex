\chapter{Relativity}

\hi{Invariance of Physical Laws}
  \hii{Einstein's First Postulate}
    \par \tb{Einstein's First Postulate}: The laws of physics are the same
      in every inertial frame of reference.
  \hii{Einstein's Second Postulate}
    \par \tb{Einstein's Second Postulate}: The speed of light in vacuum is
      the same in all inertial frames of reference and is independent of the
      motion of the source.
  \hii{The Ultimate Speed Limit}
    \par \tb{Result of the Second Postulate}: It is impossible for an
      inertial observer to travel at $c$, the speed of light in vacuum.
    \begin{smfont}
      \par Proof: Suppose a spacecraft $S'$ is moving at the speed of light
        relative to an observer $S$ on the earth, so that $v_{S'/S} = c$.
        If the spacecraft turns on its headlight, the second postulate now
        asserts that the earth observer $S$ measures the headlight beam
        to be also moving at $c$. Thus this observer measures that the
        headlight beam and the spacecraft move together and are always at
        the same point in space. However, Einstein's Second Postulate also
        asserts that the headlight beam moves at a speed $c$ relative to
        the spacecraft, so they cannot be at the same point in space. This
        contradictory result can be avoided only if it is impossible for
        an inertial observer, such as a passenger on the spacecraft, to
        move at $c$.
    \end{smfont}