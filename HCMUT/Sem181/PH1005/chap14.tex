\chapter{Periodic Motions}


\hi{Describing Oscillation}
  \hii{Terminology}
    \begin{itemize}
      \item \tb{equilibrium}
      \item \tb{restoring force}
      \item \tb{cycle}
      \item \tb{amplitude ($A$)}: the maximum magnitude of displacement from
        equilibrium.
        \begin{eqbox}
          A = \max(\abs{x})
        \end{eqbox}
      \item \tb{period ($T$)}: the time for one cycle.
        \par unit: $[s]$
      \item \tb{frequency ($f$)}: the number of cycles in a unit of time.
        \par unit: $[Hz]$
        \begin{eqbox}
          f = \frac{1}{T}
        \end{eqbox}
      \item \tb{angular frequency}:
        \begin{eqbox}
          \omega = 2\pi f = \frac{2\pi}{T}
        \end{eqbox}
    \end{itemize}


\hi{Spring Pendulum}
  \hii{Spring Pendulum}
    \par In a ideal spring pendulum system, the restoring force $F_{x}$ is directly
      propotional to the displacement (from equilibrium).
    \begin{align*}
      F_{x} = -kx
    \end{align*}
    \par When the restoring force is directly propotional to the displacement,
      the oscillation is called \tb{simple harmonic motion} (SHM).
    \begin{flalign*}
      & a_{x} = \frac{d^{2}x}{dt^{2}} = -\frac{k}{m}x
    \end{flalign*}
    \par We can show that the movement of a body under the effect of a spring
      pendulum is exactly the same with the movement of the $x$-component of
      a rotational motion which $r = A/2$ and
      $\omega_{pendulum} = \omega_{rotation}$.
    \par Therefore:
      \begin{align*}
        x = A \cos(\omega t + \varphi)
      \end{align*}
    \par Angular frequency of spring pendulum:
    \begin{eqbox}
      \omega = \sqrt{\frac{k}{m}}
    \end{eqbox}

  \hii{Displacement, Velocity and Acceleration}
    \hiii{Displacement}
      \begin{eqbox}
        x = A \cos(\omega t + \varphi)
      \end{eqbox}
      where
      \begin{itemize}
        \item $\omega$: angular frequency
        \item $\varphi$: phase angle
      \end{itemize}
    \hiii{Velocity}
      \begin{eqbox}
        v = -\omega A \sin(\omega t + \varphi) = -\omega A \cos(\omega t + \varphi + \frac{\pi}{2})
      \end{eqbox}
    \hiii{Acceleration}
      \begin{eqbox}
        a = -\omega^{2} A \cos(\omega t + \varphi) = \omega^{2} A \cos(\omega t + \varphi + \pi)
      \end{eqbox}

  \hii{Amplitude and Period}
    \par The amplitude of the SHM has no effect on the period/frequency of oscillation.


\hi{Energy in Simple Harmonic Motion}
  \begin{eqbox}
    E = K + U = \frac{1}{2} mv^2 + \frac{1}{2} kx^2 = \text{constant}
  \end{eqbox}


\hi{Simple Pendulum}
  \hii{Definition}
    \par A \tb{simple pendulum} is an idealized model consisting of a point
      mass suspended by a massless, unstreachable string.

  \hii{SHM proof}
    \par Define $F_T$ as the tangential component of the net force exerted
      on the mass.
      \begin{align*}
        F_T = -mgsin(\theta)
      \end{align*}
      where $\theta$ is the angle between the vertical line and the string.
    \par When $\theta$ is small, $\sin(\theta) \approx \theta \approx tan(\theta)$.
      \begin{align*}
        F_T \approx -mg\theta \approx -mg\tan(\theta) = -mg \frac{x}{L}
      \end{align*}
      where $x$ is the displacement relative to the equilibrium and
        $L$ is the length of the string.
    \par Let $k = -\frac{mg}{L}$. We return to the form of SHM:
      \begin{align*}
        F_T = -kx
      \end{align*}

  \hii{Angular frequency}
    \begin{align*}
      \omega = \sqrt{\frac{k}{m}} = \sqrt{\frac{mg/l}{m}} = \sqrt{\frac{g}{l}}
    \end{align*}
    \begin{align*}
      f = \frac{\omega}{2\pi} = \frac{1}{2\pi} \sqrt{\frac{g}{l}}
    \end{align*}
    \begin{align*}
      T = \frac{2\pi}{\omega} = \frac{1}{f} = 2\pi\sqrt{\frac{l}{g}}
    \end{align*}


\hi{Physical Pendulum}
  \hii{Physical Pendulum and SHM}
    \BOOKSECTION{14.6}
    \par Define $z$ as the axis of rotation of the pendulum.
      \begin{align*}
        \tau_z = Fd = F_{W}d \sin(\theta) = -mgd\sin(\theta)
      \end{align*}
      where
      \begin{itemize}
        \item $d$: lever arm
      \end{itemize}
    \par In the case of the physical pendulum, the lever arm is the distance between
      the pivot and the CM.
    \par The minus sign indicates that the restoring force is clockwise when the
      displacement is counterclockwise, and vice versa.
    \par Consider the motion to be \ti{approximately} SHM.
      \begin{align*}
        \tau_z = F_{W}d = -mgd\theta
      \end{align*}
    \par The relationship between torque and moment of inertia:
      \begin{align*}
        \tau_z = I\alpha_z
      \end{align*}
      where $\alpha_z$ is the angular acceleration.
      \begin{flalign*}
        & \ra I\alpha_z = -mgd\theta && \\
        & \ra I \ddif{\theta}{t} = -mgd\theta && \\
        & \ra \ddif{\theta}{t} = -\frac{mgd\theta}{I} && \\
        & \ra \ddif{\theta}{t} \cdot d = -\frac{mgd}{I} (\theta \cdot d) && \\
        & \ra \ddif{x}{t} = -\frac{mgd}{I} x && \\
        & \ra m \ddif{x}{t} = -\frac{m^2gd}{I} x && \\
        & \ra F_{restore} = -\frac{m^2gd}{I} x && \\
      \end{flalign*}
    \par If we define $k = m^2gd/I$, we return to the form of SHM.
  \hii{Angular frequency}
    \begin{eqbox}
      \omega = \sqrt{\frac{k}{m}} = \sqrt{\frac{mgd}{I}}
    \end{eqbox}
