\chapter{The Processor}

\hi{Introduction}
  \hii{Terminology}
    \begin{itemize}
      \item \tb{Register file}: an array of processor registers
    \end{itemize}

  \hii{Basic MIPS Implementation}
    \par In this part, we consider the basic subset of MIPS instructions,
    including:
    \begin{itemize}
      \item \tb{Memory reference} instructions: \lstinline{lw} and
        \lstinline{sw}.
      \item \tb{Arithmetic-Logical} instructions: \lstinline{add},
        \lstinline{sub}, \lstinline{AND}, \lstinline{OR} and \lstinline{slt}.
      \item \tb{Branch} instructions: \lstinline{beq} and \lstinline{j}.
    \end{itemize}
    \par For each instructions, the first two steps are identical:
    \begin{enumerate}[1.]
      \item Send the PC to the memory that contains the code and fetch the
        instruction from that memory.
      \item Read 1 or 2 registers accoding to the instruction.
    \end{enumerate}
    \par After the first two steps, the actions required to complete the
      instruction depend on the instruction class.

  \hii{Multiplexor}

\hi{Logic Design Conventions}
  \par \ti{This part deals largely with the topic of the previous course,
    including}:
    \begin{itemize}
      \item Combinational circuit vs. Sequential circuit
      \item Clock
    \end{itemize}

\hi{Building Datapath}
