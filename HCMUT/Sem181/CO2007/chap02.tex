\chapter{Instructions: Language of the Computer}

\hi{Operations of the Computer Hardware}
  \BOOKSECTION{2.2}
  \par \tb{Instruction types}:
  \begin{itemize}
    \item Arithmetic
    \item Data transfer
    \item Logical
    \item Conditional branch
    \item Unconditional jump
  \end{itemize}

\hi{Operands}
  \par The operands of arithmetic instructions must be from a limited number
    of special locations built directly in hardware called \ti{registers}.
  \par The three operands of MIPS arithmetic instructions must each be chosen
    from one of the 32 32-bit registers.

  \hii{Registers}
    \par In the MIPS architecture:
    \begin{itemize}
      \item the size of a register is \tb{32 bits}.
      \item a group of 32 bits is called \tb{word}.
      \item there are 32 different registers.
      \item each word must start at addresses that are multiples of 4 (known
        as alignment restriction).
    \end{itemize}

  \hii{Arithmetic operands}
    \par One arithmetic operand is followed by 3 registers.
    \begin{lstlisting}
# $s1 contains value of a
# $s2 contains value of b
# $t0 contains value of c
add $s1, $s2, $t0 # a = b + c
    \end{lstlisting}

  \hii{Memory operands}
    \par \tb{Data transfer instructions} are instructions that transfer data
      between memory and registers.
    \par The data transfer instruction that copies data from memory to a
      register is called \tb{load}.
    \par The data transfer instruction that copies data from a register back
      to memory is called \tb{store}.

    \begin{lstlisting}
# $s3 keeps the address of array a
lw $t0, 8($s3) # reg $t0 gets the value of a[8]
    \end{lstlisting}

    \par In the example:
      \begin{itemize}
        \item The constant 8 is called the \tb{offset}.
        \item The register added to form the address \reg{s3} is called
          \tb{base register}.
      \end{itemize}

    \par \tb{Big-endian and Little-endian}: There are two types of architecture:
      \begin{itemize}
        \item Using the address of the leftmost byte, or MSb, of the word address.
          This type of architecture is called \tb{big-endian}.
        \item Using the address of the rightmost byte, or LSb, byte of the word
          adress. This type of architecture is called \tb{little-endian}.
      \end{itemize}

  \hii{Constant/Immediate Operands}
    \par There is a special register: \reg{zero}, which stores the number 0 and
      cannot be overwritten.


\hi{Representing Signed and Unsigned Numbers}
  \par \ti{Two's complement}


\hi{Representing Instructions in the Computer}
  \par Each MIPS instruction are exactly 32 bits long.
  \par A MIPS instruction when converted into 32-bit binary code consists of
    different parts called fields.

  \hii{R-format instructions}
    \par \tb{R-format} (R for register) is the kind of format used for
      arithmetic instructions and bit-shifting instructions.
    \par The fields of R-format are:
    \begin{itemize}
      \item \tb{op}: (6 bits) Opcode, or basic operation of the instruction
      \item \tb{rs}: (5-bits) The first source register
      \item \tb{rt}: (5-bits) The second source register
      \item \tb{rd}: (5-bits) The destination register
      \item \tb{shamt}: (5-bits) Shift amount
      \item \tb{funct}: (6-bits) Function code, which selects the specific
        variant of the operation in the \tb{op} field.
    \end{itemize}

  \hii{I-format instructions}
    \par \tb{I-format} (I for immediate) is the kind of format used for
      immediate and data transfer instructions.
    \par The fields of I-format are:
    \begin{itemize}
      \item \tb{op}: (6 bits) Opcode, or basic operation of the instruction
      \item \tb{rs}: (5-bits) The first source register
      \item \tb{rt}: (5-bits) The second source register
      \item \tb{constant or address}: (16-bit) Constant or address
    \end{itemize}


\hi{Logical operations}
  \begin{itemize}
    \item Shift left: \code{sll}
    \item Shift right: \code{srl}
    \item Bit-by-bit AND: \code{and}, \code{andi}
    \item Bit-by-bit OR: \code{or}, \code{ori}
    \item Bit-by-bit NOT: \code{nor}
  \end{itemize}
  \par Note: Bit-by-bit NOT is replaced by NOR with 0.
\begin{lstlisting}
nor $t0, $t1, $zero
\end{lstlisting}


\hi{Instructions for Making Decisions}
  \begin{itemize}
    \item \code{beq}, \code{bne}
    \item \code{j}
    \item \code{slt, slti}
  \end{itemize}

\hi{Supporting Procedures in Computer Hardware}
  \hii{Procedure Execution Steps}
    \par Steps in the execution of a procedure:
    \begin{enumerate}
      \item Put parameters in a place where the procedure can access them
      \item Transfer control to the procedure
      \item Acquire the storage resources needed for the procedure
      \item Perform the task
      \item Put the result value in a place where the calling program can access
        it
      \item Return control to the point of origin, since a procedure can be
        called from several points in a program
    \end{enumerate}

  \hii{Registers}
    \begin{itemize}
      \item \reg{a0} - \reg{a3}: four argument registers in which to pass
        parameters
      \item \reg{v0} - \reg{v1}: two value registers in which to return values
      \item \reg{ra}: one return address register to return to the point of
        origin
    \end{itemize}

  \hii{Instructions}
    \begin{itemize}
      \item \code{jal}: stands for jump-and-link. This instruction jumps to the
        address of the procedure and simultaneously saves the address of the
        \code{jal} instruction in the register \reg{ra}.
      \item \code{jr}: stands for jump register. This instruction jumps to the
        address which is currently saved in the \reg{ra} register. It is often
        used to return to the previous address after a procedure finishes its
        execution.
    \end{itemize}

  \hii{Program Counter}
    \par The \tb{program counter}, abbreviated PC, is a register holding the
      address of the current instruction being executed.
    \par The \lstinline{jal} instruction actually saves PC + 4 in the register
      \reg{ra} to link to the folloing instruction to set up the procedure
      return.

  \hii{Using More Registers}
    \hiii{Spilling}
      \par The advantage of using registers is speed. However, a computer has
        limited number of registers. Therefore, not all variables can be
        assigned to registers.
      \par A \tb{spilled variable} is a variable in main memory rather than in
        a register.
      \par The operation of moving a variable from a register to memory is
        called \tb{spilling}, while the reverse operation of moving a variable
        from memory to a register is called \tb{filling}.

    \hiii{Situation}
      \par Suppose a compiler needs more registers for a procedure than the
        4 argument and 2 return value registers. We cannot simply use other
        registers, because after a procedure, we must restore the value of all
        other registers to their state before the procedure is called.
      \par In this situation, we need to move spill registers to memory.

    \hiii{Stack}
      \par The data structure for spilling registers is a \tb{stack}.
      \par A stack needs a pointer to the most recently allocated address in
        the stack to show where the next procedure should place the registers
        to be spilled or where old register values are found. For each register
        that is saved or restored, The \tb{stack pointer} is adjusted by 1 word.
      \par The stack pointer of MIPS is stored in register \reg{sp} - number 29.
      \par A stack grows from higher addresses to lower addresses.
      \par Example:
      \begin{itemize}
        \item C code:
          \begin{lstlisting}
int leaf_example (int g, int h, int i, int j)
{
  int f;
  f = (g + h) – (i + j);
  return f;
}
          \end{lstlisting}
          \item MIPS code:
            \begin{lstlisting}
leaf_example:
  addi $sp, $sp, -12 # adjust stack to make room for 3 items
  sw $t1, 8($sp) # save register $t1 for use afterwards
  sw $t0, 4($sp) # save register $t0 for use afterwards
  sw $s0, 0($sp) # save register $s0 for use afterwards

  add $t0,$a0,$a1 # register $t0 contains g + h
  add $t1,$a2,$a3 # register $t1 contains i + j
  sub $s0,$t0,$t1 # f = $t0 – $t1, which is (g + h) – (i + j)

  add $v0,$s0,$zero # returns f ($v0 = $s0 + 0)

  lw $s0, 0($sp) # restore register $s0 for caller
  lw $t0, 4($sp) # restore register $t0 for caller
  lw $t1, 8($sp) # restore register $t1 for caller
  addi $sp,$sp,12 # adjust stack to delete 3 items
            \end{lstlisting}
      \end{itemize}

    \hiii{Register Use Convention}
      \par To reduce register spilling, we follow the following convention:
      \begin{itemize}
        \item \reg{t0} - \reg{t9}: temporary register that are \tb{not}
          preserved by the callee (procedure) on a procedure call
        \item \reg{s0} - \reg{s7}: saved registers that must be preserved on
          a procedure call (if used, the callee saves and restores them by
          spilling)
      \end{itemize}

    \hiii{Nested Procedures}
      \par Procedures that do not call others are called \tb{leaf procedures}.
      \par Procedures that call others are called \tb{non-leaf procedures}.
        For non-leaf procedures, use stack!

\hi{MIPS Addressing for 32-bit Immediate and Addresses}
  \hii{32-bit Immediate Operands}
    \par The R-format instructions only preserve 16 bits for constants.
    \par To work with 32-bit immediate, one can use the \lstinline{lui}
      instruction, which stands for \ti{load upper immediate} to load the
      upper 16 bits of the immediate into the register.
    \par Example: Load the constant $4,000,000$ into the register \reg{s0}.
    \begin{lstlisting}
  # 4,000,000 decimal = 0000 0000 0011 1101 0000 1001 0000 0000 binary
  lui $s0, 61        # 61 decimal = 0000 0000 0011 1101 binary = upper 16 bits
  ori $s0, $s0, 2304 # 61 decimal = 0000 1001 0000 0000 binary = lower 16 bits
    \end{lstlisting}

  \hi{Addressing in Branch and Jumps}
    \hii{J-format}
      \par The \lstinline{j} instruction is a J-format instruction:
      \par The fields of J-format are:
      \begin{itemize}
        \item \tb{op}: (6 bits) Opcode
        \item \tb{addr}: (26 bits) Address to jump to
      \end{itemize}
      \par Unlike the \lstinline{j} instruction, the conditional branch
        instruction must specify two operands in addition to the branch address.
    
    \hii{Dealing with addresses}
      \par If the addresses are limited to $2^{16}$, then the program will be
        far too small for realistic use.
      \par To deal with that problem, relative addressing is used instead of
        absolute addressing:
        \begin{align*}
          \mbox{ Program Counter } = \mbox{ Register } + \mbox { Branch Address }
        \end{align*}
      \par This form of branch addressing is called \tb{PC-relative
        addressing}.
      \par Usually, conditional branch instructions tend to branch to a nearby
        instructions. On the other hand, unconditional jump instructions do
        not. Normally, they would use other forms of addressing. The MIPS
        architecture offers long addresses for procedure calls by using the
        J-format for both jump and jump-and-link instructions.
