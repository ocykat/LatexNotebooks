\chapter{Computer Abstractions and Technology}

\hi{Terminology}
  \BOOKSECTION{1.1 - 1.5}
  \begin{itemize}
    \item \tb{integrated circuit (chip)}: is a device consists of transistors.  
    \item \tb{central processing unit (CPU, processor)}: the active part of the
      computer that contains the datapath and control and which adds number,
      tests numbers, signals I/O devices to acivate, and so on.

    \item \tb{datapath}: the component of the processor that performs arithmetic
      operations.

    \item \tb{control}: the component of the processor that commands the datapath,
      memory, and I/O devices according to the instructions of the program.

    \item \tb{memory}: the storage area in which programs are kept when they are
      running and that contains the data needed by the running programs.

    \item \tb{dynamic random access memory (DRAM)}: memory built as an integrated
      circuit; it provides random access to any location.

    \item \tb{cache memory}: a smalll, fast memory that acts as a buffer for
      a slower, larger memory.

    \item \tb{static random access memory (SRAM)}: memory built as an integrated
      circuit; it is faster and less dense than DRAM.

    \item \tb{instruction set architecture (architecture)}: an abstract interface
      between the the hardware and the lowest-level software that encompasses all
      the information necessary to write a machine language program that will
      run correctly, including instructions registers, memory access, I/O, and
      so on.

    \item \tb{application binary interface (ABI)}: the user portion of the
      instruction set plus the operating system interfaces used by application
      programmers. It defines a standard for binary portability accross all
      computers.

    \item \tb{main memory}: (a.k.a \tb{primary memory}) memory used to hold
      programs while they are running; typically consists of DRAM
    
    \item \tb{secondary memory}: nonviolatile memory used to store programs and data
      between runs (nonviolatile: unchanged)

    \item \tb{magnetic disk/hard disk}: a form of nonviolatile secondary memory
      composed of rotating platters coated with a magnetic recording material.

    \item \tb{flash memory}: a nonviolatile semiconductor memory
  \end{itemize}

\hi{Performance}
  \BOOKSECTION{1.6}

  \hii{Terminology}
    \begin{itemize}
      \item \tb{execution time/response time}: the time between the start and completion of a task
      \item \tb{throughput/bandwidth}: the total amount of work done in a given
        time
    \end{itemize}

  \hii{Performance}
    \par Performance is defined as:
    \begin{eqbox}
      \text{Performance}_x = \frac{1}{\text{Execution time}_x}
    \end{eqbox}
    \par Relative performance:
    \begin{eqbox}
      \ratio{\text{Performance}}{a}{b} = \ratio{\text{Execution time}}{b}{a} 
    \end{eqbox}

  \hii{CPU execution time}
    \begin{itemize}
      \item \tb{CPU execution time/CPU time}: the time the CPU spends computing
        for one task and does not include time spent waiting for I/O or running
        other programs.
      \item \tb{Clock cycle}: clock period
      \item \tb{Clock rate}: clock frequency
    \end{itemize}
    
    \begin{eqbox}
      \text{CPU time} = \text{\# Clock cycles} \times \text{Clock cycle} \\
      t = n_{cycles} \times T
    \end{eqbox}

    \begin{eqbox}
      \text{CPU time} = \frac{\text{\# Clock cycles}}{\text{Clock cycle}} \\
      t = \frac{n_{cycles}}{f}
    \end{eqbox}

  \hii{Number of Cycles for a program}
    \begin{eqbox}
      \text{\# Clock cycles}
      = \text{\# Instructions} \times \text{(average) \# Cycles per instruction}
    \end{eqbox}
    \begin{eqbox}
      \text{\# Clock cycles}
      = \text{Instruction count} \times \text{CPI}
    \end{eqbox}
    \par Composed formula: 
    \begin{eqbox}
      \text{CPU time}
      = \frac{\text{Instruction count} \times \text{CPI}}{\text{Clock rate}}
    \end{eqbox}

