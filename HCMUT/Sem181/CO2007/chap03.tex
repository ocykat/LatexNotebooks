\chapter{Arithmetic For Computers}

\hi{Introduction}
  \par This chapter deals with these issues of arithmetic for computers:
  \begin{itemize}
    \item Fractions and floating-point numbers
    \item Overflows
    \item Multiplications and Divisions
  \end{itemize}

\hi{Addition and Subtraction}
  \hii{How Addition and Subtraction works}
    \par For \tb{addition}, digits are added bit by bit from right to left, with
      carries passes to the next digit to the left.
    \par For \tb{subtraction}, the subtrahend is transformed to its two's
      complement before it is added to the minuend.

\hi{Overflow}
  \hii{When overflow occurs}
  \par \tb{Overflow} occurs when the result from an operation cannot be
    represented  with the available hardware, in this case, the 32-bit word.
  \par \tb{Overflow} cannot occur by:
    \begin{itemize} 
      \item adding operands of different sign
      \item subtracting operands of the same sign
    \end{itemize}
  \par To detect overflow, one can use the result of the operation.
    \begin{center}
    \begin{tabular}{|c|c|c|c|}
    \hline
    \tb{Operation} & \tb{A} & \tb{B} & \tb{Result indicating overflow} \\ \hline
    A + B            & +          & +          & -                     \\ \hline
    A + B            & -          & -          & +                     \\ \hline
    A - B            & +          & -          & -                     \\ \hline
    A - B            & -          & +          & +                     \\ \hline
    \end{tabular}
    \end{center}

  \hii{Overflow in programming languages}
    \par Overflow is ignored in some languages (C) while will cause exception
      in some others (Fortran).
    \par Usually, unsigned number are used for memory addresses where overflows
      are ignored. Therefore, add unsigned and add immediate unsigned do not
      cause exceptions on overlow. Only instructions working with signed
      number will.

\hi{Multiplication}
  \hii{Introduction}
    \opmul[displayintermediary=all]{1000}{1001}

    \par The first operand is called the \tb{multiplicand} and the second is
      called the \tb{multiplier}.
    \par If we ignore the sign bit, the multiplication of a n-bit multiplicand
      and a m-bit multiplier is a product that is n + m bits long.

  \hii{Algorithm and Hardware}
    \smf{\ti{Refer to page 184 for the picture of the hardware}}

    \par The multiplication algorithm is as follow:

      \begin{algorithm}[H]
        %\caption{}
        \begin{algorithmic}[1]
          \Procedure{MULTIPLY}{a, b}
            \Let{prod}{0}
            \For{i = 1 to 32}
              \If{\Call{LSB}{b} = 1}
                \Let{prod}{prod + a}
              \ENDIF
              \Let{a}{a $\SHIFTLEFT$ 1}
              \Let{b}{b $\SHIFTRIGHT$ 1}
            \ENDFOR
          \ENDPROCEDURE
        \end{algorithmic}
      \end{algorithm}

    \par A 64-bit register is required to store the multiplicand, since it has to
      be shifted left 32 times.

  \hii{Multiplication in MIPS}
    \par MIPS provides a separate pair of 32-bit registers to contain the
    64-bit product: \lstinline{HI} and \lstinline{LO}.
    \par Instructions for multiplication:
      \begin{itemize}
        \item \lstinline{mult rs, rt}: multiply two signed numbers and store the
          64-bit product in HI/LO
        \item \lstinline{multu rs, rt}: multiply two unsigned numbers and store the
          64-bit product in HI/LO
        \item \lstinline{mfhi rd}: move the value of \lstinline{HI} to \lstinline{rd}
        \item \lstinline{mflo rd}: move the value of \lstinline{LO} to \lstinline{rd}
        \item \lstinline{rd, rs, rt}: store the least-significant 32 bits of
          the product to rd.
      \end{itemize}
    \par Usually, we want a 32-bit result when we multiply two 32-bit numbers.
    In that case, we can use the HI register to check for overflow.

\hi{Division}
  \hii{Introduction}
    \opdiv[maxdivstep=4]{1001010}{1000}
    \par The first operand is called the \tb{dividend} and the second is called
    the \tb{divisor}. The result is called the \tb{quotient}, followed by a
    reminder.
    \begin{align*}
      \mbox{ Dividend } = \mbox{ Quotient } \times \mbox{ Divisor } + \mbox{
        Remainder } \qquad \mbox{ where Remainder } < \mbox{ Divisor }
    \end{align*}

  \hii{Algorithm and Hardware}
    \img[width=12cm]{img/divisor-01.png}

    \par Input:
    \begin{itemize}
      \item Dividend can be 64-bit.
      \item Divisor must be 32-bit.
      \item Quotient and remainder are 32-bit.
    \end{itemize}
    \par At the beginning:
    \begin{itemize}
      \item The 32-bit quotient register is set to 0.
      \item The divisor is placed in the left half of the 64-bit divisor
        register. In each step, the divisor is shifted right one digit to align
        with the dividend.
      \item The 64-bit remainder register is initialized with the dividend.
    \end{itemize}

    \par The division algorithm is as follow:
      \begin{algorithm}[H]
        %\caption{}
        \begin{algorithmic}[1]
          \Procedure{DIVIDE}{a, b}
            \Let{q}{0}
            \Let{r}{a}
            \Let{b}{b $\SHIFTLEFT$ 32}
            \For{i = 1 to 33}
              \Let{r}{r - a}
              \If{r $\geq$ 0}
                \Let{q}{q $\SHIFTLEFT$ 1}
                \Let{\Call{LSB}{q}}{1}
              \Else
                \Let{q}{q $\SHIFTLEFT$ 1}
                \Let{\Call{LSB}{q}}{0}
                \Let{r}{r + a}
              \ENDIF
              \Let{b}{b $\SHIFTRIGHT$ 1}
            \ENDFOR
          \ENDPROCEDURE
        \end{algorithmic}
      \end{algorithm}

  \hii{Signed Division}
    \par The correct signed division algorithm negates the quotient if the
    signs of the operands are opposite and makes the sign of the nonzero
    remainder match the dividend.

  \hii{Division in MIPS}
    \par The HI register stores the remainder and the LO register stores the
      quotient after a division.
    \par Instructions for division:
      \begin{itemize}
        \item \lstinline{div rs, rt}: does division on two signed numbers and
          store the results in HI/LO.
        \item \lstinline{divu rs, rt}: does division on two unsigned numbers and
          store the 64-bit product in HI/LO
        \item \lstinline{mfhi rd}: move the value of \lstinline{HI} to \lstinline{rd}
        \item \lstinline{mflo rd}: move the value of \lstinline{LO} to \lstinline{rd}
      \end{itemize}
    \par There is no instruction that handles divide-by-0. Therefore, the
      software must perform manual check if required.

\hi{Floating Points}
  \hii{Normalized Form}
    \par A floating-point number is in \tb{normalized form} if there is exactly
    one nonzero number to the left of the \ti{binary} point. In binary, the
    form is:
    \begin{align*}
      1.xxxxxxxx_{two} \times 2^{yyyy}
    \end{align*}
    where
    \begin{itemize}
      \item $xxxxxxx$: fraction
      \item $yyyy$: exponent
    \end{itemize}

  \hii{Representation}
    \hiii{Compromise between fraction and exponent}
      \par There must be a compromise between the size of the \tb{fraction} and
      the size of the \tb{exponent}.
      \begin{itemize}
        \item Increasing the size of the fraction enhances the precision
        \item Increasing the size of the exponent increase the range of numbers
          that can be represented.
      \end{itemize}
      \par In general, floating-point numbers are of the form:
        \begin{align*}
          (-1)^{S} \times F \times 2^{E}
        \end{align*}
        where $S$: sign, $F$: fraction, $E$: exponent.

    \hiii{Underflow and overflow}
      \par Overflow occurs when the exponent is too large to be represented in
        the exponent field.
      \par Undervlow means the nonzero fraction has become so small that it
        cannot be represented.

    \hiii{Double Precision}
      \par To reduce chances of underflow or overflow, the \tb{double
        precision} format is offered. The representation of a double precision
        floating-point number takes two MIPS words instead of one like single
        precision.

  \hii{IEEE 754 floating-point standard}
    \par Order from MSB to LSB: 1 sign bit, exponent bits, fraction bits.
    \par For \tb{single-precision} numbers:
      \begin{itemize}
        \item exponent: 8 bits
        \item fraction: 23 bits
      \end{itemize}
    \par For \tb{double-precision} numbers:
      \begin{itemize}
        \item exponent: 11 bits
        \item fraction: 20 + 32 = 52 bits
      \end{itemize}
    \par IEEE 754 also makes the leading 1-bit of normalized binary numbers
      implicit. Hence, a single precision number with 23 bits for fraction is
      24-bit long and a double precision number with 52 bits for fraction is
      53-bit long.
    \par Exponent is placed before the fraction for comparison purpose: for two
      exponent having the same sign, the number with bigger exponent is bigger.
    \par Also, negative exponent would complicate comparison. The solution is
      \tb{biased notation}. In bias notation, the exponent is replaced by
      exponent $-$ bias:
      \begin{itemize}
        \item For single precision: the bias is \tb{127}.
        \item For double precision: the bias is \tb{1023}.
      \end{itemize}
    \par The final representation of floating-point number is:
        \begin{align*}
          (-1)^{S} \times (1 + \mbox{ Fraction }) \times 2^{(\mbox{ Exponent }
          - \mbox{ Bias }}
        \end{align*}
    \par \tb{Example}: Show the IEEE 754 binary representation of the number
      $-0.75_{10}$ in single and double precision.
      \begin{itemize}
        \item Convert to binary:
          \par $-0.75_{10} = -0.11_{2}$
        \item Normalize the number:
          \par $-0.11_{2} = -0.11_{2} \times 2^{0} = -1.1_{2} \times 2^{-1}$
        \item The single precision representation of the number $-1.1_{2}
          \times 2^{-1}$ is:
          \par $(-1)^{1} \times (1 + .1000 0000 0000 0000 0000 000_{2}) \times
            2^{126 - 127}$ (23 numbers after binary point)
        \item The double precision representation of the number $-1.1_{2}
          \times 2^{-1}$ is:
          \par $(-1)^{1} \times (1 + .1000 0000 0000 0000 0000 0000 0000 0000
          0000 0000 0000 0000 0000_{2}) \times 2^{1022 - 1023}$ (52 numbers
          after binary point)
      \end{itemize}

  \hii{Range and Encoding}

  \hii{Floating-Point Addition}
    \par \tb{Algorithm}:
    \begin{algorithm}[H]
      %\caption{}
      \begin{algorithmic}[1]
        \Function{FPAdd}{addend1, addend2}
          \Let{a}{\Call{Larger}{addend1, addend2}}
          \Let{b}{\Call{Smaller}{addend1, addend2}}
          \While{b.exponent $!=$ a.exponent}
            \Let{b}{b $\SHIFTRIGHT$ 1}
          \EndWhile
          \Let{s}{a + b}
          \While{s \tb{not} normalized}
            \If{s $\geq$ 10}
              \While{s \tb{not} normalized}
                \Let{s}{s $\SHIFTRIGHT$ 1}
                \Let{s.exponent}{s.exponent + 1}
              \EndWhile
            \ElsIf{s $\leq$ 0}
              \While{s \tb{not} normalized}
                \Let{s}{s $\SHIFTLEFT$ 1}
                \Let{s.exponent}{s.exponent - 1}
              \EndWhile
            \EndIf
            \If{overflow \tb{or} underflow} \tb{throw} Exception
            \EndIf
            \State \Call{Round}{s}
          \EndWhile
          \RETURN{s}
        \EndFunction
      \end{algorithmic}
    \end{algorithm}

    \par \tb{Example}:
      \par Add $0.5_{10}$ and $-0.4375_{10}$ in binary. Expected result should
        be a single-precision number and we keep 4 bits of precision.
      \begin{itemize}
        \item Step 0: Convert to binary
          \par $0.5_{10} = 0.1_{2} = 0.1_{2} \times 2^{0} = 1 \times 2^{-1}$
          \par $-0.4375_{10} = -0.0111_{2} = -0.0111_{2} \times 2^{0} =
            -1.11_{2} \times 2^{-2}$
        \item Step 1: The sigificand of the number with the lesser exponent is
          shifted right until its exponent matches the larger number:
          \par $-1.11_{2} \times 2^{-2} = -0.111_2 \times 2^{-1}$
        \item Step 2: Add the significands:
          \par $1 \times 2^{-1} + (-0.111_{2}) \times 2^{-1} = 0.001_{2} \times
            2^{-1}$
        \item Step 3: Normalized the sum and check for overflow or underflow:
          \par $0.001_{2} \times 2^{-1} = 1 \times 2^{-4}$
          \par Since $-4$ is in the range $[-126, 127]$, there is no overflow
          or underflow. (The biased exponent must be in the range $[1, 254]$).
        \item Step 4: Round the sum.
          \par Since the sum $1.000 \times 2^{-4}$ already fits exactly in 4
          bits, there is no change to the bits due to rounding.
      \end{itemize}

  \hii{Floating-Point Multiplication}
    \par \tb{Algorithm}:
    \begin{algorithm}[H]
      %\caption{}
      \begin{algorithmic}[1]
        \Function{FPMult}{a, b}
          \Let{prod.exponent}{a.exponent + b.exponent - bias}
          \Let{prod}{a $\times$ b}
          \While{prod \tb{not} normalized}
            \If{prod $\geq$ 10}
              \While{prod \tb{not} normalized}
                \Let{prod}{prod $\SHIFTRIGHT$ 1}
                \Let{prod.exponent}{prod.exponent + 1}
              \EndWhile
            \ElsIf{prod $\leq$ 0}
              \While{prod \tb{not} normalized}
                \Let{prod}{prod $\SHIFTLEFT$ 1}
                \Let{prod.exponent}{prod.exponent - 1}
              \EndWhile
            \EndIf
            \If{overflow \tb{or} underflow} \tb{throw} Exception
            \EndIf
            \State \Call{Round}{prod}
          \EndWhile
          \RETURN{prod}
        \EndFunction
      \end{algorithmic}
    \end{algorithm}

