\hii{Dijkstra Algorithm}

  \par \ti{A lot of people can pronounce the name Van \tb{Dijk} (a football
    player) without any problem, while they would pronounce the name
    \tb{Dijk}stra incorrectly any day of the week}.

  \hiii{Introduction}

    \par The \tb{Dijkstra Algorithm} solves the single-source shortest path
      problem: Given a weighted graph and a source vertex. Find the shortest
      path from source to all other vertices in the graph.

    \par Dijkstra \tb{does not work on graph with negative weights}.

  \hiii{Notations}
    \par Given a graph $G = (V, E)$, that $\forall e = (u, v, w) \in E: w \geq
    0$, and a source vertex $s$.
    \par We denote:
    \begin{itemize}
      \item $(u, v)$ as the edge from vertex $u$ to vertex $v$.
      \item $w_{uv}$ as the weight of the edge from vertex $u$ to vertex $v$.
      \item $d[u]$ as the distance from vertex $s$ to vertex $u$.
      \item $\delta[u]$ as the \tb{minimum} distance from vertex $s$ to vertex
        $u$.
      \item $Q$ as the min-priority-queue storing the vertices keyed by the
        $d$ values. \ti{Note that all vertices are distinct}.
      \item $S$ as the set of vertices that have their min distance from source
        determined already.
    \end{itemize}

  \hiii{Pseudocode}
    \begin{algorithm}[H]
      \caption{Dijkstra}
      \begin{algorithmic}[1]
        \Function{Dijkstra}{$G$, $s$}
          \For{$u \in G$}
            \Let{$d[u]$}{$\infty$}
          \EndFor
          \State\Call{Push}{$Q$, $s$}
          \Let{$d[s]$}{$0$}
          \While{$Q \neq \emptyset$}
            \Let{$u$}{\Call{ExtractMin}{$Q$}}
            \Let{$S$}{$S \cup \set{u}$}
            \For{all vertices $v \in G.adj[u]$}
              \State\Call{Relax}{$u$, $v$, $w$}
            \EndFor
          \EndWhile
        \EndFunction

        \State

        \Function{Relax}{$u$, $v$, $w$}
          \If{$d[u] + w < d[v]$}
            \Let{$d[v]$}{$d[u] + w$}
            \If{$v \in Q$}
              \State\Call{Push}{$Q$, $v$}
            \Else
              \State\Call{DecreaseKey}{$Q$, $v$, $w$}
            \EndIf
          \EndIf
        \EndFunction
      \end{algorithmic}
    \end{algorithm}

  \hiii{Proof Of Correctness}
    \par \tb{Statement}:
    \par \fbox{
      \begin{fboxenv}
        \par For every vertices $u$ added to $S$, the distance between $u$ and
        $s$ is the shortest possible and will not be changed.
				\[
					\forall u \in S, d[u] = \delta[u]
				\]
      \end{fboxenv}
    }
    \par \tb{Proof}: By mathematical induction:
    \begin{itemize}
      \item \tb{Base case}: $|S| = 1$. This case is trivial.
      \item \tb{Inductive Step}:
        \par \ti{Inductive Hypothesis}: Suppose that the statement is true
          $\forall u \in S$ up to the previous iteration.
        \par Let $v$ be the vertex added to $S$ in the current iteration.
        \par Let $u'$ be the vertex that $v$ is relaxed from. This means that
          $u'$ has been added to $S$ in a previous iteration. According to
          the inductive hypothesis, $d[u'] = \delta[u']$. Based on the
          relaxation step, it is clear that:
          \[
            \not \exists u \in Q, u \not \equiv u':
              \delta[u] + w_{uv} < \delta[u'] + w_{u'v} = d[v]
          \]
        \par We interpret this as follow: if there exists a shorter path to
          $v$, then that path must contain at least 1 vertex not in $S$.
        \par Let $P$ be the shortest path from $s$ to $v$ that is not the
        currently explored path. This means:
				\[
          l(P) = \delta[v] < d[v] \mbox{ and } (u', v) \not \in P \eqnumber{1}
				\]
        \par $P$ can be split into two parts:
          \begin{itemize}
            \item $P_1$ which contains only vertices in $S$ and ends with vertex
              $x$.
            \item $P_2$ starts with vertex $y \not \in S$.
          \end{itemize}
        \par Because $y$ comes before $v$ on the shortest path $P$, and also
        because the graph only contains positive weights:
        \[
          \delta[y] < \delta[v] \eqnumber{2}
        \]
        \par $y$ is on the shortest path from $s$ to $v$, meaning that $P_1 +
        (x, y)$ must also be the shortest path from $s$ to $y$. At the moment,
        since $x \in S$, $d[y]$ must have been updated by relaxation from $x$.
        Therefore:
        \[
          d[y] = \delta[y] = \delta[x] + w_{xy} \eqnumber{3}
        \]
        \par On the other hand, in this iteration, $v$ is chosen from $Q$
        instead of $y$, which means:
        \[
          d[v] < d[y] \eqnumber{4}
        \]
        \par Combining the inequalities (1), (2), (3), and (4), we obtain the
        following result:
        \[
          \bm{d[v]} < d[y] = \delta[y] < \delta[v] < \bm{d[v]}
        \]
        \par By contradiction, such path $P$ does not exists. Therefore,
        $d[v] = \delta[v]$. (Q.E.D)
    \end{itemize}
