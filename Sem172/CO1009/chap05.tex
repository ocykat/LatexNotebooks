\chapter{Flip-flops and Related Devices}

\hi{Introduction}
    \par Most digital systems consist of both combinational circuits and
    memory elements.
    \par The most important memory element is the \textbf{flip-flop} (FF),
    which is made up of an assembly of logic gates.
    \par Even though a logic gate, by itself, has no storage capability,
    several can be connected together in ways that permit information
    to be stored.
    \begin{itemize}
        \item $Q$: \textit{normal} FF output
        \item $\bar{Q}$: \textit{inverted} FF output, which is always in the
        opposite state with $Q$.
        \item $Q = 1$: the \textbf{SET} state. If the input cause $Q$ to go to 1,
        we call it "setting the FF/the FF has been set".
        \item $Q = 0$: the \textbf{CLEAR/RESET} state. If the input cause $Q$ to
        go to 0, we call it "clearing or reseting the FF/the FF has been cleared
        or reset".
        \item A FF can have one or more inputs. These inputs are used to cause the
        FF to switch.
        \item Most FF inputs need only to be momentarily activated (pulsed) to switch
        the FF, and the output will remain in that new state even after the input
        pulse is over (FF's \textit{memory} characteristic).
    \end{itemize}
    \par The most basic FF circuit can be constructed from either two NAND gates or
    two NOR gates.

\hi{NAND Gate Latch}
    \begin{itemize}
        \item The two NAND gates are cross-coupled so that the \textbf{output} of
        NAND-1 is connected to one \textbf{input} of NAND-2, and vice versa
        \item There are two latch inputs:
        \begin{itemize}
            \item the SET input is the input that sets Q to the 1 state
            \item the RESET input is the input that resets Q to the 0 state.
        \end{itemize}
        \item The SET and RESET inputs are both normally resting in the HIGH state,
        and one of them will be pulsed LOW whenever we want to change the latch
        outputs.
        \item There are two different possibilities, which \textbf{depends on what
        has occurred previously at the inputs}:
        \begin{itemize}
            \item $Q = 0$ and $\bar{Q} = 1$
            \item $Q = 1$ and $\bar{Q} = 0$
        \end{itemize}
    \end{itemize}

\hi{Setting and Resetting the FF}
    \begin{itemize}
        \item If the SET input is pulsed LOW while RESET is kept HIGH, then $Q = 1$.
        We call this \textbf{setting the FF}.
        \item If the RESET input is pulsed LOW while RESET is kept HIGH, then $Q = 0$.
        We call this \textbf{clearing/resetting the FF}.
        \item If simultaneous setting and resetting happens, the result is unpredictable.
        Therefore, SET = RESET = 0 condition is normally not used for the NAND latch.
    \end{itemize}

\hi{Summary of NAND Latch}
    \begin{itemize}
        \item SET = RESET = 1: normal resting state, having no effect on the output.
        $Q$ and $\bar{Q}$ remain in whatever state they were in prior to this input
        condition.
        \item SET = 0, RESET = 1: Q = 1
        \item SET = 1, RESET = 0: Q = 0
        \item SET = 0, RESET = 0: unpredictable behaviour
    \end{itemize}

\hi{Summary of NOR Latch}
    \begin{itemize}
        \item SET = RESET = 0: normal resting state, having no effect on the output.
        $Q$ and $\bar{Q}$ remain in whatever state they were in prior to this input
        condition.
        \item SET = 1, RESET = 0: Q = 1
        \item SET = 0, RESET = 1: Q = 0
        \item SET = 1, RESET = 1: unpredictable behaviour
    \end{itemize}

\hi{Pulse}
    \par A pulse that performs its intended function when it goes HIGH is called
    a positive pulse.
    \par A pulse that performs its intended function when it goes LOW is called
    a negative pulse.
    \par In actual circuits it takes time for a pulse waveform to change from one
    level to the other. These transition times are called the rise time ($t_{r}$) and
    the fall time ($t_{f}$) and are defined as the time it takes the voltage to change
    between 10\% and 90\% of the HIGH level voltage.
    \par The transition at the beginning of the pulse is called the leading edge.
    \par The transition at the end of the pulse is the trailing edge.
    \par The duration (width) of the pulse ($t_{w}$) is defined as the time between
    the points when the leading and trailing edges are at 50\% of the HIGH level voltage.

\hi{Clock Signals and Clocked Flip-flops}
    \hii{Asynchronous and Synchronous Systems}
        \begin{itemize}
            \item In asynchronous systems, the outputs can change state any time one
              or more of the inputs change.
            \item In synchronous systems, the exact times at which any output can change
              states are determined by a signal commonly called the \textbf{clock}.
        \end{itemize}

    \hii{Clock}
        \par Most of the system outputs can change state only when the clock makes
        a \textbf{transition} (also called \textbf{edge}).
        \begin{itemize}
            \item When the clock changes from 0 to 1: \textbf{positive-going
                transition (PGT)} or \textbf{rising edge}.
            \item When the clock changes from 1 to 0: \textbf{negative-going
                transition (NGT)} or \textbf{falling edge}.
        \end{itemize}
        \par One \textbf{cycle} is measured from one PGT to the next one, or from
        one NGT to the next one.

    \hii{Clocked Flip-flops}
        \par Clocked FFs have a clock input (often labeled CLK, CK or CP).
        \par In most clocked FFs, the CLK input is edge-triggered, which means
        that it is activated by a signal transition.
        This contrasts with the latches, which are level-triggered.
        \begin{itemize}
            \item A FF with a small triangle on its CLK input to indicate
              that this input is activated only when a \textbf{PGT} occurs;
              no other part of the input pulse will have an effect
              on the CLK input.
            \item A FF with a small triangle with a bubble on its CLK input to indicate
              that this input is activated only when a \textbf{NGT}
             occurs; no other part of the input pulse will have an effect
            on the CLK input.
        \end{itemize}
        \par Clocked FFs also have one or more control inputs. The control
        inputs will have no effect on Q until the active clock transition occurs.
        They are called \textbf{synchronous control inputs}.
        \par In summary, we can say that the control inputs get the FF outputs ready
        to change, while the active transition at the CLK input actually triggers
        the change. The control inputs control the WHAT; the CLK input determines the WHEN.

    \hii{Setup and Hold time}
        \par Before and after a transition (either PGT or NGT) the signal
        must be kept at a stable value for the clocked FF to respond reliably.
        \begin{itemize}
          \item The \textbf{setup time} $t_{s}$ is the required time
              \textbf{preceding} a transition.
          \item The \textbf{hold time} $t_{H}$ is the required time
              \textbf{following} a transition.
        \end{itemize}

\hi{Clocked S-R FF}
\begin{itemize}
  \item \tb{Triggered by}: PGT
  \item \tb{Operation:}
\end{itemize}

\hi{Clocked S-R FF}
\begin{itemize}
  \item \tb{Triggered by}: PGT
  \item \tb{Operation:}
\end{itemize}

\hi{Clocked D Flip-flop}
  \hii{Triggering}
    \par A \textbf{clocked D flip-flop} triggers on a PGT.
    \par This flip-flop has only one synchronous control input, $D$ (data).
    \par Operation: $Q$ will go to the same state that is present on the $D$ input
    when a PGT occurs at CLK.
