\chapter{Combinational Logic Circuits}

\hi{Sum-of-Product form}
    \hii{Sum-of-Product form and Product-of-Sum form}
        \par A \textbf{sum-of-product (SOP)} form consists of two or more AND terms
        that are ORed together.
        \par A \textbf{product-of-sum (POS)} form consists of two or more OR terms
        that are ANDed together.
    \hii{Sum-of-Product form in Simplyfying Logic Circuits}
        \par Methods of logic-circuit simplification require the logic expression
        to be in a \textbf{sum-of-product} form.

\hi{Simplyfying Logic Circuits}
    \par There are two methods for simplyfying logic circuits:
    \begin{itemize}
        \item Utilizing Boolean algebra theorems
        \item Karnaugh mapping
    \end{itemize}

\hii{Designing Combinational Logic Circuits}
    \par The complete design procedure for a combinational logic circuit
        consists of these steps:
    \begin{itemize}
        \item Intepret the problem and set up a truth table to describe its
            operations.
        \item Write the AND (product) term for each case where the output is 1.
        \item Write the sum-of-product (SOP) expression for the output.
        \item Simplify the output expression if possible.
        \item Implement the circuit for the final, simplify expression.
    \end{itemize}

\hi{Karnaugh Map Method}
    \hi{Introduction}
        \par K map, like truth table, is a means for showing the relationship between
        the logic inputs and the desired output.
    \hi{Karnaugh Map Format}
        \begin{itemize}
            \item Each case in the truth table corresponds to a square in the K map.
            \item The K-maps squares a labeled so that horizontally adjacent squares differ
                only in one variable. Note that: each square in the top row is considered to be
                adjacent to a corresponding square in the bottom row; each square in the leftmost
                column are adjacent to corresponding squares in the rightmost column.
            \item Top-to-bottom and left-to-right labelling must be done following the rule of
                Grey code.
            \item Once a K map has been filled with 0s and 1s.
        \end{itemize}
    \hii{Complete Simplification Process}
        \begin{enumerate}
            \item Construct the K map according to the truth table.
            \item Loop isolated 1s (1s that are not adjacent to any other 1s).
            \item Look for 1s that are adjacent to only one other 1 and loop
            those pairs.
            \item Loop any octet even if it contains some 1s that have been looped.
            \item Loop any quad that contains one or more 1s that have not already been looped,
                \textit{making sure to use the minimum number of loops}.
            \item Loop any pairs necessary to include any 1s that have not yet been looped,
                \textit{making sure to use the minimum number of loops}.
            \item Form the OR sum of all the terms generated by each loop.
        \end{enumerate}

\hi{Canonical Normal Form}
    \hii{Introduction}
        \par In Boolean algebra, any Boolean function can be put into the canonical disjunctive
        normal form (CDNF) or \textbf{minterm} canonical form and its dual canonical conjunctive
        normal form (CCNF) or \textbf{maxterm} canonical form.
    \hii{Minterms}
        \par For a boolean function of $n$ variables $x_{1}, x_{2}, \ldots, x_{n}$, a product term
        in which each of the $n$ variables appears once (in either its complemented or
        uncomplemented form) is called a \textbf{minterm}.
        \par A \textbf{minterm} is a logical expression of $n$ variables that employs only the
        \textit{complement} (NOT) operator and the \textit{conjunction} (OR) operator.
        \par Sum of minterms:
        \begin{align*}
            \SUM{m(i_{1}, i_{2}, ..., i_{k})}
        \end{align*}
        where: $i_{1}, i_{2}, ..., i_{k}$ are the indices of the terms that return 1.
    \hii{Maxterms}
         \par For a boolean function of $n$ variables $x_{1}, x_{2}, \ldots, x_{n}$, a sum term
        in which each of the $n$ variables appears once (in either its complemented or
        uncomplemented form) is called a \textbf{maxterm}.
        \par A \textbf{maxterm} is a logical expression of $n$ variables that employs only the
        \textit{complement} (NOT) operator and the \textit{disjunction} (AND) operator.   
        \begin{align*}
            \PROD{M(i_{1}, i_{2}, ..., i_{k})}
        \end{align*}
        where: $i_{1}, i_{2}, ..., i_{k}$ are the indices of the terms that return 1.
