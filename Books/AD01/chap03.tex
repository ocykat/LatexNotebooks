\chapter{Order of Growth}

\hi{Asymptotic efficiency}
  \par The \tb{asymptotic} efficiency of algorithm is the efficiency
  evaluated only by the \tb{order of growth} of the algorithm while leaving
  out unnecessary precisions (multiplicative constants, lower-order terms,
  etc.)
  \par Usually, an algorithm that is asymptotically more efficient will
  be the best choice, except for the case of very small inputs.

\hi{Asymptotic notations}
  \par The notations to describe asymptotic running time of an algorithm
  are defined in terms of functions whose domains are the set of natural
  numbers $\mathbb{N}$.
  
  \hii{$\Theta$-notation}
    \begin{equation}
      \Theta(g(n)) = \{f(n) | \exists c_{1}, c_{2}, n_{0} > 0:
      0 \leq c_{1}g(n) \leq f(n) \leq c_{2}g(n) \forall n \geq n_{0}\}
    \end{equation}
    \par Intuitive understanding: if $n$ is sufficienty large, then $f(n)$
    is bounded between $c_{1}g(n)$ and $c_{2}g(n)$, that is, $f(n)$
    and $g(n)$ is nearly equivalent when it comes to order of growth.
    \par $g(n)$ is called the \tb{asymptotic tight bound} of $f(n)$.


  \hii{$O$-notation}
    \par $O(n)$ is called the \tb{asymptotic upper bound} of $f(n)$.

  \hii{$\Omega$-notation}
    \par $O(n)$ is called the \tb{asymptotic upper bound} of $f(n)$.

  \hii{Theorem}
    \par \ti{For any two functions $f(n)$ and $g(n)$, we have
      $f(n) = \Theta (g(n))$ if and only if $f(n) = O(g(n))$ and 
      $f(n) = \Omega(g(n))$}.

