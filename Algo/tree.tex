\chapter{Tree}

\hi{Basics}
  \hii{Definition}
    \par A \tb{tree} is a \tb{connected graph} with \tb{no cycle}.
  \hii{Terminology}
    \begin{itemize}
      \item \tb{Internal node}: any node that has children
    \end{itemize}


\hi{Binary Search Tree}
  \hii{Definition}
    \par A \tb{Binary Search Tree (BST)} is a tree that has satisfy the
      following condition:
    \par Let $x$, $y$ be a nodes in a binary search tree. 
      \begin{itemize}
        \item If $y$ is in the left subtree of $x$, then $y.key \leq x.key$.
        \item If $y$ is in the right subtree of $x$, then $y.key \geq x.key$.
      \end{itemize}
  \hii{Time Complexity}
    \par Let $h$ be the height of the BST.
      \begin{center}
        \begin{tabular}{|l|l|}
        \hline
               &        \\ \hline
        search & $O(h)$ \\ \hline
        insert & $O(h)$ \\ \hline
        remove & $O(h)$ \\ \hline
        \end{tabular}
      \end{center}


\hi{AVL Tree}
  \hii{Some notes}
    \begin{itemize}
      \item A tree is \tb{balanced} if $h = \theta(\log(n))$.
      \item The \tb{height of a node} is the length of the longest path
      from it down to a leaf.
      \item The height of the tree is equals to the height of its root.
      \begin{align*}
        h(\mbox{node}) = max(h(\mbox{left child}), h(\mbox{right child})) + 1
      \end{align*}
      \item To avoid dealing with the special case, children of leaf have
        height equals to $-1$.
    \end{itemize}
  \hii{Definition}
    \par An \tb{AVL tree} require heights of left and right children of
      every node to differ by at most $\pm 1$.
  \hii{Proof for Balance property}
    \par Worst case for AVL tree: right subtree has height 1 more than
      left subtree for every node. (this worst case is derived from the
      definition)
    \par Define $n_{h}$ as the fewest node in an AVL tree of height $h$.
    \par Formula for $n_{h}$:
    \begin{align*}
      \begin{cases}
        n_{1} = 1 \\
        n_{h} = 1 + n_{h - 1} + n_{h - 2} \\
      \end{cases} \\
    \end{align*}
    \par Here we have 2 methods to deal with the recurrence:
    \begin{itemize}
      \item Method 1: Using Fibonacci sequence
        \begin{flalign*}
          & n_{h} = 1 + n_{h - 1} + n_{h - 2} && \\
          & \Rightarrow n_{h} > fib_{h}
            = fib_{h - 1} + fib_{h - 2} && \\
          & \Rightarrow n_{h} > \frac{\Phi^{h}}{\sqrt{5}} && \\
          & \Rightarrow n_{h} > \frac{\Phi^{h}}{\sqrt{5}} && \\
          & \Rightarrow n_{h}\sqrt{5} > \Phi^{h} && \\
          & \Rightarrow h < \log_{\Phi}(n_{h}\sqrt{5}) &&
        \end{flalign*}
        where $\Phi \approx 1.618$ is the golden ratio.
      \item Method 2
        \begin{flalign*}
          & n_{h} = 1 + n_{h - 1} + n_{h - 2} && \\
          & \Rightarrow n_{h} > 2n_{h - 2} = \Theta(2^{h/2}) && \\
          & \Rightarrow h < 2\log_{2}(n) &&
        \end{flalign*}
    \end{itemize}
  \hii{Operations}
    \begin{itemize}
      \item \tb{Insert}
        \par Steps:
        \begin{itemize}
          \item BST insert
          \item Fix AVL property
        \end{itemize}
    \end{itemize}


\hi{Red-Black Tree}
  \hii{Definition}
    \par A \tb{red-black tree} is a \tb{binary tree} such that:
    \begin{itemize}
      \item Every node is red or black.
      \item The root is black.
      \item Leaves (NIL) are black.
      \item If a node is red, its children are all black.
      \item For each node, all simple paths from it to its decendant leaves
        contain the same number of black nodes.
    \end{itemize}
  \hii{Terminology}
    \par \tb{black-height}: the number of black notes on any simple path
      from, but \ti{not including}, a node $x$ down to a leaf.
  \hii{Maximum Height}
    \par \tb{Lemma}: For a red black tree with $n$ internal nodes,
      $h \leq 2\log(n + 1)$.
    \par \tb{Proof}:
    \par First, we need to prove that: a tree with $n$ nodes contains
    at least $2^{bh(n)} - 1$ internal nodes.
    \par By induction:
      \begin{itemize}
        \item \tu{Basis step}:
          \par $bh(n) = 0$. If $bh(n) = 0$, then the tree has only a leaf,
          containing $2^{bh(n)} - 1 = 2^{0} - 1 = 0$ internal nodes.
        \item \tu{Inductive Hypothesis}:
          \par Consider a tree with $n$ nodes. Then each of its subtree
          (left and right) would have black-height of either
          $bh(n)$ or $bh(n) - 1$ depends on
          whether the subtree's root is red or black, respectively.
          \par Apply the inductive hypothesis: each subtree has at least 
          $2^{bh(n) - 1} - 1$ internal nodes. Therefore, the tree
          contains at least $2 \times (2^{bh(n) - 1} - 1) + 1 = 2^{bh(n) - 1}$
          internal nodes.
      \end{itemize}
    \par Let $h$ be the height of the tree.
    \par Because a red node has all black children, $bh(n) \geq h/2$.
      \begin{flalign*}
        & \Rightarrow n \geq 2^{bh(n)} - 1 \geq 2^{h/2} - 1  &&\\
        & \Rightarrow 2^{h/2} \leq n + 1 &&\\
        & \Rightarrow h/2 \leq \log_{2}(n + 1) &&\\
        & \Rightarrow h \leq 2\log_{2}(n + 1) &&\\
      \end{flalign*}


\hi{Fenwick Tree (Binary Indexed Tree)}
  \hii{When to use}
    \par Fenwick Tree is useful when we need to do two types of query:
    \begin{itemize}
      \item updating an element
      \item calculate the prefix sum upto element $i^{th}$
    \end{itemize}
    \par Denote the prefix sum upto $i$ as $query(i)$.
    \par Denote the value stored at position $i$ in the array as $tree(i)$.

  \hii{General Idea}
    \hiii{Naive query approach}
      \par The naive approach is to accumulate the sum as follow:
      \begin{align*}
        a[i] += a[i - 1] \quad (2 \leq i \leq n)
      \end{align*}
      \par Query would cost $O(1)$ but update would cost $O(n)$.
      \par To reduce update time downto $O(log(n))$, we need a better way to store the accumulated sum.
    \hiii{Fenwick Tree Query}
      \par To calculate $query(i)$, we consider the following fact:
      \par Any number $i$ can be expressed as follow:
      \begin{align*}
        i = 2^{j_{n}} + 2^{j_{n - 1}} + \ldots + 2^{j_{1}} \quad (j_{n} > j_{n - 1} > \ldots > j_{1}) \quad (*)
      \end{align*}
      \par First, set the result to 0.
      \par Then, step by step:
      \begin{itemize}
        \item Add $tree(2^{j_{n}} + 2^{j_{n - 1}} + \ldots + 2^{j_{1}})$ to the result.
        \item Add $tree(2^{j_{n}} + 2^{j_{n - 1}} + \ldots + 2^{j_{2}})$ to the result.
        \item Add $tree(2^{j_{n}})$ to the result.
      \end{itemize}
      \par Each time, the tree index is reduced by the smallest factor of 2 remaining
        according to the sum (*). This factor can be extracted using the bitwise trick
        mentioned later.

  \hii{Bitwise trick}
    \begin{itemize}
      \item Every number can be represented as the sum of powers of two.
      \item Given a number $x$. The operation $x \& (-x)$ returns the last 1-bit
      together with all the 0-bits that comes after it.
    \end{itemize}
    \par \ti{Proof for the second statement}:
    \par Represent the number $x$ in the form of $a1b$ where $b$ contains only
      zeros. Then:
    \begin{flalign*}
      & (-x) = (a\bm{1}b)' + 1 = a'\bm{1}0\ldots01 && \\
      & \Rightarrow x \& (-x) = [a \& a']\bm{1}[0\ldots0 \& 0\ldots1] && \\
      & \Rightarrow x \& (-x) = \bm{1}0\ldots0 &&
    \end{flalign*}


\hi{Segment Tree}
  \hii{Structure}
    \par A segment tree
