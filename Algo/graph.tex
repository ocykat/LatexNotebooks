\chapter{Graph}


\hi{Graph}
  \hii{Definition} 
    \par A \tb{graph} $G = (V, E)$ consists of:
    \begin{itemize}
      \item $V$: a \ti{non-empty} set of vertices (nodes)
      \item $E$: a set of edges
    \end{itemize}
    \par Each vertex associating with an edge is called an \tb{endpoint}.
  \hii{Undirected Graph}
    \hiii{Degree of Vertex}
    \par The \tb{degree} of a vertex is the number of edges incident with it.
    \par \tb{Handshaking theorem}
    \begin{eqbox}
      \SUM_{v \in V} deg(V) = 2\abs{E}
    \end{eqbox}
  \hii{Directed Graph}
    \hiii{Conectivity}
      \par A directed graph is \tb{strongly connected} if there is a path from
        $u$ to $v$ every ordered-pair $(u, v)$ in the graph.
  \hii{Graph Representations}
    \par There are two common graph representations:
      \begin{itemize}
        \item adjacency-list representation
        \item adjacency-matrix representation
      \end{itemize}
    \par Adjacency-list representation is better when the graph is \tb{sparse},
      while the adjacency-matrix representation is better when the graph is
      \tb{dense}.
    \par A graph is dense if $\abs{E}$ is close to $\abs{V^{2}}$.
  \hii{Cut}
    \par A \tb{cut} is a partition of the vertices of a graph into two disjoint
      subsets $G_{1}$ and $G_{2}$.
    \par Any cut determines a \tb{cut set} $C$ such that:
    \begin{align*}
      \forall (u, v) \in C: (u \in G_{1} \land v \in G_{2}) \lor (u \in G_{2} \land v \in G_{1})
    \end{align*}


\hi{Breadth-First Search (BFS)}
  \hii{Time Complexity}
    \begin{itemize}
      \item Queue operations: $O(V)$
      \item Traversing adjacency lists: $O(E)$
    \end{itemize}
    \par Total time complexity: $O(V + E)$

  \hii{Theorem on correctness of BFS and shortest path}
    \par Define $\delta(s, v)$ as the \tb{shortest-path distance} from
      $s$ to $v$ as the minimum number of edges in any path from $s$ to $v$.
      If there is no path from $s$ to $v$, $\delta(s, v) = \infty$.
    \par \tb{Theorem}:
      \par Let $G = (V, E)$ be a directed graph. Suppose that BFS is run on G
        from a given source vertex $s \in V$. Then during its execution, BFS
        discovers every vertex $v \in V$ that is reachable from the source $s$,
        and upon termination, $\forall v \in V: v.d = \delta(s, v)$.


\hi{Depth-First Search (DFS)}
  \hii{Edge types}
    \begin{itemize}
      \item \tb{Tree edge}: an edge in the DF forest. More specifically,
        $(u, v)$ is a tree edge if $v$ was first discovered by $(u, v)$.
      \item \tb{Back edge}: any edge $(u, v)$ which has $u$ as a descendant of $v$.
      \item \tb{Forward edge}: any \ti{nontree} edge $(u, v)$ which has $u$ as
        an ancestor of $v$.
      \item \tb{Cross edge}: every edge which does not fall into the 3
        mentioned categories. A cross edge can be edge between vertices in
        the same tree with no relationship, or vertices of different DF trees,
    \end{itemize}

  \hii{Theorems}
    \hiii{Parenthesis theorem}
      \par In a DFS of a graph $G = (V, E)$ (di/undi), $\forall (u, v)$:
        exactly one of the three conditions holds:
        \begin{itemize}
          \item $[u.enter, u.leave] \disjoint [v.enter, v.leave]$
          \item $[u.enter, u.leave] \subset [v.enter, v.leave]$
          \item $[v.enter, v.leave] \subset [u.enter, u.leave]$
        \end{itemize}
      \par Corollary: $v$ is a \ti{proper descendant} of $u$ in the DF forest
        for a graph $G$ iff
        \begin{align*}
          u.enter < v.enter < v.leave < u.leave
        \end{align*}

    \hiii{White-path theorem}
      \par In a DF forest of a graph $G$ (undi/di), $v$ is a descendant of $u$
        iff at $u.enter$ there is a path $u \to v$ consists entirely of white
        vertices (vertices that have not been visited by DFS).

    \hiii{Edge Type theorem}
      \par In a DFS of an undirected graph $G$, every edge of $G$ is either a
        tree edge or a back edge.


\hi{Strongly Connected Component}
  \hii{Definition}
    \par A \tb{strongly connected component} (SCC) of a directed graph
      $G = (V, E)$ is a maximal set of vertices $C \subseteq V$ such that for
      every pair of vertices $(u, v)$, we have both paths $u \to v$ and $v \to u$.

  \hii{Component Graph}
    \par A \tb{component graph} $G^{SCC} = \{ V^{SCC}, E^{SCC} \}$ is a graph
      obtained from $G$ by compressing every SCC into one vertex.
    \par \tb{Property}: A component graph is a \tb{DAG}.
    \par \tb{Lemma}: Let $C$ and $C'$ be distinct SCCs in $G$. Let $u, v \in C$,
      $u', v' \in C'$. Suppose that $G$ contains a path $u \leadsto u'$.
      Then $G$ cannot also contain a path $v' \to v$.
      \begin{smfont}
        \par \tb{Proof}: If there is a path $u \leadsto u'$, then there is also
          a path $u \leadsto v'$. If there is a path $v' to v$, then $C$ and
          $C'$ are strongly connected. This contradicts the assumption that $C$
          and $C'$ are distinct.
      \end{smfont}
    \par Define $C.enter = x_{0}.enter$ where $x_{0} \in C$ and $x_{0}.enter$
      is \tb{minimum}.
    \par Define $C.leave = x_{0}.leave$ where $x_{0} \in C$ and $x_{0}.enter$
      is \tb{maximum}.
    \par \tb{Lemma}: Let $C$ and $C'$ be distinct SCCs. Suppose that there is
      an edge $u \to v$ where $u \in C$ and $v \in C'$. Then
      $C.leave > C'.leave$.
      \begin{smfont}
        \par \tb{Proof}: There are two different cases:
          \begin{itemize}
            \item $C.enter < C'.enter$: Let $x_{0}$ be the first vertex
              discovered in $C$. At $x_{0}.enter$, the path from $x$ to any
              vertex in $C$ or $C'$ consists of only white vertices. By the
              white path theorem, $x_{0}$ is the ancestor of all vertices in
              $C$ and $C'$. Therefore, $C.leave = x_{0}.leave > C'.leave$.
          \end{itemize}
      \end{smfont}


\hi{Minimum Spanning Tree}
  \hii{Cut property}
    \par For any cut $C$ of the graph, if the weight of an edge $e$ in the cut set of $C$
      is strictly smaller than the weights of all other edges of the cut-set of $C$, then
      this edge belongs to all MSTs of the graph.
    \par \tb{Proof}:
    \begin{smfont}
      \par Define $T$ as the MST of $G$. Assume that $e \not \in T$.
      \par The cut $C$ partition the graph into $G_{1}$ and $G_{2}$.
      If $e \not \in T$, then $\exists e' \in C: e' \not \equiv e \land e' \in T$ connecting $G_{1}$
      and $G_{2}$. By adding $e$ and removing $e'$ from $T$, we get a MST that
      is strictly smaller weight than $T$. This contradicts the assumption.
    \end{smfont}
