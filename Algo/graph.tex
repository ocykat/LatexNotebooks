\chapter{Graph}


\hi{Graph}
  \hii{Definition} 
    \par A \tb{graph} $G = (V, E)$ consists of:
    \begin{itemize}
      \item $V$: a \ti{non-empty} set of vertices (nodes)
      \item $E$: a set of edges
    \end{itemize}
    \par Each vertex associating with an edge is called an \tb{endpoint}.
  \hii{Undirected Graph}
    \hiii{Degree of Vertex}
    \par The \tb{degree} of a vertex is the number of edges incident with it.
    \par \tb{Handshaking theorem}
    \begin{eqbox}
      \SUM_{v \in V} deg(V) = 2\abs{E}
    \end{eqbox}
  \hii{Directed Graph}
    \hiii{Conectivity}
      \par A directed graph is \tb{strongly connected} if there is a path from
        $u$ to $v$ every ordered-pair $(u, v)$ in the graph.
  \hii{Graph Representations}
    \par There are two common graph representations:
      \begin{itemize}
        \item adjacency-list representation
        \item adjacency-matrix representation
      \end{itemize}
    \par Adjacency-list representation is better when the graph is \tb{sparse},
      while the adjacency-matrix representation is better when the graph is
      \tb{dense}.
    \par A graph is dense if $\abs{E}$ is close to $\abs{V^{2}}$.


\hi{Breadth-First Search (BFS)}
  \hii{Theorem on correctness of BFS}
    \par Define $\delta(s, v)$ as the \tb{shortest-path distance} from
      $s$ to $v$ as the minimum number of edges in any path from $s$ to $v$.
      If there is no path from $s$ to $v$, $\delta(s, v) = \infty$.
    \par \tb{Theorem}:
      \par Let $G = (V, E)$ be a directed graph. Suppose that BFS is run on G
        from a given source vertex $s \in V$. Then during its execution, BFS
        discovers every vertex $v \in V$ that is reachable from the source $s$,
        and upon termination, $v.d = \delta(s, v) \forall v in V$.
